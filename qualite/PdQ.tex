%!TEX TS-program = xelatex
%!TEX encoding = UTF-8 Unicode

\documentclass[a4paper,11pt,french]{article}

%Import des packages utilisés pour le document
\usepackage[french]{babel}
\usepackage{chngpage}
\usepackage[colorlinks=true,linkcolor=black,urlcolor=blue]{hyperref}
\usepackage{graphicx, amssymb, color, listings}
\usepackage{fontspec,xltxtra,xunicode,color}
\usepackage{tabularx, longtable}
\usepackage[table]{xcolor}
\usepackage{fancyhdr}
\usepackage{tikz}
\usetikzlibrary{shapes}
\usepackage{lastpage}

\definecolor{gris}{rgb}{0.95, 0.95, 0.95}

%Redéfinition des marges
\addtolength{\hoffset}{-2cm}
\addtolength{\textwidth}{4cm}
\addtolength{\topmargin}{-2cm}
\addtolength{\textheight}{1cm}
\addtolength{\headsep}{0.8cm} 
\addtolength{\footskip}{1cm}


%Import page de garde et structures pour la gestion de projet
\usepackage{res/structures} 

%Variables
\def\matiere{Conduite de Projet}
\def\filiere{Master 2 SSI}
\def\projectDesc{Smart Social Network}
\def\projectName{\emph{SSN}~}
\def\completeName{\projectDesc ~- \projectName}
\def\docType{Plan de Qualité}
\def\docDate{\today}
\def\version{0.1}
\def\author{Baptiste \textsc{Dolbeau}}
\def\checked{}
\def\approved{}


% -- Début du document -- %
\begin{document}
%Page de garde
\makeFirstPage
\clearpage

%Tableau de mises à jour
\vspace*{1cm}
\begin{center}
\textbf{\huge{MISES À JOUR}}\\
\vspace*{3cm}
	\begin{tabularx}{16cm}{|c|c|X|}
	\hline
	\bfseries{Version} & \bfseries{Date} & \bfseries{Modifications réalisées}\\
	\hline
	0.1 & 31/01/2013 & Création\\
	\hline
	&&\\
	\hline
	&&\\
	\hline
	\end{tabularx}
\end{center}

%La table des matières
\clearpage
\tableofcontents
\clearpage

\renewcommand\labelitemi{\textbullet} %style des puces
\renewcommand\labelitemii{$\circ$} %style des puces 2e niveau
\section{Objet du Plan de Qualité}
\subsection{Cadre}
	Ce plan de qualité a été élaboré pour le projet \emph{Smart Social
Network} associant la promotion de master 2 Sécurité des Systèmes
Informatiques avec Madame Bardet ainsi que Monsieur Otmani.

\subsection{Objectif}
	Il a pour but de mettre en évidence la définition et les mesures à
prendre pour le projet \emph{Smart Social Network} afin d'assurer un seuil de
qualité défini et atteindre les résultats demandés.

\subsection{Utilisation}
	Le plan de qualité est un outil de suivi et de gestion du projet.
De part sa nature, il est amené à être modifié selon les ajouts que l'on
apportera au projet.

\newpage
\section{Données}
\subsection{Contexte}
	Dans le cadre de notre master 2 Sécurité des Systèmes Informatiques,
nous avons à réaliser un projet de fin d'année. Ce projet \emph{Smart Social
Network} est devenu l'agrégat des deux sous-projets initiaux \emph{Smart
Card} et \emph{Secure Social Network}. L'effectif n'étant pas assez 
suffisant, il a été convenu par les deux clients "Mme Bardet" et "M Otmani"
de réunir ces deux projets en un seul.


	Ces projets sont nés d'une part du besoin d'un cas d'utilisation 
de cartes à puce et d'autre part du besoin de rester propriétaire des
données postées sur un réseau social.


\subsection{Rappels}
\subsubsection{Objectifs projet}
	Notre projet possède deux objectifs principaux :
\begin{itemize}
	\item Authentification d'une personne via carte à puce.
	\item Chiffrement des données déposées sur le réseau social
"Facebook".
\end{itemize}

	Ainsi, \emph{Smart Social Network} fournit une solution utilisateur
en permettant de rendre les informations illisibles pour leur hébergeur, ici
Facebook, au moyen d'une carte à puce.

\subsubsection{Livrables}
\begin{itemize}
	\item Void.
	\item Void.
	\item Void.
	\item Projet Fini.
\end{itemize}

\subsubsection{Objectifs produit}
	Ces deux objectifs s'articulent dans Smart Social Network dans le
sens où, une fois la personne authentifiée, la carte à puce sera utilisée
pour fournir des éléments utilisés pour le chiffrement et déchiffrement
des données utilisateurs de Facebook.

\subsubsection{Limites}
	Ce projet, bien que pensé pour fonctionner avec plusieurs réseaux
sociaux, est implanté, dans notre cas, uniquement pour Facebook. De plus,
il ne fontionnera que sur un navigateur : \emph{Firefox}.


	Un utilisateur \emph{Smart Social Network} pourra poster des données
telles que des images, des messages ou des commentaires, tout ceci de manière
chiffrée. Ainsi, ces données ne seront pas lisibles par Facebook ou les
autres personnes non spécifiées. Cependant, la liste de ses amis, les
conversations instantanées ainsi que ses données personnelles pourront être
lues par Facebook.

\subsection{Moyens}
	Pour la réalisation de ce projet, nous avons mis en place plusieurs
concept afin d'améliorer l'efficacité de notre travail :
\begin{itemize}
	\item Un espace partagé de travail (github)
	\item Un environnement de developpement (eclipse)
	\item Un kit de developpement pour firefox (addon-sdk)
\end{itemize}

	De plus, nous avons reçu des lecteurs de cartes à puces ainsi qu'un
lot de cartes.


\subsection{Organismes, parties prennantes}
	Les parties prennantes du projet sont la promotion du master 2 SSI
en tant que fournisseurs et Mme Bardet ainsi que M Otmani en tant que clients.


\newpage

\section{Organisation}
\subsection{Structure}
	Afin de réaliser notre projet, nous avons décidé de garder la même
structure présente avant la réunion des deux projets. Il y a donc une équipe
spécialisé pour l'authentification par carte à puce (équipe \emph{Smart 
card}) et une équipe pour le chiffrement des données sur le réseau social
(équipe \emph{Secure Social Network}). Le chef de projet aura pour rôle
la coordination de ces deux équipes.

\subsection{Acteurs}
\subsubsection{Organismes participants}
\begin{center}
	\begin{tabularx}{16cm}{|X|X|X|X|}
	\hline
	\bfseries{Organisme} & \bfseries{Nom} & \bfseries{Fonctions} & \bfseries{Rôle}\\
	\hline
	Université de Rouen & Mme Bardet & Professeur & Client\\
	\hline
	Université de Rouen & M Otmani & Professeur & Client\\
	\hline
	Université de Rouen & Promotion M2SSI & Etudiants & Fournisseurs\\
	\hline
	\end{tabularx}
\end{center}
\vspace*{0cm}

\subsubsection{Liste des acteurs de la promotions}
\begin{center}
	\begin{tabularx}{16cm}{|X|X|X|X|}
	\hline
	\bfseries{Equipe} & \bfseries{Nom} & \bfseries{Rôle}\\
	\hline
		& Guilbert Florian & Chef de Projet\\
	\hline
	Secure Social Network & Addi Zakaria & Developpeur\\
	\hline
	Secure Social Network & Péchoux Maxence & Developpeur\\
	\hline
	Secure Social Network & Dolbeau Baptiste & Testeur\\
	\hline
	Smart Card & Mocquet Emmanuel & Developpeur\\
	\hline
	Smart Card & Yicheng Gao & Developpeur\\
	\hline
	Smart Card & Pignard Romain & Testeur\\
	\hline
	Smart Card & Huet Giovanni & Testeur\\
	\hline
	\end{tabularx}
\end{center}
\vspace*{0cm}

\subsection{Communications}
\subsubsection{Documentation}
\begin{itemize}
	\item STB Secure Social Network
	\item DAL Secure Social Network
	\item Tutoriel d'utilisation
	\item Plan de developpement
	\item Manuel d'utilisation
	\item Analyse des risques
	\item Cahier de recette
	\item Plan de qualité
	\item STB Smart Card
	\item DAL Smart Card
	\item Comptes rendus
\end{itemize}

	Chaque document est soumis à une relecture par une personne tierce.
Le document n'est pas présenté au client tant qu'il y a des remarques faites
par la personne tierce. 

	Une fois présenté au client, le document peut être accepté directement
sinon, celui ci est a nouveau révisé en fonction des remarques du client.


\newpage
\section{L'assurance qualité}





\end{document}
