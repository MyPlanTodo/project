%!TEX TS-program = xelatex
%!TEX encoding = UTF-8 Unicode

\documentclass[a4paper,11pt,french]{article}

%Import des packages utilisés pour le document
\usepackage[french]{babel}
\usepackage{chngpage}
\usepackage[colorlinks=true,linkcolor=black,urlcolor=blue]{hyperref}
\usepackage{graphicx, amssymb, color, listings}
\usepackage{fontspec,xltxtra,xunicode,color}
\usepackage{tabularx, longtable}
\usepackage[table]{xcolor}
\usepackage{fancyhdr}
\usepackage{tikz}
\usetikzlibrary{shapes}
\usepackage{lastpage}

\definecolor{gris}{rgb}{0.95, 0.95, 0.95}

%Redéfinition des marges
\addtolength{\hoffset}{-2cm}
\addtolength{\textwidth}{4cm}
\addtolength{\topmargin}{-2cm}
\addtolength{\textheight}{1cm}
\addtolength{\headsep}{0.8cm} 
\addtolength{\footskip}{1cm}


%Import page de garde et structures pour la gestion de projet
\usepackage{res/structures} 

%Variables
\def\matiere{Conduite de Projet}
\def\filiere{Master 2 SSI}
\def\projectDesc{Smart Social Network}
\def\projectName{\emph{SSN}~}
\def\completeName{\projectDesc ~- \projectName}
\def\docType{Plan de Qualité}
\def\docDate{\today}
\def\version{0.1}
\def\author{Baptiste \textsc{Dolbeau}}
\def\checked{}
\def\approved{}


% -- Début du document -- %
\begin{document}
%Page de garde
\makeFirstPage
\clearpage

%Tableau de mises à jour
\vspace*{1cm}
\begin{center}
\textbf{\huge{MISES À JOUR}}\\
\vspace*{3cm}
	\begin{tabularx}{16cm}{|c|c|X|}
	\hline
	\bfseries{Version} & \bfseries{Date} & \bfseries{Modifications réalisées}\\
	\hline
	0.1 & 31/01/2013 & Création\\
	\hline
	&&\\
	\hline
	&&\\
	\hline
	\end{tabularx}
\end{center}

%La table des matières
\clearpage
\tableofcontents
\clearpage

\renewcommand\labelitemi{\textbullet} %style des puces
\renewcommand\labelitemii{$\circ$} %style des puces 2e niveau
\section{Objet du Plan de Qualité}
\subsection{Cadre}
	Ce plan de qualité a été élaboré pour le projet \emph{Smart Social
Network} associant la promotion de master 2 Sécurité des Systèmes
Informatiques avec Madame Bardet ainsi que Monsieur Otmani.

\subsection{Objectif}
	Il a pour but de mettre en évidence la définition et les mesures à
prendre pour le projet \emph{Smart Social Network} afin d'assurer un seuil de
qualité défini et atteindre les résultats demandés.

\subsection{Utilisation}
	Le plan de qualité est un outil de suivi et de gestion du projet.
De part sa nature, il est amené à être modifié selon les ajouts ou 
suppressions que l'on apportera au projet.

\section{Données}
\subsection{Contexte}
	Dans le cadre de notre master 2 Sécurité des Systèmes Informatiques,
nous avons à réaliser un projet de fin d'année. Ce projet \emph{Smart Social
Network} est devenu l'agrégat des deux projets initiaux \emph{Smart Card} et
\emph{Secure Social Network}. L'effectif n'étant pas assez suffisant, il a
été convenu par les deux clients "Mme Bardet" et "M Otmani" de réunir ces
deux projets en un seul.


	Ces projets sont nés d'une part du besoin d'un cas d'utilisation de
cartes à puce et d'autre part du besoin de rester propriétaire des données
postées sur un réseau social.


\subsection{Rappels}
\subsubsection{Objectifs projet}
	Notre projet possède deux objectifs principaux :
\begin{itemize}
	\item Authentification d'une personne via carte à puce.
	\item Chiffrement des données déposées sur le réseau social
"Facebook".
\end{itemize}

	Ainsi, \emph{Smart Social Network} fournit une solution utilisateur
en permettant de rendre les informations illisibles pour leur hébergeur, ici
Facebook, au moyen d'une carte à puce.

\subsubsection{Livrables}
\begin{itemize}
	\item Trois applications permettants respectivement de : générer des
nombres pseudo-aléatoires, de chiffrer/déchiffrer des données et d'un add-on
communiquant avec le navigateur firefox.
	\item Stockage de données sur la carte, injection de code sur le site
du réseau social grâce à l'extension et stockage permanent d'informations sur
la troisième application.
	\item Les trois applications fonctionnent et peuvent dialoguer entre
elles à l'aide de "tunnels" sécurisés
	\item Projet Fini.
\end{itemize}

\subsubsection{Objectifs produit}
	Ces deux objectifs s'articulent dans \emph{Smart Social Network} dans
le sens où, une fois la personne authentifiée, nous utiliserons la carte à
puce pour fournir des éléments utilisés pour le chiffrement et déchiffrement
des données utilisateurs de Facebook.

\subsubsection{Limites}
	Ce projet, bien que pensé pour fonctionner avec plusieurs réseaux
sociaux, est implanté, dans notre cas, uniquement pour Facebook. De plus,
il ne fontionnera que sur un navigateur : \emph{Firefox}.


	Un utilisateur de \emph{Smart Social Network} pourra poster des
données telles que des images, des messages ou des commentaires, tout ceci de
manière chiffrée. Ainsi, ces données ne seront lisibles ni par Facebook ni par
les autres personnes non spécifiées. Cependant, la liste de ses amis, les
conversations instantanées ainsi que ses données personnelles ne seront pas
prises en compte par notre système et pourront donc être lues par Facebook.

\subsection{Moyens}
	Pour la réalisation de ce projet, nous avons mis en place plusieurs
concept afin d'améliorer l'efficacité de notre travail :
\begin{itemize}
	\item Un espace partagé de travail (github)
	\item Un environnement de developpement (eclipse)
	\item Un kit de developpement pour firefox (addon-sdk)
\end{itemize}

	De plus, nous avons reçu des lecteurs de cartes à puces ainsi qu'un
lot de cartes.


\subsection{Organismes, parties prennantes}
	Les parties prennantes du projet sont la promotion du master 2 SSI
en tant que fournisseurs et Mme Bardet ainsi que M Otmani en tant que clients.



\section{Organisation}
\subsection{Structure}
	Afin de réaliser notre projet, nous avons décidé de garder la même
structure présente avant la réunion des deux projets. Il y aura donc une
équipe spécialisée pour l'authentification par carte à puce (équipe 
\emph{Smart card}) et une équipe pour le chiffrement des données sur le
réseau social (équipe \emph{Secure Social Network}). Le chef de projet aura
pour rôle la coordination de ces deux équipes.

\subsection{Acteurs}
\subsubsection{Organismes participants}
\begin{center}
	\begin{tabularx}{16cm}{|X|X|X|X|}
	\hline
	\bfseries{Organisme} & \bfseries{Nom} & \bfseries{Fonctions} & \bfseries{Rôle}\\
	\hline
	Université de Rouen & Mme Bardet & Professeur & Client\\
	\hline
	Université de Rouen & M Otmani & Professeur & Client\\
	\hline
	Université de Rouen & Promotion M2SSI & Etudiants & Fournisseurs\\
	\hline
	\end{tabularx}
\end{center}
\vspace*{0cm}

\subsubsection{Liste des acteurs de la promotions}
\begin{center}
	\begin{tabularx}{16cm}{|X|X|X|X|}
	\hline
	\bfseries{Equipe} & \bfseries{Nom} & \bfseries{Rôle}\\
	\hline
		& Guilbert Florian & Chef de Projet\\
	\hline
	Secure Social Network & Addi Zakaria & Developpeur\\
	\hline
	Secure Social Network & Péchoux Maxence & Developpeur\\
	\hline
	Secure Social Network & Dolbeau Baptiste & Testeur\\
	\hline
	Smart Card & Mocquet Emmanuel & Developpeur\\
	\hline
	Smart Card & Yicheng Gao & Developpeur\\
	\hline
	Smart Card & Pignard Romain & Testeur\\
	\hline
	Smart Card & Huet Giovanni & Testeur\\
	\hline
	\end{tabularx}
\end{center}
\vspace*{0cm}

\subsection{Communications}
\subsubsection{Documentation}
\begin{itemize}
	\item STB Secure Social Network
	\item DAL Secure Social Network
	\item Tutoriel d'utilisation
	\item Plan de developpement
	\item Manuel d'utilisation
	\item Analyse des risques
	\item Cahiers de recette
	\item Plan de qualité
	\item STB Smart Card
	\item DAL Smart Card
	\item Comptes rendus
\end{itemize}

	Chaque document est soumis à une relecture par une personne tierce.
Le document n'est pas présenté au client tant qu'il y a des remarques faites
par la personne tierce.

	Une fois présenté au client, le document peut être accepté directement
sinon, celui ci est à nouveau révisé en fonction des remarques du client.


\section{L'assurance qualité}

	À partir de nos documents "Spécifications Techniques des Besoins", nous
avons pu mettre en évidence des exigences de qualité. Celles ci sont les
suivantes :
\begin{itemize}
	\item Le rendu du livrable
	\item Les opérations cryptographiques de SmartCard
	\item Les opérations cryptographiques de FaceCrypt
	\item Les accès à la carte à puce
\end{itemize}

	Cette partie du plan de qualité explique comment chaque exigence est
controlée et est appréciée. Nous mettrons en place une échelle de mesure afin
de quantifier la qualité de chacune de ces exigences.

\newpage
\subsection{Le rendu du livrable}
	Celui ci consiste en une application facile d'utilisation répondant
aux exigences des clients.
\subsubsection{Mise en place de l'exigence}
	Pour mesurer la qualité du livrable, nous établissons une échelle
par rapport aux exigences des clients et si ceux ci sont remplis par notre
application.
\subsubsection{Vérification du fonctionnement}
	Les critères de vérifications sont :
\paragraph{OK : } Notre livrable final devra remplir toutes les exigences tant
fonctionnelles qu'opérationnelles.
\paragraph{Partiel : } Notre livrable final ne remplira pas toutes les
exigences mais implémentera toutes les exigences indispensables.
\paragraph{NOT OK : } Notre livrable n'implémentera pas une ou plusieurs
exigences indispensables.

\subsection{Les opérations cryptographiques de SmartCard}
	La carte à puce \emph{SmartCard} est une des composantes de notre
projet. Celle ci va permettre de stocker le biclef (clef publique, clef
privée) de l'utilisateur, ainsi qu'effectuer des opérations de cryptographie.
Dans notre projet, nous utiliserons la carte uniquement pour déchiffrer,
signer et générer des données aléatoires.
\subsubsection{Mise en place de l'exigence}
	Pour éprouver la qualité des opérations cryptographiques de la carte
à puce (SmartCard), nous avons effectué une série de tests qui consistent à
déchiffrer, signer et produire des données arbitraires. Cette qualité sera
évalué en fonction du temps nécessaire pour accomplir l'opération.

\subsubsection{Vérification du fonctionnement}
	Les critères de vérifications sont :
\paragraph{OK : } Le déchiffrement et la signature devront se faire en moins
de 150 ms et la production de données arbitraires en moins de 50 ms.
\paragraph{Partiel : } Toutes ces opérations se feront entre le seuil minimal
et la seconde.
\paragraph{NOT OK : } Toutes ces opérations se feront au dessus de la secondes.


\subsection{Les opérations cryptographiques de FaceCrypt}
	Notre application java "FaceCrypt" est une autre composante de notre
projet. Elle est utilisée pour effectuer des opérations de chiffrement, de
déchiffrement et de vérification.
\subsubsection{Mise en place de l'exigence}
	Pour éprouver la qualité des opérations cryptographiques de notre
application java (FaceCrypt), nous avons effectué une série de tests qui
consistent à chiffrer, déchiffrer et vérifier des données arbitraires. Pour
ces tests, nous prendrons un message défini, de X octets. Cette qualité sera
évalué en fonction du temps nécessaire pour accomplir l'opération.

\subsubsection{Vérification du fonctionnement}
	Les critères de vérifications sont :
\paragraph{OK : } Le chiffrement se fera en moins de X ms, le déchiffrement
en moins de X ms et la vérification en moins de X ms.
\paragraph{Partiel : } Toutes ces opérations devront se faire entre le seuil
minimal et la seconde.
\paragraph{NOT OK : } Toutes ces opérations se feront au dessus de la
secondes.



\subsection{Les accès à la carte à puce}
	Lors de notre projet, un grand nombre de requêtes vont transiter
entre les entités SoftCard et SmartCard. Il est donc nécessaire que ces
accès soient rapides.
\subsubsection{Mise en place de l'exigence}
	Pour mesurer la qualité de cette exigence, nous avons effectué une
série de tests. Ces derniers consistent à effectuer des demandes à la carte
(établissement d'un tunnel sécurisé, nombre aléatoire). Une fois encore,
notre critère de qualité sera le temps requis pour l'opération.
\subsubsection{Vérification du fonctionnement}
	Les critères de vérifications sont :
\paragraph{OK : } L'établissement du tunnel se fera en moins de 106ms et le
nombre aléatoire demandé transitant par le tunnel sécurisé sera reçu en
moins de 160ms.
\paragraph{Partiel : } L'établissement du tunnel devra se faire entre le
seuil optimal et 110 ms et le nombre aléatoire demandé transitant par le
tunnel sécurisé sera reçu entre le seuil optimal et 180 ms.
\paragraph{NOT OK : } L'établissement du tunnel se fera en plus de 110 ms
et le nombre aléatoire demandé transitant par le tunnel sécurisé sera reçu
en plus de 180ms.

\end{document}
