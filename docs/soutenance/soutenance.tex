\documentclass{beamer}
\usepackage[utf8]{inputenc}
\usepackage[francais]{babel} 
\usepackage{graphicx} 
\usepackage{textcomp}
\usepackage{latexsym}
\usepackage{amssymb}
\usepackage{enumerate}
\usepackage{listings}
\usepackage{tikz}


\usetheme{Warsaw}

%\setbeamercolor{normal text}{fg=white,bg=black!90}
%\setbeamercolor{structure}{fg=white}
%
%\setbeamercolor{alerted text}{fg=red!85!black}
%
%\setbeamercolor{item projected}{use=item,fg=black,bg=item.fg!35}
%
%\setbeamercolor*{palette primary}{use=structure,fg=structure.fg}
%\setbeamercolor*{palette secondary}{use=structure,fg=structure.fg!95!black}
%\setbeamercolor*{palette tertiary}{use=structure,fg=structure.fg!90!black}
%\setbeamercolor*{palette quaternary}{use=structure,fg=structure.fg!95!black,bg=black!80}
%
%\setbeamercolor*{framesubtitle}{fg=white}
%
%\setbeamercolor*{block title}{parent=structure,bg=black!60}
%\setbeamercolor*{block body}{fg=black,bg=black!10}
%\setbeamercolor*{block title alerted}{parent=alerted text,bg=black!15}
%\setbeamercolor*{block title example}{parent=example text,bg=black!15}


\title{Smart Social Network - Projet de Master 2 SSI}

\author{
    Zakaria \textsc{Addi}
    Baptiste \textsc{Dolbeau}\\
    Yicheng \textsc{Gao}
    Florian \textsc{Guilbert}\\
    Giovanni \textsc{Huet}
    Emmanuel \textsc{Mocquet}\\
    Maxence \textsc{Péchoux}
    Romain \textsc{Pignard}
}
\institute{Université de Rouen}

\AtBeginSection[]
{
    \begin{frame}
        \tableofcontents[currentsection]
    \end{frame}
}

\begin{document}

\begin{frame}
    \titlepage 
\end{frame}

\begin{frame}
    \frametitle{Plan}
    \tableofcontents[hideallsubsections]
\end{frame}

\section{Introduction}

\subsection{Présentation}
\begin{frame}
    \frametitle{Contexte}
    \begin{block}{SmartCard}
        titi
    \end{block}
    \begin{block}{Secure Social Network}
        toto
    \end{block}
\end{frame}

\subsection{Gestion de projet}
\begin{frame}
    \frametitle{Gestion de projet}
    \begin{block}{ }
    \end{block}
\end{frame}

\section{Carte à puce}

\subsection{Introduction}
\begin{frame}
    \frametitle{Introduction}
    \begin{block}{}
    \end{block}
\end{frame}

\subsection{Java Card}
\begin{frame}
    \frametitle{Présentation}
    \begin{block}{Rappel sur la carte à puce}
        \begin{itemize}
            \item Dispose d'un processeur pour du traitement d'informations.
            \item Permet de stocker des données cachées.
            \item Assure l'authentification de l'utilisateur.
        \end{itemize}
    \end{block}
    \begin{block}{Qu'est-ce que "Java Card" ?}
        \begin{itemize}
            \item Désigne la technologie permettant de développer
                des applets \og sécurisées \fg{} sur carte à puce.

            \item Mais c'est aussi une carte à puce : 
                \begin{itemize}
                    \item programmable 
                    \item multi-applications
                    \item interopérable
                \end{itemize}
        \end{itemize}
    \end{block}
\end{frame}

\begin{frame}
    \frametitle{Fonctionnement}
    \begin{block}{Les APDU}
        \begin{itemize}
            \item Application Protocol Data Unit.
            \item Unité de communication entre le lecteur et la carte.
        \end{itemize}
    \end{block}
    \begin{table}
        \begin{tabular}{|c|c|c|c|c|c|c|}
            \hline
            CLA & INS & P1 & P2 & Lc & Données & Le\\
            \hline
        \end{tabular}
        \newline
        \begin{tabular}{|c|c|}
            \hline
            Données & Status\\
            \hline
        \end{tabular}
        \caption{Structures d'une commande et d'une réponse}
    \end{table}
    \begin{block}{Exemple}
        \begin{itemize}
            \item Commande : 0xB0 0x00 0x00 0x00 0x01 0x05
            \item Réponse : 0x02 0xf2 0x23 0x42 0xcf 0x90 0x00
        \end{itemize}
    \end{block}
\end{frame}

\begin{frame}
    \frametitle{Abstraction}
    \begin{block}{L'API Java Card}
        \begin{itemize}
            \item Permet de s'abstraire de l'assembleur $\rightarrow$ Java
            \item Fournit un certain nombres d'objets : PIN, clefs RSA...
        \end{itemize}
    \end{block}
    \begin{block}{Exemple}
        \begin{itemize}
            \item Todo
                %Todo 
        \end{itemize}
    \end{block}
\end{frame}

\begin{frame}
    \frametitle{Principales contraintes}
    \begin{block}{Les limitations de l'API Java Card}
        \begin{itemize}
            \item types : boolean, byte, short, tableaux associés
            \item pas de \og garbage collector \fg{}
        \end{itemize}
    \end{block}
\end{frame}

\subsection{Les applications développées}
\begin{frame}
    \frametitle{Les applications développées}
    \includegraphics[width=10cm]{graphe_dep}
    \begin{block}{}
    \end{block}
\end{frame}

\subsection{L'aspect sécurité}
\begin{frame}
    \frametitle{L'aspect sécurité}
    \includegraphics[width=9cm]{stack}
    % \begin{block}{}
    % \end{block}
\end{frame}

\subsection{Démonstration}
\begin{frame}
    \frametitle{Démonstration}
    \begin{block}{}
    \end{block}
\end{frame}

\frametitle{L'interface entre SSN et la carte}
\begin{frame}
    \frametitle{L'interface entre SSN et la carte}
    \begin{block}{Actuellement}
        \begin{itemize}
            \item Applications de chiffrement, déchiffrement, signature, stockage...
            \item Client testant ces applications.
        \end{itemize}
    \end{block}

    Mais par rapport à Facebook ?
\end{frame}

\begin{frame}
    \frametitle{L'interface entre SSN et la carte}
    \begin{block}{Un serveur vis-à-vis de SSN}
        \begin{itemize}
            \item Une application (SoftCardServer) se met en
                attente de connexions.
            \item Pour chaque requête reçue, une action est transmise
                à une seconde application : SoftCard.
            \item SoftCardServer renvoie le résultat de SoftCard au client.
        \end{itemize}
    \end{block}
    \begin{block}{Un client vis-à-vis de la carte}
        \begin{itemize}
            \item Une unique instance se connecte au lecteur puis à la carte.
            \item Différentes méthodes permettent de déchiffrer, signer...
            \item Pour certaines, sensibles, la carte devra être déverrouillée.
        \end{itemize}
    \end{block}
\end{frame}

\section{Une protection vis-à-vis de Facebook}

\subsection{Les besoins et exigences}
\begin{frame}
    \frametitle{Les besoins et exigences}
    \begin{block}{}

        Protection des données utilisateur vis-à-vis de tiers

        Authentification forte par carte à puce 
    \end{block}
\end{frame}

\subsection{Présentation des composants}
\begin{frame}
    \frametitle{Présentation des composants}
    %\begin{block}{ }
    \includegraphics[width=12cm]{schema_zako}
    % \end{block}
\end{frame}


\subsection{Présentation des composants}
\begin{frame}
    \frametitle{Base de données}
    \begin{block}{Moteur SQLite }
        Base de données locale

        Accessible depuis Java et l'extension 
    \end{block}
    \begin{block}{Stockage des liens d'amitié dans la base}
        Listes d'amis


        Clés publiques


    \end{block}
\end{frame}


\subsection{Facecrypt}
\begin{frame}
    \frametitle{La communication}
    \includegraphics[width=11cm]{schema_dolby}
\end{frame}

\begin{frame}
    \frametitle{Composition}
    \begin{block}{Six classes java}
        \begin{itemize}
            \item ASymCypher
            \item SymCypher
            \item ServerSSL
            \item Client
            \item Dataprocess
            \item CacheManager
        \end{itemize}
    \end{block}
\end{frame}

\begin{frame}
    \frametitle{Exemple de cycle}

    \begin{itemize}
        \item Received from Facecrypt : \{"action":"getID"\}
        \item Sent to Softcard : 47
        \item Received from Softcard : $\underbrace{666f6f2e6261722e33333434393133}_{\textrm{login}} 20 \underbrace{726f6f74726f6f74}_{\textrm{password}}$
        \item Sent to Facecrypt : \{"action":"getID" ,"login":"foo.bar.3344913","firstConnection":false, "pass":"rootroot"\}
    \end{itemize}

    % \end{block}
\end{frame}

\section{Démonstration}
\begin{frame}
    \frametitle{Démonstration}
    schéma
\end{frame}

\section{Conclusion}
\begin{frame}
    \frametitle{Conclusion}
    \begin{block}{Difficultés rencontrées}
        \begin{itemize}
            \item SmartCard : 
                \begin{itemize}
                    \item taille des données; % limitées dans les APDUs (256)
                    \item communications sécurisées entre la carte à puce et l'application
                        cliente;
                    \item installations des lecteurs;
                    \item stockage "caché"; % vérifier que c'est vraiment caché
                    \item algorithmes implantés sur la carte;
                \end{itemize}
            \item Secure Social Network : 
                \begin{itemize}
                    \item manipulation de la page Facebook;
                    \item communications sécurisées entre SSNExt et FaceCrypt;
                    \item fonctionnement d'une extension.
                \end{itemize}
        \end{itemize}
    \end{block}
\end{frame}

\begin{frame}
    \frametitle{Conclusion}
    \begin{block}{Améliorations possibles}
        \begin{itemize}
            \item SmartCard : 
                \begin{itemize}
                    \item IHM pour entrer le code PIN; % => démon
                    \item Gestion de l'arrachage de la carte; % lors d'une opération, 
                        % la carte plante le cliente, débranchage, rebranchage avant une opération est géré.
                    \item communications sécurisées entre la carte à puce et l'application
                        cliente;
                    \item One Time Password;
                    \item prendre en compte les attaques (canaux cachés);
                    \item algorithmes implantés sur la carte;
                \end{itemize}
            \item Secure Social Network : 
                \begin{itemize}
                    \item finalisation pour mise en production;
                    \item étudier le tatouage d'images.
                \end{itemize}
        \end{itemize}
    \end{block}
\end{frame}

\begin{frame}
    \frametitle{Conclusion}
    \begin{block}{Les apports}
        \begin{itemize}
            \item SmartCard : 
                \begin{itemize}
                    \item manipulation d'une carte à puce;
                \end{itemize}
            \item Secure Social Network : 
                \begin{itemize}
                    \item gestion d'une communication sécurisées entre plusieurs
                        composants;
                \end{itemize}
            \item utilisation concrête de la cryptographie.
        \end{itemize}
    \end{block}
\end{frame}

\end{document}
