\documentclass{beamer}
\usepackage[utf8]{inputenc}
\usepackage[francais]{babel} 
\usepackage{graphicx} 
\usepackage{textcomp}
\usepackage{latexsym}
\usepackage{amssymb}
\usepackage{enumerate}
\usepackage{listings}
\usepackage{tikz}


\usetheme{Warsaw}

%\setbeamercolor{normal text}{fg=white,bg=black!90}
%\setbeamercolor{structure}{fg=white}
%
%\setbeamercolor{alerted text}{fg=red!85!black}
%
%\setbeamercolor{item projected}{use=item,fg=black,bg=item.fg!35}
%
%\setbeamercolor*{palette primary}{use=structure,fg=structure.fg}
%\setbeamercolor*{palette secondary}{use=structure,fg=structure.fg!95!black}
%\setbeamercolor*{palette tertiary}{use=structure,fg=structure.fg!90!black}
%\setbeamercolor*{palette quaternary}{use=structure,fg=structure.fg!95!black,bg=black!80}
%
%\setbeamercolor*{framesubtitle}{fg=white}
%
%\setbeamercolor*{block title}{parent=structure,bg=black!60}
%\setbeamercolor*{block body}{fg=black,bg=black!10}
%\setbeamercolor*{block title alerted}{parent=alerted text,bg=black!15}
%\setbeamercolor*{block title example}{parent=example text,bg=black!15}


\title{Smart Social Network - Projet de Master 2 SSI}

\author{
    Zakaria \textsc{Addi}
    Baptiste \textsc{Dolbeau}\\
    Yicheng \textsc{Gao}
    Florian \textsc{Guilbert}\\
    Giovanni \textsc{Huet}
    Emmanuel \textsc{Mocquet}\\
    Maxence \textsc{Péchoux}
    Romain \textsc{Pignard}
}
\institute{Université de Rouen}

\AtBeginSection[]
{
    \begin{frame}
        \tableofcontents[currentsection]
    \end{frame}
}

\begin{document}

\begin{frame}
\titlepage 
\end{frame}

\begin{frame}
\frametitle{Plan}
\tableofcontents[hideallsubsections]
\end{frame}

\section{Introduction}

\subsection{Présentation}
\begin{frame}
    \frametitle{Présentation}
    \begin{block}{ }
    \end{block}
\end{frame}

\subsection{Gestion de projet}
\begin{frame}
    \frametitle{Gestion de projet}
    \begin{block}{ }
    \end{block}
\end{frame}

\section{Carte à puce}

\subsection{Introduction}
\begin{frame}
    \frametitle{Introduction}
    \begin{block}{}
    \end{block}
\end{frame}

\subsection{Java Card}
\begin{frame}
    \begin{block}{Rappel sur la carte à puce}
        stockage et traitement d'infos
        de base assure authentification
    \end{block}
\end{frame}

\begin{frame}
    \begin{block}{Qu'est-ce que Java Card ?}
        \begin{itemize}
            \item Désigne la technologie permettant de développer
                des applets \og sécurisées \fg{} sur carte à puce 

            \item Côté carte : 
                \begin{itemize}
                    \item carte à puce programmable 
                    \item carte à puce multi-applications
                \end{itemize}
        \end{itemize}
    \end{block}
\end{frame}

\begin{frame}
    \frametitle{Fonctionnement}
    \begin{block}{APDU}
    \end{block}
    \begin{block}{L'API Java Card }
    \end{block}
    \begin{block}{Identifications des applets}
    \end{block}
\end{frame}

\begin{frame}
    \begin{block}{Principales limitations}
        types primitifs : boolean, byte, short
        pas de "garbage collector"

    \end{block}
\end{frame}

\subsection{Les applications développées}
\begin{frame}
    \frametitle{Les applications développées}
    \includegraphics[width=10cm]{graphe_dep}
    \begin{block}{}
    \end{block}
\end{frame}

\subsection{L'aspect sécurité}
\begin{frame}
    \frametitle{L'aspect sécurité}
    \includegraphics[width=9cm]{stack}
    % \begin{block}{}
    % \end{block}
\end{frame}

\subsection{Démonstration}
\begin{frame}
    \frametitle{Démonstration}
    \begin{block}{}
    \end{block}
\end{frame}

\subsection{L'interface avec SSN : SoftCard}
\begin{frame}
    \frametitle{L'interface avec SSN : SoftCard}
    \begin{block}{Actuellement}
        \begin{itemize}
            \item Applications de chiffrement, signature, stockage...
            \item Client testant ces applications
        \end{itemize}
    \end{block}

    Mais par rapport à Facebook ?
\end{frame}

\begin{frame}
    \frametitle{L'interface avec SSN : SoftCard}
    \begin{block}{Un serveur}
        \begin{itemize}
            \item 
        \end{itemize}
    \end{block}
\end{frame}

\section{Une protection vis-à-vis de Facebook}

\subsection{Les besoins et exigences}
\begin{frame}
    \frametitle{Les besoins et exigences}
    \begin{block}{}

Protection des données utilisateur vis-à-vis de tiers

Authentification forte par carte à puce 
    \end{block}
\end{frame}

\subsection{Présentation des composants}
\begin{frame}
    \frametitle{Présentation des composants}
    %\begin{block}{ }
 \includegraphics[width=12cm]{schema_zako}
   % \end{block}
\end{frame}


\subsection{Présentation des composants}
\begin{frame}
    \frametitle{Base de données}
    \begin{block}{Moteur SQLite }
 Base de données locale

Accessible depuis Java et l'extension 
    \end{block}
  \begin{block}{Stockage des liens d'amitié dans la base}
Listes d'amis


Clés publiques


    \end{block}
\end{frame}


\subsection{Facecrypt}
\begin{frame}
    \frametitle{La communication}
    \includegraphics[width=11cm]{schema_dolby}
\end{frame}

\begin{frame}
    \frametitle{Composition}
	\begin{block}{Six classes java}
	\begin{itemize}
	\item ASymCypher
	\item SymCypher
	\item ServerSSL
	\item Client
	\item Dataprocess
	\item CacheManager
	\end{itemize}
	\end{block}
\end{frame}

\begin{frame}
    \frametitle{Exemple de cycle}

	\begin{itemize}
	\item Received from Facecrypt : \{"action":"getID"\}
	\item Sent to Softcard : 47
	\item Received from Softcard : $\underbrace{666f6f2e6261722e33333434393133}_{\textrm{login}} 20 \underbrace{726f6f74726f6f74}_{\textrm{password}}$
	\item Sent to Facecrypt : \{"action":"getID" ,"login":"foo.bar.3344913","firstConnection":false, "pass":"rootroot"\}
\end{itemize}
\end{frame}


\end{document}
