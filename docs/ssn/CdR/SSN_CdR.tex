
%!TEX encoding = UTF-8 Unicode

\documentclass[a4paper,11pt,french]{article}

%Import des packages utilisés pour le document
\usepackage[french]{babel}
\usepackage{chngpage}
\usepackage[colorlinks=true,linkcolor=black,urlcolor=blue]{hyperref}
\usepackage{graphicx, amssymb, color, listings}
\usepackage{fontspec,xltxtra,xunicode,color}
\usepackage{tabularx, longtable}
\usepackage[table]{xcolor}
\usepackage{fancyhdr}
\usepackage{tikz}
\usetikzlibrary{shapes}
\usepackage{lastpage}

\definecolor{gris}{rgb}{0.95, 0.95, 0.95}

%Redéfinition des marges
\addtolength{\hoffset}{-2cm}
\addtolength{\textwidth}{4cm}
\addtolength{\topmargin}{-2cm}
\addtolength{\textheight}{1cm}
\addtolength{\headsep}{0.8cm}
\addtolength{\footskip}{1cm}


%Import page de garde et structures pour la gestion de projet
\usepackage{res/structures}

%Variables
\def\matiere{Conduite de Projet}
\def\filiere{Master 2 SSI}
\def\projectDesc{Smart Social Network}
\def\projectName{\emph{SSN}~}
\def\completeName{\projectDesc ~- \projectName}
\def\docType{Cahier de recettes}
\def\docDate{\today}
\def\version{1.0}
\def\author{Baptiste \textsc{Dolbeau}}
\def\checked{}
\def\approved{}


% -- Début du document -- %
\begin{document}
%Page de garde
\makeFirstPage
\clearpage

%Tableau de mises à jour
\vspace*{1cm}
\begin{center}
\textbf{\huge{MISES À JOUR}}\\
\vspace*{3cm}
\begin{tabularx}{16cm}{|c|c|X|}
\hline
\bfseries{Version} & \bfseries{Date} & \bfseries{Modifications réalisées}\\
\hline
0.1 & 24/02/2013 & Création\\
\hline
1.0 & 28/02/2013 & Finalisation\\
\hline
\end{tabularx}
\end{center}

%La table des matières
\clearpage
\tableofcontents
\clearpage

\section{Introduction}

Ce document est un support pour la validation du logiciel lors de la recette
auprès du client. Il est consacré à la définition des moyens et des
procédures utilisés pour assurer la recette du produit développé. La recette
est un procédé permettant d’assurer la conformité du logiciel à la
spécification déjà définie. Nous allons recenser dans ce document les
objectifs de tests de validation et les moyens nécessaires pour les atteindre
en précisant :
\begin{itemize}
	\item Les pré-conditions à satisfaire;
	\item Les moyens matériels requis (plate-forme de tests);
	\item La logique de leur déroulement (étapes successives).
\end{itemize}
Notre logiciel peut être divisé en une liste de constituants qui seront
testés à tour de rôle. L'ensemble des opérations devra être transparent
vis à vis de l'utilisateur. Les différents cas d’utilisation prélevés
de la spécification technique de besoin sont les suivants (par ordre de
priorité) :

\subsection*{Chiffrement et déchiffrement d'un statut}
Notre application doit pouvoir chiffrer et déchiffrer le statut d'un
utilisateur du réseau social "Facebook". S'il choisit de chiffrer, il
doit spécifier les utilisateurs qui seront autorisés à le déchiffrer.

\subsection*{Chiffrement et déchiffrement d'un commentaire}
Notre application doit pouvoir chiffrer et déchiffrer un commentaire d'un
utilisateur du réseau social "Facebook". S'il choisit de chiffrer, il doit
spécifier les utilisateurs qui seront autorisés à le déchiffrer.

\subsection*{Déploiement du système sur un compte déjà existant}
Notre application finale doit pouvoir être applicable à un compte d'un
utilisateur de "Facebook" déjà existant. Les opérations offertes par notre
système à cet utilisateur ne seront disponibles que pour les futurs messages
qu'il écrira.

\subsection*{Gestion des liens d'amitiés}
Afin d'améliorer l'ergonomie de notre projet, nous proposons de gérer les
liens d'amitiés de l'utilisateur de "Facebook". Il pourra ainsi organiser
ses amis en listes. Ces listes seront utiles lors d'opérations telles que
le chiffrement d'un message ou d'un commentaire.

\subsection*{Chiffrement et déchiffrement d'un document}
Notre application doit pouvoir chiffrer et déchiffrer un document d'un
utilisateur du réseau social "Facebook". S'il choisi de chiffrer, il
doit spécifier les utilisateurs qui seront autorisés à le déchiffrer.

\subsection*{Génération et changement d'un mot de passe}
Notre application doit pouvoir gérer la génération d'un nouveau mot de passe
ainsi que son enregistrement.

\newpage

\section{Documents applicables et de références}
\begin{itemize}
	\item SSN\_STB : Le document renfermant les spécifications techniques de Besoin;
	\item SSN\_DAL : Le document contenant l'architecture logicielle;
	\item Les comptes rendu de réunion du projet;
	\item Le sujet du projet : "proxy-encryption.pdf".
\end{itemize}

\section{Terminologie et sigles utilisés}
\begin{description}
    \item[CdR :] Cahier de Recettes;
    \item[AdR :] Analyse des Risques;
    \item[DAL :] Document d'Architecture Logicielle;
    \item[STB :] Spécification Technique de Besoin;
    \item[SC :] \emph{SmartCard}, relatif au sous-projet sur les cartes à puce;
    \item[SSN :] \emph{Secure Social Network}, relatif au sous-projet sur les
	réseaux sociaux;
    \item[PIN :] \emph{Personal Identification Number} - Code servant à
	authentifier l'utilisateur;
\end{description}

\section{Environnement de tests}
L'ensemble des tests se sont effectués sur des machines ayant les
caractéristiques suivantes :
\begin{itemize} 
	\item Système d'exploitation : Ubuntu 12.04;
	\item Processeur : Intel(R) Core(TM)2 Duo CPU E8400  @ 3.00GHz;
	\item Mémoire : 2Go RAM;
	\item Logiciel : Eclipse Platform Version: 3.8.0, Java 1.6 (Client) et
Firefox 18.0.2.
\end{itemize}


Nous utilisons également des cartes Java Card J3A (marque NXP) avec 40K
d'EEPROM et des lecteurs Omnikey 3121. Les cartes sont conformes aux standards
Java Card 2.2.2 et Global Platform 2.1.1.

\section{Responsabilité}
Afin de mener les tests dans les meilleurs conditions, une organisation au sein
du groupe a été mise en place :\\

La conception et la définition des données de tests ont été réalisées par
Baptiste \textsc{Dolbeau}. Après avoir exécuté les différents tests, les
responsables de ce module transmettront aux développeurs un compte rendu
contenant les résultats afin d’améliorer la version actuelle du logiciel
et d'en fournir une nouvelle à évaluer. Chaque version fournie doit être
testée et validée.

\newpage
	
\section{Stratégie de tests}

La démarche utilisée pour effectuer les tests est la suivante :
\begin{itemize}
\item Mettre à la disposition de l’équipe testeur les modules développés.
\item Réalisation des tests à travers une procédure, celle ci comportera
un jeu de tests ainsi que la modalité de leur exécution.
\item Élaboration d'un compte rendu des résultats des tests qui sera
transmis aux développeurs.
\item Correction des anomalies par l'équipe développeur.
\item Des tests secondaires seront effectués pour s'assurer que toutes les
anomalies ont été corrigées.
\end{itemize}

Les tests seront réalisés par ordre de priorité. Les modules ayant une
priorité indispensable seront pris en compte dès que possible. La condition
d'arrêt des tests sera le succès de ces derniers après correction des
anomalies.

\section{Gestion des anomalies}

A chaque modification apportée (correction), nous devrons réaliser un nombre
de tests permettant de détecter les anomalies persistantes. Toute anomalie
détectée sera notée dans un rapport et ce dernier sera envoyé aux
développeurs afin qu'ils apportent les modifications nécessaires.

\section{Procédures de tests}
Pour chaque cas d’utilisation, nous décrivons une procédure de test
détaillée. Chaque procédure dispose d'un jeu de tests basé sur des
données réelles.

\vspace{1cm}
\hspace{-1cm}
\begin{tabular}{|p{1cm}|p{5cm}|p{5cm}|p{2cm}|p{2cm}|}
\hline
\multicolumn{3}{|l|}{Objet testé : Chiffrement et déchiffrement d'un statut facebook.} & \multicolumn{2}{|l|}{Version: 1.0} \\
\hline
\multicolumn{5}{|l|}{\begin{tabular}{l} Objectif de test : Vérifier le comportement d'une demande de chiffrement ou déchiffrement \\ d'un statut facebook.\end{tabular}} \\
\hline
\multicolumn{5}{|l|}{Procédure n°1 : Chiffrer un statut en mode anonyme.} \\
\hline
N° & Actions & Résultats attendus & Exigence & OK/NOK \\
\hline
2 & Aprés entrée du texte, Appuyer sur le bouton "chiffrer". & Une "pop-up" apparait pour la selection des listes et du mode. & F-Gl-10 & OK \\
\hline
3 & Sélectionner une liste d'amis et le mode anonyme. & Les cases sélectionnées sont cochées. & F-Gl-10 & OK \\
\hline
4 & Appuyer sur le bouton "chiffrer". & Obtention d'un chiffré, ne contenant pas les noms des amis de la liste spécifiée, à la place de notre message. & F-Gl-10 & OK \\
\hline
\end{tabular}

\vspace{1cm}
\hspace{-1cm}
\begin{tabular}{|p{1cm}|p{5cm}|p{5cm}|p{2cm}|p{2cm}|}
\hline
\multicolumn{5}{|l|}{Procédure n°2 : Déchiffrer un statut en mode anonyme.} \\
\hline
N° & Actions & Résultats attendus & Exigence & OK/NOK \\
\hline
1 & Sélectionner un message chiffré et appuyer sur le bouton \"Déchiffrer\". & Obtention du message déchiffré si il lui était destiné,ou sinon d'un message "Ce message ne vous concerne pas.", à la place de notre message chiffré. & F-Gl-10 & OK \\
\hline
\end{tabular}

\vspace{1cm}
\hspace{-1cm}
\begin{tabular}{|p{1cm}|p{5cm}|p{5cm}|p{2cm}|p{2cm}|}
\hline
\multicolumn{5}{|l|}{Procédure n°3 : Chiffrer un statut en mode non anonyme.} \\
\hline
N° & Actions & Résultats attendus & Exigence & OK/NOK \\
\hline
1 & Écrire un message dans la zone reservée. & Le bouton "chiffrer" apparait. & F-Gl-10 & OK \\
\hline
2 & Appuyer sur le bouton "chiffrer". & Une "pop-up" apparait pour la selection des listes et du mode. & F-Gl-10 & OK \\
\hline
3 & Sélectionner une liste d'amis et le mode non-anonyme. & Les cases sélectionnées sont cochées. & F-Gl-10 & OK \\
\hline
4 & Appuyer sur le bouton "chiffrer". & Un chiffré, contenant les noms des amis de la liste spécifiée, apparait à la place de notre message. & F-Gl-10 & OK \\
\hline
\end{tabular}

\vspace{1cm}
\hspace{-1cm}
\begin{tabular}{|p{1cm}|p{5cm}|p{5cm}|p{2cm}|p{2cm}|}
\hline
\multicolumn{5}{|l|}{Procédure n°4 : Déchiffrer un statut en mode non anonyme.} \\
\hline
N° & Actions & Résultats attendus & Exigence & OK/NOK \\
\hline
1 & Aucune action nécessaire. & Obtention du message déchiffré si il lui était destiné ou du message \"Ce message ne vous concerne pas\", à la place de notre message chiffré. & F-Gl-10 & OK \\
\hline
\end{tabular}

\vspace{3cm}
\hspace{-1cm}
\begin{tabular}{|p{1cm}|p{5cm}|p{5cm}|p{2cm}|p{2cm}|}
\hline
\multicolumn{3}{|l|}{Objet testé : Chiffrement et déchiffrement d'un document.} & \multicolumn{2}{|l|}{Version: 1.0} \\
\hline
\multicolumn{5}{|l|}{\begin{tabular}{l}Objectif de test : Vérifier le comportement d'une demande de chiffrement ou déchiffrement\\ d'un document sur facebook.\end{tabular}} \\
\hline
\multicolumn{5}{|l|}{Procédure n°5 : Chiffrer un document sur facebook.} \\
\hline
N° & Actions & Résultats attendus & Exigence & OK/NOK \\
\hline
1 & Chiffrer un document telle qu'une image sur facebook & Non implémenté sur ce projet. & F-Gl-20 & OK \\
\hline
\end{tabular}

\vspace{1cm}
\hspace{-1cm}
\begin{tabular}{|p{1cm}|p{5cm}|p{5cm}|p{2cm}|p{2cm}|}
\hline
\multicolumn{5}{|l|}{Procédure n°6 : Déchiffrer un document sur facebook.} \\
\hline
N° & Actions & Résultats attendus & Exigence & OK/NOK \\
\hline
1 & Déchiffrer un document telle qu'une image sur facebook & Non implémenté sur ce projet. & F-Gl-20 & OK \\
\hline
\end{tabular}

\vspace{3cm}
\hspace{-1cm}
\begin{tabular}{|p{1cm}|p{5cm}|p{5cm}|p{2cm}|p{2cm}|}
\hline
\multicolumn{3}{|l|}{Objet testé : Gestion des liens d'amitiés.} & \multicolumn{2}{|l|}{Version: 1.0} \\
\hline
\multicolumn{5}{|l|}{Objectif de test : Vérifier la manipulation des listes d'amis.} \\
\hline
\multicolumn{5}{|l|}{Procédure n°7 : Création d'une liste d'amis.} \\
\hline
N° & Actions & Résultats attendus & Exigence & OK/NOK \\
\hline
1 & Appuyer sur le bouton "Créer" dans la partie gauche de la page. & Une "pop-up" vous demande un nom de liste. & F-Gl-30 & OK \\
\hline
2 & Entrer un nom de liste. & Création d'une liste d'amis vide visible en dessous du bouton "créer" et dans la base de données. & F-Gl-30 & OK \\
\hline
\end{tabular}

\vspace{1cm}
\hspace{-1cm}
\begin{tabular}{|p{1cm}|p{5cm}|p{5cm}|p{2cm}|p{2cm}|}
\hline
\multicolumn{5}{|l|}{Procédure n°8 : Modification d'une liste.} \\
\hline
N° & Actions & Résultats attendus & Exigence & OK/NOK \\
\hline
1 & Appuyer sur le bouton "modifier" à droite d'un nom de liste. & Une "pop-up" s'ouvre et affiche la liste de vos amis. Les amis appartenant à la liste sont déjà cochés & F-Gl-30 & OK \\
\hline
2 & Cocher ou décocher un ami dans la liste. & Cet ami est respectivement ajouté ou supprimé de la liste sélectionnée. Un lien est créé ou supprimé dans la base de données. & F-Gl-30 & OK \\
\hline
\end{tabular}

\vspace{1cm}
\hspace{-1cm}
\begin{tabular}{|p{1cm}|p{5cm}|p{5cm}|p{2cm}|p{2cm}|}
\hline
\multicolumn{5}{|l|}{Procédure n°9 : Suppression d'une liste d'amis.} \\
\hline
N° & Actions & Résultats attendus & Exigence & OK/NOK \\
\hline
1 & Appuyer sur le bouton "supprimer" à droite d'un nom de liste. & La liste est supprimée de la page et de la base de données. & F-Gl-30 & OK \\
\hline
\end{tabular}


\vspace{1cm}
\hspace{-1cm}
\begin{tabular}{|p{1cm}|p{5cm}|p{5cm}|p{2cm}|p{2cm}|}
\hline
\multicolumn{3}{|l|}{\begin{tabular}{l}Objet testé : Chiffrement et déchiffrement d'un commentaire.\end{tabular}} & \multicolumn{2}{|l|}{Version: 1.0} \\
\hline
\multicolumn{5}{|l|}{\begin{tabular}{l}Objectif de test : Vérifier le comportement d'une demande de chiffrement ou déchiffrement\\ d'un commentaire attaché à un statut facebook.\end{tabular}} \\
\hline
\multicolumn{5}{|l|}{Procédure n°10 : Chiffrement d'un commentaire lié à un statut facebook que l'on peut déchiffrer.} \\
\hline
N° & Actions & Résultats attendus & Exigence & OK/NOK \\
\hline
1 & Utiliser la procédure 3 ou 4 pour déchiffrer un statut. & Obtention du message déchiffré à la place de notre message chiffré. & F-Gl-40 & OK \\
\hline
2 & Ajouter un commentaire lié au statut déchiffré & Un bouton "chiffrer" apparait & F-Gl-40 & OK \\
\hline
3 & Appuyer sur le bouton "chiffrer". & Le commentaire chiffré apparait à la place de notre commentaire, prêt à l'envoi. & F-Gl-40 & OK \\
\hline
\end{tabular}

\vspace{1cm}
\hspace{-1cm}
\begin{tabular}{|p{1cm}|p{5cm}|p{5cm}|p{2cm}|p{2cm}|}
\hline
\multicolumn{5}{|l|}{\begin{tabular}{l}Procédure n°11 : Chiffrement d'un commentaire lié à un statut chiffré facebook\\ que l'on ne peut pas déchiffrer.\end{tabular}} \\
\hline
N° & Actions & Résultats attendus & Exigence & OK/NOK \\
\hline
1 & Utiliser la procédure 3 ou 4 pour déchiffrer un statut. & Obtention du message "Ce message ne vous concerne pas", à la place de notre message chiffré. & F-Gl-40 & OK \\
\hline
2 & Écrire un commentaire & Aucun bouton "chiffrer" n'est disponible. L'action n'est donc pas possible. & F-Gl-40 & OK \\
\hline
\end{tabular}

\vspace{1cm}
\hspace{-1cm}
\begin{tabular}{|p{1cm}|p{5cm}|p{5cm}|p{2cm}|p{2cm}|}
\hline
\multicolumn{5}{|l|}{Procédure n°12 : Post d'un commentaire non chiffré lié à un statut chiffré.} \\
\hline
N° & Actions & Résultats attendus & Exigence & OK/NOK \\
\hline
1 & Insertion d'un commentaire relié à un message chiffré sans le dechiffrer. & Le commentaire non chiffré apparait rattaché au statut. & F-Gl-40 & OK \\
\hline
\end{tabular}

\vspace{1cm}
\hspace{-1cm}
\begin{tabular}{|p{1cm}|p{5cm}|p{5cm}|p{2cm}|p{2cm}|}
\hline
\multicolumn{5}{|l|}{Procédure n°13 : Déchiffrement d'un (ou plusieurs) commentaire(s) lié(s) à un statut chiffré.} \\
\hline
N° & Actions & Résultats attendus & Exigence & OK/NOK \\
\hline
1 & Appuyer sur le bouton "déchiffrer" d'un message chiffré contenant des commentaires chiffrés ou non. & Obtention du statut déchiffré et de la liste des commentaires déchiffrés ou laissés tels quels s'ils ne sont pas chiffrés. & F-Gl-40 & OK \\
\hline
\end{tabular}

\vspace{3cm}
\hspace{-1cm}
\begin{tabular}{|p{1cm}|p{5cm}|p{5cm}|p{2cm}|p{2cm}|}
\hline
\multicolumn{3}{|l|}{\begin{tabular}{l}Objet testé : Déploiement du système sur un compte facebook\\ déjà existant.\end{tabular}} & \multicolumn{2}{|l|}{Version: 1.0} \\
\hline
\multicolumn{5}{|l|}{Objectif de test : Vérifier l'intégration du système sur un compte pré-existant.} \\
\hline
\multicolumn{5}{|l|}{Procédure n°14 : Chiffrement possible.} \\
\hline
N° & Actions & Résultats attendus & Exigence & OK/NOK \\
\hline
1 & Création d'un statut facebook. & Le bouton "chiffrer" est disponible. & F-Gl-50 & OK \\
\hline
\end{tabular}

\vspace{1cm}
\hspace{-1cm}
\begin{tabular}{|p{1cm}|p{5cm}|p{5cm}|p{2cm}|p{2cm}|}
\hline
\multicolumn{5}{|l|}{Procédure n°15 : Déchiffrement possible.} \\
\hline
N° & Actions & Résultats attendus & Exigence & OK/NOK \\
\hline
1 & Aucune action nécessaire. & Le bouton "déchiffrer" apparait à suite à un message chiffré. & F-Gl-50 & OK \\
\hline
\end{tabular}

\vspace{1cm}
\hspace{-1cm}
\begin{tabular}{|p{1cm}|p{5cm}|p{5cm}|p{2cm}|p{2cm}|}
\hline
\multicolumn{5}{|l|}{Procédure n°16 : Identification possible via la carte à puce (carte vierge).} \\
\hline
N° & Actions & Résultats attendus & Exigence & OK/NOK \\
\hline
1 & Aller sur la page "facebook.com". & Une demande de code PIN apparait dans le terminal. & F-Gl-50 & OK \\
\hline
2 & Rentrer le code PIN dans le terminal. & Une "pop-up" apparait et nous demande de rentrer les identifiants & F-Gl-50 & OK \\
\hline
3 & Rentrer les identifiants & Connexion au site, la carte contient maintenant les données. & F-Gl-50 & OK \\
\hline
\end{tabular}

\vspace{1cm}
\hspace{-1cm}
\begin{tabular}{|p{1cm}|p{5cm}|p{5cm}|p{2cm}|p{2cm}|}
\hline
\multicolumn{5}{|l|}{Procédure n°17 : Identification possible via la carte à puce (carte pré-remplie).} \\
\hline
N° & Actions & Résultats attendus & Exigence & OK/NOK \\
\hline
1 & Aller sur la page "facebook.com". & Une demande de code PIN apparait dans le terminal. & F-Gl-50 & OK \\
\hline
2 & Rentrer le code PIN dans le terminal. & Connexion au site, la carte contient maintenant les données. & F-Gl-50 & OK \\
\hline
\end{tabular}

\vspace{1cm}
\hspace{-1cm}
\begin{tabular}{|p{1cm}|p{5cm}|p{5cm}|p{2cm}|p{2cm}|}
\hline
\multicolumn{5}{|l|}{Procédure n°18 : Possibilité de changer de mot de passe.} \\
\hline
N° & Actions & Résultats attendus & Exigence & OK/NOK \\
\hline
1 & Aller sur dans l'onglet "compte" de l'icone outils (roue crantée) & Une liste de paramètres modifiables apparait. & F-Gl-50 & OK \\
\hline
2 & Appuyer sur le bouton "modifier" lié au paramètre "mot de passe" & Une liste apparait avec les champs pré-remplis. & F-Gl-50 & OK \\
\hline
3 & Appuyer sur le bouton "enregistrer les modifications". & Le nouveau mot de passe est pris en compte par la carte et par facebook. & F-Gl-50 & OK \\
\hline
\end{tabular}

\vspace{1cm}
\hspace{-1cm}
\begin{tabular}{|p{1cm}|p{5cm}|p{5cm}|p{2cm}|p{2cm}|}
\hline
\multicolumn{5}{|l|}{Procédure n°19 : Insertion de la clef publique.} \\
\hline
N° & Actions & Résultats attendus & Exigence & OK/NOK \\
\hline
1 & Aller sur la page principale. & Un bouton "publier ma clef" & F-Gl-50 & OK \\
\hline
2 & Appuyer sur le bouton "publier ma clef". & La clef publique de l'utilisateur est affichée dans le champs "à propos de moi" sur notre profil. & F-Gl-50 & OK \\
\hline
\end{tabular}

\vspace{1cm}
\hspace{-1cm}
\begin{tabular}{|p{1cm}|p{5cm}|p{5cm}|p{2cm}|p{2cm}|}
\hline
\multicolumn{5}{|l|}{Procédure n°20 : Récupération des clefs publiques.} \\
\hline
N° & Actions & Résultats attendus & Exigence & OK/NOK \\
\hline
1 & Aller sur la page principale. & Un bouton "synchroniser les clefs" & F-Gl-50 & OK \\
\hline
2 & Appuyer sur le bouton "synchroniser les clef". & La base de donnée est remplie avec le nom des utilisateurs et leur clef publique. & F-Gl-50 & OK \\
\hline
\end{tabular}

\newpage

\section{Couverture de tests}
Ce tableau reprend les exigences de la STB et précise, pour chacune d’entre elles, la méthode
de vérification (démonstration / tests) et une description de celle ci.

\vspace{1cm}
\hspace{-1cm}
\begin{tabular}{|p{1.5cm}|p{2.5cm}|p{2.5cm}|p{9cm}|}
\hline
Exigence & Méthode de vérification & Procédure utilisée & Commentaire \\
\hline
F-Gl-10 & Test & Procédure 1 & Ce test consiste à chiffrer un statut en mode anonyme. \\
\hline
F-Gl-10 & Test & Procédure 2 & Le test consiste à déchiffrer un statut en mode anonyme. \\
\hline
F-Gl-10 & Test & Procédure 3 & Le test consiste à chiffrer un statut en mode non anonyme. \\
\hline
F-Gl-10 & Test & Procédure 4 & Le test consiste à déchiffrer un statut en mode non anonyme. \\
\hline
F-Gl-20 & Test & Procédure 5 & Le test consiste à chiffrer un document sur facebook. \\
\hline
F-Gl-20 & Test & Procédure 6 & Ce test consiste à déchiffrer un document sur facebook. \\
\hline
F-Gl-30 & Test & Procédure 7 & Le test consiste à créer une liste d'amis. \\
\hline
F-Gl-30 & Test & Procédure 8 & Le test consiste à modifier une liste d'amis. \\
\hline
F-Gl-30 & Test & Procédure 9 & Le test consiste à supprimer une liste d'amis. \\
\hline
F-Gl-40 & Test & Procédure 10 & Le test consiste à chiffrer un commentaire lié à un statut que l'on peut déchiffrer. \\
\hline
F-Gl-40 & Test & Procédure 11 & Le test consiste à chiffrer un commentaire lié à un statut que l'on ne peut pas déchiffrer. \\
\hline
F-Gl-40 & Test & Procédure 12 & Le test consiste à poster un commentaire non chiffré lié à un statut chiffré. \\
\hline
F-Gl-40 & Test & Procédure 13 & Le test consiste à déchiffrer un (ou plusieurs) commentaire(s) lié(s) à un statut chiffré. \\
\hline
F-Gl-50 & Démonstration & Procédure 14 & Le test consiste à évaluer le chiffrement. \\
\hline
F-Gl-50 & Démonstration & Procédure 15 & Le test consiste à évaluer déchiffrement. \\
\hline 
F-Gl-50 & Démonstration & Procédure 16 & Le test consiste à évaluer l'identification via la carte à puce (carte vierge). \\
\hline 
F-Gl-50 & Démonstration & Procédure 17 & Le test consiste à évaluer l'identification via la carte à puce (carte pré-remplie). \\
\hline
F-Gl-50 & Démonstration & Procédure 18 & Le test consiste à évaluer le changement de mot de passe. \\
\hline
F-Gl-50 & Démonstration & Procédure 19 & Le test consiste à évaluer l'affichage de la clef publique. \\
\hline
F-Gl-50 & Démonstration & Procédure 20 & Le test consiste à évaluer la récupération des clefs publiques. \\
\hline
\end{tabular}
\end{document}
