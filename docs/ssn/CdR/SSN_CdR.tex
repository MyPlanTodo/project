
%!TEX encoding = UTF-8 Unicode

\documentclass[a4paper,11pt,french]{article}

%Import des packages utilisés pour le document
\usepackage[french]{babel}
\usepackage{chngpage}
\usepackage[colorlinks=true,linkcolor=black,urlcolor=blue]{hyperref}
\usepackage{graphicx, amssymb, color, listings}
\usepackage{fontspec,xltxtra,xunicode,color}
\usepackage{tabularx, longtable}
\usepackage[table]{xcolor}
\usepackage{fancyhdr}
\usepackage{tikz}
\usetikzlibrary{shapes}
\usepackage{lastpage}

\definecolor{gris}{rgb}{0.95, 0.95, 0.95}

%Redéfinition des marges
\addtolength{\hoffset}{-2cm}
\addtolength{\textwidth}{4cm}
\addtolength{\topmargin}{-2cm}
\addtolength{\textheight}{1cm}
\addtolength{\headsep}{0.8cm}
\addtolength{\footskip}{1cm}


%Import page de garde et structures pour la gestion de projet
\usepackage{res/structures}

%Variables
\def\matiere{Conduite de Projet}
\def\filiere{Master 2 SSI}
\def\projectDesc{Smart Social Network}
\def\projectName{\emph{SSN}~}
\def\completeName{\projectDesc ~- \projectName}
\def\docType{Cahier de recettes}
\def\docDate{\today}
\def\version{0.1}
\def\author{Giovanni \textsc{Huet}, Romain \textsc{Pignard}}
\def\checked{Florian \textsc{Guilbert}, Emmanuel \textsc{Mocquet}}
\def\approved{}


% -- Début du document -- %
\begin{document}
%Page de garde
\makeFirstPage
\clearpage

%Tableau de mises à jour
\vspace*{1cm}
\begin{center}
\textbf{\huge{MISES À JOUR}}\\
\vspace*{3cm}
\begin{tabularx}{16cm}{|c|c|X|}
\hline
\bfseries{Version} & \bfseries{Date} & \bfseries{Modifications réalisées}\\
\hline
0.1 & 05/02/2013 & Création\\
\hline
&&\\
\hline
\end{tabularx}
\end{center}

%La table des matières
\clearpage
\tableofcontents
\clearpage

\section{Introduction}

Ce document est un support pour la validation du logiciel lors de la recette auprès du client.
Il est consacré à la définition des moyens et des procédures utilisés pour assurer la recette du
produit développée. La recette est un procédé permettant d’assurer que le logiciel est conforme à la
spécification déjà définie.
Nous allons recenser dans ce cahier de recette les objectifs de tests de validation et les moyens
nécessaires pour les atteindre en précisant :
\begin{itemize}
	\item les conditions à satisfaire préalablement à l’exécution des tests ;
	\item les moyens matériels requis (plate-forme de tests) ;
	\item la logique de leur déroulement (étapes successives).
\end{itemize}
Les fonctionnalités de notre logiciel peuvent être divisées en une liste de constituants qui seront
testés à tour de rôle. L'ensemble des opérations devront être transparentes vis à vis de l'utilisateur. 
Nous donnerons par la suite les différents cas d’utilisation prélevé de la spécification technique de besoin :

\subsection*{Génération de nombres aléatoires}
La carte à puce doit pouvoir générer des nombres aléatoires de manière sécurisée, c'est à dire non prévisible.

\subsection*{Déblocage de la carte (via authentification par code PIN ou via PUK)}
Afin d'utiliser la carte, il nous faut pour cela nous authentifier à l'aide d'un code PIN. Cependant si l'utilisateur échoue à l'authentification par code PIN suite à un certain nombre de tentatives alors la carte sera verrouillée. Pour la déverrouiller on utilisera un code PUK.

\subsection*{Transimission de données}
La carte doit pouvoir transmettre des données stockées à SoftCard (login, mot de passe, clef publique,...).

\subsection*{Chiffrement/Déchiffrement de données}
Sur la carte est stockée la clef public et la clef privée préalablement générées (Crypto système assymétrique de type RSA). Ces clefs nous permettront de chiffrer et de déchiffrer des données reçu ou à envoyer.

\subsection*{Signature/Vérification de données}
Par le biais de la carte, nous serons en mesure de signer des données avec notre clef privée afin d'authentifier des données afin d'appliquer la non répudiation. A contrario, nous devons également pouvoir vérifier l'auteur des données. Pour cela une vérification des données devra être possible via la clef publique. 

\subsection*{Administration des cartes}
On devra pouvoir également administrer les cartes, telle que la réinitialisation du code PIN, attribution du code PIN,...


\section{Documents applicables et de référence}
\begin{itemize}
	\item SC\_STB : le document renfermant les spécifications techniques de Besoin;
	\item SC\_DaL : le document contenant l'architecture du logiciel;
	\item Les comptes rendu de réunion du projet;
	\item cartes-a-puce.pdf, le sujet du projet.
\end{itemize}

\section{Terminologie et sigles utilisés}
\begin{description}
    \item[CdR :] Cahier de Recettes;
    \item[AdR :] Analyse des Risques;
    \item[DAL :] Document d'Architecture Logicielle;
    \item[PdD :] Plan de développement;
    \item[STB :] Spécification Technique de Besoins;
    \item[SC :] SmartCard, relatif au sous-projet sur les cartes à puce;
    \item[SSN :] \emph{Secure Social Network};
    \item[FaceCrypt :] Application Java gérant les traitements lourds
    (chiffrement/déchiffrement) de l'extension et étant en relation avec
        la carte à puce;
    \item[IHM :] Interface Homme-Machine, (interface graphique);
    \item[Utilisateur :] entité (humain ou programme) intéragissant avec ce
        sous-projet;
    \item[Système :] ce sous-projet;
    \item[SoftCard :] Application effectuant le relais entre la carte
        et FaceCrypt;
    \item[Extension :] programme incorporé dans le navigateur;
    \item[Aléatoire :] indistingable en temps polynomial, distribution
        de probabilité uniforme;
    \item[PRNG :] (Pseudo Random Number Generator) générateur de nombres
        pseudo-aléatoires;
    \item[PIN :] (Personal Identification Number) code servant à authentifier
        l'utilisateur;
    \item[PUK :] (Personal Unlock Key) code servant à débloquer la carte quand
        trop de codes PIN erronés ont été entrés.
\end{description}

\section{Environnemment de tests}
L'ensemble des tests ce sont effectués sur des machines ayant ces caractéristiques:
\begin{itemize} 
	\item Système d'exploitation: Ubuntu 12.04
	\item Processeur: Intel(R) Core(TM)2 Duo CPU E8400  @ 3.00GHz
	\item Mémoire: 2Go RAM
	\item Logiciel: Eclipse Platform Version: 3.8.0, Java 1.6 (Client) et Java 1.5 (Java card).
\end{itemize}



Nous utilisons également des cartes Java Card J3A (marque NXP) avec 40K d'EEPROM et des lecteurs Omnikey 3121.
Les cartes sont conformes aux standards Java Card 2.2.2 et Global Platform 2.1.1.

\section{Responsabilité}
Afin de mener les tests dans les meilleurs conditions, une organisation au sein du groupe a été mise en place: \\

La conception et la définition des données de tests a été réalisée par Giovanni \textsc{Huet} et Romain \textsc{Pignard}. 
Après avoir exécuté les différents tests, les responsables de ce module transmettront aux développeurs un compte rendu contenant
les résultats de ces tests afin d’améliorer la version courante du logiciel et de fournir une nouvelle
version à tester. Chaque version fournie doit être testée et validée.

\section{Stratégie de tests}

La démarche utilisée pour effectuer les tests est la suivante :
\begin{itemize}
\item Mettre à la disposition de l’équipe testeur les modules développés.
\item Réalisation des tests à travers une procédure de tests, cette procédure comportera un jeu
de tests et les modalités d’exécution des tests procédure de test.
\item Élaboration d'un compte rendu des résultats des tests qui sera transmis aux développeurs.
\item Correction des anomalies par l'équipe développeur.
\item Des tests secondaires seront effectués pour s'assurer que toutes les anomalies ont été corrigées.
\end{itemize}

Les tests seront réalisés par ordre de priorité, les modules ayant une priorité indispensable seront
pris en compte dès que possible. La condition d'arrêt des tests sera le succès de ces derniers après correction 
des anomalies.

\section{Gestion des anomalies}

A chaque modification apportée (corrigé), nous devons réaliser un nombre de tests permettant de détecter les
anomalies persistantes. Toute anomalie détectée sera noté en détails dans un rapport et ce dernier sera envoyé
 aux développeurs afin qu'ils apportent les modifications nécessaires.

\section{Procédures de tests}
Pour chaque cas d’utilisation, nous décrivons une procédure de test détaillée, chaque
procédure dispose d'un jeu de test basé sur des données réelles.


\begin{figure}[!h]
\begin{tabular}{|p{1cm}|p{5cm}|p{5cm}|p{2cm}|p{2cm}|}
\hline
\multicolumn{3}{|l|}{Objet testé: Génération de nombres aléatoire} & \multicolumn{2}{|l|}{Version: 1.0} \\
\hline
\multicolumn{5}{|l|}{Objectif de test: vérifier le comportement du générateur aléatoire de la carte à puce.} \\
\hline
\multicolumn{5}{|l|}{Procédure n°1: générer un nombre aléatoire} \\
\hline
N° & Actions & Résultats attendus & Exigence & OK/NOK \\
\hline
1 & Générer un nombre aléatoire à l'aide des fonctions javacard disponibles. & Obtention d'un nombre aléatoire. & F-Gl-10 & OK \\
\hline
\end{tabular}
\end{figure}


\begin{figure}[!h]
\begin{tabular}{|p{1cm}|p{5cm}|p{5cm}|p{2cm}|p{2cm}|}
\hline
\multicolumn{5}{|l|}{Procédure n°2: évaluer le niveau de l'aléatoire} \\
\hline
N° & Actions & Résultats attendus & Exigence & OK/NOK \\
\hline
1 & Générer plusieurs (millions) nombres aléatoires afin d'établir des stastitiques et vérifier le niveau de l'aléatoire. &
Générateur non prévisible (Probabilité uniforme) & F-Gl-10 & NC \\
\hline
\end{tabular}
\end{figure}



\begin{figure}[!h]
\begin{tabular}{|p{1cm}|p{5cm}|p{5cm}|p{2cm}|p{2cm}|}
\hline
\multicolumn{5}{|l|}{Procédure n°3: évaluer le temps d'éxecution} \\
\hline
N° & Actions & Résultats attendus & Exigence & OK/NOK \\
\hline
1 & Etablir une moyenne pour la génération d'un nombre aléatoire. & Pour être transparent à l'utilisateur, nous souhaitons que le temps de génération soit < 300ms. & F-Gl-10 & OK \\
\hline
\end{tabular}
\end{figure}



\begin{figure}[!h]
\begin{tabular}{|p{1cm}|p{5cm}|p{5cm}|p{2cm}|p{2cm}|}
\hline
\multicolumn{3}{|l|}{\begin{tabular}{l}Objet testé: Déblocage de la carte\\ (via authentification par code PIN et PUK)\end{tabular}} & \multicolumn{2}{|l|}{Version: 1.0} \\
\hline
\multicolumn{5}{|l|}{Objectif de test: vérifier le comportement de la carte lors de plusieurs tentive d'authentification.} \\
\hline
\multicolumn{5}{|l|}{Procédure n°4: verrouillage/deverouillage de la carte.} \\
\hline
N° & Actions & Résultats attendus & Exigence & OK/NOK \\
\hline
1 & Entrer le code PIN valide. & Déblocage de la carte et authentification de l'utilisateur rendant la carte utilisable. & F-Gl-20 & OK \\
\hline
\end{tabular}
\end{figure}


\begin{figure}[!h]
\begin{tabular}{|p{1cm}|p{5cm}|p{5cm}|p{2cm}|p{2cm}|}
\hline
\multicolumn{5}{|l|}{Procédure n°5: verrouillage de la carte} \\
\hline
N° & Actions & Résultats attendus & Exigence & OK/NOK \\
\hline
1 & Effectuer un certain nombre d'authentification erroné. & Verrouillage de la carte au bout d'un nombre prédéfini de tentative. & F-Gl-20 & OK \\
\hline
\end{tabular}
\end{figure}


\begin{figure}[!h]
\begin{tabular}{|p{1cm}|p{5cm}|p{5cm}|p{2cm}|p{2cm}|}
\hline
\multicolumn{5}{|l|}{Procédure n°6: déverrouillage de la carte} \\
\hline
N° & Actions & Résultats attendus & Exigence & OK/NOK \\
\hline
1 & Entrer le code PUK valide. & Deverrouillage de la carte rendant celle ci de nouveau opérationnelle. & F-Gl-20 & OK \\
\hline
\end{tabular}
\end{figure}


\begin{figure}[!h]
\begin{tabular}{|p{1cm}|p{5cm}|p{5cm}|p{2cm}|p{2cm}|}
\hline
\multicolumn{3}{|l|}{Objet testé: transmission de données} & \multicolumn{2}{|l|}{Version: 1.0} \\
\hline
\multicolumn{5}{|l|}{Objectif de test: vérifier si les données contenu dans la carte peuvent être transmise.} \\
\hline
\multicolumn{5}{|l|}{Procédure n°7: transmettre des données.} \\
\hline
N° & Actions & Résultats attendus & Exigence & OK/NOK \\
\hline
1 & Envoyer des données de la carte à SoftCard. & Réception intégrale des données par SoftCard. & F-Gl-30 & OK \\
\hline
\end{tabular}
\end{figure}


\begin{figure}[!h]
\begin{tabular}{|p{1cm}|p{5cm}|p{5cm}|p{2cm}|p{2cm}|}
\hline
\multicolumn{3}{|l|}{Objet testé: communication sécurisée lecteur/carte} & \multicolumn{2}{|l|}{Version: 1.0} \\
\hline
\multicolumn{5}{|l|}{Objectif de test: vérifier les fonctions de chiffrement, d'authentification et d'intégrité.} \\
\hline
\multicolumn{5}{|l|}{Procédure n°8: envoie de données chiffrées} \\
\hline
N° & Actions & Résultats attendus & Exigence & OK/NOK \\
\hline
1 & Envoyer des données chiffrées & Données non compréhensibles sans la clef de déchiffrement. & ? & OK \\
\hline
\end{tabular}
\end{figure}



\begin{figure}[!h]
\begin{tabular}{|p{1cm}|p{5cm}|p{5cm}|p{2cm}|p{2cm}|}
\hline
\multicolumn{5}{|l|}{Procédure n°9: envoie de données authentifiées} \\
\hline
N° & Actions & Résultats attendus & Exigence & OK/NOK \\
\hline
1 & Envoyer des données authentifiées sans altérer le contenu. & Validation de la non modification des données reçues. & ? & OK \\
\hline
\end{tabular}
\end{figure}


\begin{figure}[!h]
\begin{tabular}{|p{1cm}|p{5cm}|p{5cm}|p{2cm}|p{2cm}|}
\hline
\multicolumn{5}{|l|}{Procédure n°10: envoie de données authentifiées avec altération} \\
\hline
N° & Actions & Résultats attendus & Exigence & OK/NOK \\
\hline
1 & Envoyer des données authentifiées en altérerant le contenu. & Détection de la non modification des données reçues. & ? & OK \\
\hline
\end{tabular}
\end{figure}


\begin{figure}[!h]
\begin{tabular}{|p{1cm}|p{5cm}|p{5cm}|p{2cm}|p{2cm}|}
\hline
\multicolumn{5}{|l|}{Procédure n°11: déchiffrement de données reçues} \\
\hline
N° & Actions & Résultats attendus & Exigence & OK/NOK \\
\hline
1 & Déchiffrement des données à la sortie du tunnel. & Récupération des données envoyées chiffrées en clair. & ? & OK \\
\hline
\end{tabular}
\end{figure}


\begin{figure}[!h]
\begin{tabular}{|p{1cm}|p{5cm}|p{5cm}|p{2cm}|p{2cm}|}
\hline
\multicolumn{3}{|l|}{Objet testé: déchiffrement de données} & \multicolumn{2}{|l|}{Version: 1.0} \\
\hline
\multicolumn{5}{|l|}{Objectif de test: vérifier si les données reçues peuvent être déchiffrées.} \\
\hline
\multicolumn{5}{|l|}{Procédure n°12: déchiffrer avec la clef associée.} \\
\hline
N° & Actions & Résultats attendus & Exigence & OK/NOK \\
\hline
1 & Déchiffrer des données à partir de la carte avec la clef privée correspondante. & Données déchiffrées. & F-Gl-40 & OK \\
\hline
\end{tabular}
\end{figure}



\begin{figure}[!h]
\begin{tabular}{|p{1cm}|p{5cm}|p{5cm}|p{2cm}|p{2cm}|}
\hline
\multicolumn{5}{|l|}{Procédure n°13: déchiffrer avec une clef invalide} \\
\hline
N° & Actions & Résultats attendus & Exigence & OK/NOK \\
\hline
1 & Déchiffrer des données à partir de la carte avec une clef privée non correspondante. & Données non déchiffrées & F-Gl-40 & OK \\
\hline
\end{tabular}
\end{figure}



\begin{figure}[!h]
\begin{tabular}{|p{1cm}|p{5cm}|p{5cm}|p{2cm}|p{2cm}|}
\hline
\multicolumn{3}{|l|}{Objet testé: Signature/Vérification de données} & \multicolumn{2}{|l|}{Version: 1.0} \\
\hline
\multicolumn{5}{|l|}{Objectif de test: Signer des données et vérifier la signature} \\
\hline
\multicolumn{5}{|l|}{Procédure n°14: Signer/Vérrifier.} \\
\hline
N° & Actions & Résultats attendus & Exigence & OK/NOK \\
\hline
1 & Signer des données à partir de la carte avec la clef privée. & Données signées. & F-Gl-50 & OK \\
\hline
\end{tabular}
\end{figure}


\begin{figure}[!h]
\begin{tabular}{|p{1cm}|p{5cm}|p{5cm}|p{2cm}|p{2cm}|}
\hline
\multicolumn{5}{|l|}{Procédure n°15: Vérification avec la clef publique associée} \\
\hline
N° & Actions & Résultats attendus & Exigence & OK/NOK \\
\hline
1 & Vérifier des données signées à partir de la carte avec la clef publique correspondante. & Données vérifiées. & F-Gl-50 & OK \\
\hline
\end{tabular}
\end{figure}


\begin{figure}[!h]
\begin{tabular}{|p{1cm}|p{5cm}|p{5cm}|p{2cm}|p{2cm}|}
\hline
\multicolumn{5}{|l|}{Procédure n°16: vérification avec une clef publique invalide} \\
\hline
N° & Actions & Résultats attendus & Exigence & OK/NOK \\
\hline
1 & Vérifier des données signées à partir de la carte avec la clef publique non correspondante. & Données non vérifiées. & F-Gl-50 & OK \\
\hline
\end{tabular}
\end{figure}



\begin{figure}[!h]
\begin{tabular}{|p{1cm}|p{5cm}|p{5cm}|p{2cm}|p{2cm}|}
\hline
\multicolumn{3}{|l|}{Objet testé: Administration des cartes} & \multicolumn{2}{|l|}{Version: 1.0} \\
\hline
\multicolumn{5}{|l|}{Objectif de test: vérifier si nous pouvons administrer les cartes} \\
\hline
\multicolumn{5}{|l|}{Procédure n°17: Attribuer un code PIN} \\
\hline
N° & Actions & Résultats attendus & Exigence & OK/NOK \\
\hline
1 & Attribution d'un code PIN à un utilisateur. & L'utilisateur posséde un code PIN qui lui est propre. & F-Gl-60 & OK \\
\hline
\end{tabular}
\end{figure}

\paragraph{}
\begin{tabular}{|p{1cm}|p{5cm}|p{5cm}|p{2cm}|p{2cm}|}
\hline
\multicolumn{5}{|l|}{Procédure n°18: insertion de données sur la carte} \\
\hline
N° & Actions & Résultats attendus & Exigence & OK/NOK \\
\hline
1 & Insérer des données qui soit propre à l'utilisateur (login, mot de passe,...). & La carte contient les données. & F-Gl-60 & OK \\
\hline
\end{tabular}

\paragraph{}
\begin{tabular}{|p{1cm}|p{5cm}|p{5cm}|p{2cm}|p{2cm}|}
\hline
\multicolumn{5}{|l|}{Procédure n°19: modification des données sur la carte} \\
\hline
N° & Actions & Résultats attendus & Exigence & OK/NOK \\
\hline
1 & Modification de données préalablement insérer (login, mot de passe, code PIN,...). & Données modifiées. & F-Gl-60 & NC \\
\hline
\end{tabular}



\section{Couverture de tests}
Ce tableau reprend les exigences de la STB et précise, pour chacune d’entre elles, la méthode
de vérification (démonstration / tests) et une description de celles ci.



\begin{figure}[!h]
\begin{tabular}{|p{2cm}|p{2.5cm}|p{2cm}|p{9cm}|}
\hline
Exigence & Méthode de vérification & Procédure utilisée & Commentaire \\
\hline
F-Gl-10 & Démonstration & Procédure 1 & Ce test consiste à utiliser une fonction disponible via la librairie javacard pour générer un nombre aléatoire. \\
\hline
F-Gl-10 & Test & Procédure 2 & Le test consiste à générer plusieurs nombres aléatoires (millions) et les soumettre à un test stastistique pour évaluer le niveau de l'aléatoire. \\
\hline
F-Gl-10 & Test & Procédure 3 & Le test consiste à générer plusieurs nombres aléatoires (10000) et nous faisons la moyenne afin de connaitre le temps d'éxecution moyen pour la génération d'un nombre aléatoire, qu'on considérera valide si inférieur à 300ms. \\
\hline
F-Gl-20 & Test & Procédure 4 & Le test consiste à débloquer la carte en s'authentifiant auprès de celle ci en entrant le code PIN associé. \\
\hline
F-Gl-20 & Test & Procédure 5 & Le test consiste à entrer 10 code PIN erronés afin de verrouiller la carte, et entrer ensuite le bon code PIN pour vérifier le verrouillage. \\
\hline
F-Gl-20 & Test & Procédure 6 & Le test consiste à déverrouiller la carte après avoir entré 10 codes PINs erroné en entrant le code PUK valide. \\
\hline
F-Gl-30 & Test & Procédure 7 & Le consiste à envoyer des données contenu sur la carte. \\
\hline
? & Test & Procédure 8 & Le test consiste à envoyer des données chiffré via une commincation sécurisé par un chiffrement symétrique. \\
\hline
? & Test & Procédure 9 & Le test consiste à envoyer des données authentifiées sans altérer le contenu. Nous devons ensuite vérifier que le contenu est bien intégre. \\
\hline 
? & Test & Procédure 10 & Le test consiste à envoyer des données authentifiées en altérant le contenu. Nous devons ensuite pouvoir voir une détection de modification du contenu. \\
\hline 
? & Test & Procédure 11 & Le test consiste à recevoir des données au préalable chiffrées, et vérifier si elles ont bien été déchiffrées. \\
\hline
F-Gl-40 & Test & Procédure 12 & Le test consiste à déchiffrer des données avec la clef privée correspondante, nous devrions alors obtenir le clair associé. \\
\hline
F-Gl-40 & Test & Procédure 13 & Le test consiste à déchiffrer des données avec la clef non valide, nous devrions alors obtenir une erreur. \\
\hline
F-Gl-50 & Test & Procédure 14 & Le test consiste à signer des données à partir de la clef privée stockée sur la carte. \\
\hline
F-Gl-50 & Test & Procédure 15 & Le test consiste à vérifier avec la clef publique correspondante des données au préalable signées. Les données doivent être alors vérifiées. \\
\hline
F-Gl-50 & Test & Procédure 16 & Le test consiste à vérifier avec une clef publique non correspondante des données au préalable signées. Les données ne sont alors pas vérifiées. \\
\hline
F-Gl-60 & Démonstration & Procédure 17 & Dans cette procédure, nous affectons à un utilisateur, donc à la carte lui appartenant, un code PIN lui étant propre afin de pouvoir débloquer la carte et l'utiliser. \\
\hline
F-Gl-60 & Démonstration & Procédure 18 & Dans cette procédure, nous insérons des données dans une carte telles qu'un mot de passe, un login,... \\
\hline
F-Gl-60 & Test & Procédure 19 & Ce test consiste à modifier les données précédemment insérer sur la carte. \\
\hline
\end{tabular}
\end{figure}





\end{document}
