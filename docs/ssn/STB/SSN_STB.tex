%!TEX TS-program = xelatex
%!TEX encoding = UTF-8 Unicode
\documentclass[a4paper,11pt,french]{article}

\usepackage[latin1]{inputenc}
\usepackage[french]{babel}
\usepackage{chngpage}
\usepackage[colorlinks=true, linkcolor=black, urlcolor=blue]{hyperref}
\usepackage{fontspec,xltxtra,xunicode,color}
\usepackage{tabularx}
\usepackage[table]{xcolor}
\usepackage{fancyhdr}
\usepackage{longtable}
\usepackage{lastpage} %pour compter le nombre de pages

\definecolor{gris}{rgb}{0.95, 0.95, 0.95}

\hypersetup{breaklinks=true}


\addtolength{\hoffset}{-2cm}
\addtolength{\textwidth}{4cm}
\addtolength{\topmargin}{-2cm}
\addtolength{\textheight}{4cm}
\addtolength{\headsep}{0.8cm} 
\addtolength{\footskip}{-0.3cm}

%les structures de gestion de projet
\usepackage{../../../lib/tex/structures} 


\def\projectName{Smart Social Network}
\def\docType{Spécification Technique de Besoin}
\def\version{0.1}
\def\author{\begin{tabular}{l}Baptiste \textsc{Dolbeau},\\Florian \textsc{Guilbert}\end{tabular}}
\def\dateStb{\today}
\def\checked{}
\def\approved{}



\begin{document}
\makeFirstPage
\clearpage
\vspace*{1cm}
%Tableau de mises à jour
\begin{center}
\textbf{\huge{MISES À JOUR}}\\
\vspace*{3cm}
	\begin{tabularx}{16cm}{|c|c|X|}
	\hline
	\bfseries{Version} & \bfseries{Date} & \bfseries{Modifications réalisées}\\
	\hline
	0.1 & 02/01/2013 & Création\\
	\hline
	\end{tabularx}
\end{center}

\clearpage
\tableofcontents
\clearpage


\section{Objet}
\renewcommand\labelitemi{\textbullet} %style des puces
\renewcommand\labelitemii{$\circ$} %style des puces 2e niveau
Proposer des solutions cryptographiques garantissant la protection de la vie 
privée des utilisateurs vis-à-vis d'un réseau social. Cette protection pourra
être effective par le chiffrement systématique des données sensibles. Et
le déchiffrement de ces données ne serait possible que par des personnes
considérées explicitement par l'utilisateur.

% TODO préciser
Le projet prendra la forme d'une extension pour le navigateur \emph{Mozilla Firefox}
s'interfaçant avec une carte à puce pour effectuer certaines tâches de chiffrement.

Il ne sera pas nécessaire de créer un compte, notre projet pourra fonctionner comme
un patch sur un compte déjà existant.

\paragraph{}
Le réseau social étudié sera \emph{Facebook} à moins que lors du développement
du projet des problèmes spécifiques à ce réseau social soient rencontrés. 
Par conséquent, la terminologie utilisée correspond à celle de \emph{Facebook} (statut, mur, ...).


\section{Documents applicables et de référence}
\begin{itemize}
    \item Manuel d'utilisation;
    \item proxy-encryption.pdf, le sujet du projet.
\end{itemize}


\section{Terminologie et sigles utilisés}
\begin{description}
	\item[SN :] Social Network, représente le réseau social que nous avons choisi comme support
    pour ce projet;
    % Proxy ?
\end{description}


\section{Exigences fonctionnelles}

\subsection{Présentation de la mission du produit logiciel}
\newcounter{FGcount}
\begin{tabularx}{16cm}{|c|X|l|c|}
\hline
\rowcolor{blue}~{\color{white}\bfseries{Reference}}&~{\color{white}\bfseries{Fonctionalité Globale}}&~{\color{white}\bfseries{Acteur}}&~{\color{white}\bfseries{Priorité}}\\
\hline
\addtocounter{FGcount}{10}
F-Gl-\arabic{FGcount} & Chiffrer/déchiffrer un statut & Utilisateur & \cellcolor{green!50}Indispensable \\
\hline
\addtocounter{FGcount}{10}
F-Gl-\arabic{FGcount} & Chiffrer/déchiffrer un document & Utilisateur & \cellcolor{green!50}Indispensable \\
\hline
\addtocounter{FGcount}{10}
F-Gl-\arabic{FGcount} & Gérer les liens d'amitiés & Utilisateur & \cellcolor{red!20}Important \\
\hline
\addtocounter{FGcount}{10}
F-Gl-\arabic{FGcount} & Chiffrer/déchiffrer un commentaire & Utilisateur & \cellcolor{blue!50}Secondaire\\
\hline
\end{tabularx}

\subsection{Chiffrer/déchiffrer un statut}
Un utilisateur peut lorsqu'il souhaite écrire un message
sur son mur le chiffrer. Il choisit, dans ce cas, les amis
qui peuvent déchiffrer ce message.

Inversement, lorsqu'un de ses amis poste (sur son mur) un message chiffré,
l'utilisateur peut tenter de le déchiffrer. Si, l'utilisateur
fait partie des personnes autorisées, il peut lire le message.


\fiche
	{Chiffrement d'un message sur son mur}
	{Utilisateur}
	{L'utilisateur chiffre un message qui sera affiché sur le mur}
	{}
	{L'utilisateur souhaite poster un message sur son mur}
	{L'utilisateur a posté un message chiffré sur son mur, lisible que des personnes autorisées}
	{\begin{itemize}
	    \item[]
	  \item[1.] L'utilisateur saisie un message et choisi de le chiffrer,
          il spécifie les personnes autorisées;
	\end{itemize}
	}
	{\begin{itemize}
        \item[]
		\item[2.] Le "proxy" chiffre le message avec une clef 
        de chiffrement, récupère les clefs publiques
        des personnes autorisées et chiffre la clef de chiffrement
        avec ces clefs;
		\item[3.] le "proxy" envoi ensuite une concaténation de 
        ce message et des clefs chiffrées aux serveurs de Facebook.
	\end{itemize}
	}
	{}
\flots
    {\begin{itemize}
    \item[]
    \item[2.] Si une des personnes choisie n'a pas de clef publique,
        elle ne pourra pas déchiffrer le message;
    \end{itemize}
    }
	{}    
\\*

\fiche
	{Déchiffrement d'un message sur un mur}
	{Utilisateur}
	{L'utilisateur déchiffre un message du mur d'un de ses amis}
	{}
    % TODO à discuter
	{L'utilisateur est arrivé sur une page contenant un message chiffré}
	{L'utilisateur a déchiffré un message}
    {}
	{\begin{itemize}
        \item[]
		\item[1.] Le "proxy" tente de déchiffrer la clef de chiffrement
            du message avec la clef publique de l'utilisateur;
		\item[2.] le "proxy" déchiffre tout le message avec la clef
            de chiffrement.
	\end{itemize}
	}
	{}
\flots
    {}
    {\begin{itemize}
    \item[]
    \item[1.] Si l'utilisateur ne fait pas partie des personnes
        autorisées, il ne pourra pas déchiffrer le message.
    \end{itemize}
    }
	{}    
\\*

\subsection{Chiffrer/déchiffrer un document}
Un utilisateur peut choisir utiliser l'option de téléversement
d'image du réseau social pour téléverser un document (image, 
fichier texte, ...) chiffré. Celui-ci sera considéré par une image
par le réseau social.\\*

\fiche
	{Chiffrement d'un document}
	{Utilisateur}
	{L'utilisateur chiffre un document qui sera interpréter comme une image par Facebook}
	{}
	{L'utilisateur souhaite téléverser un document}
	{L'utilisateur a téléverser un document, lisible que des personnes autorisées}
	{\begin{itemize}
	    \item[]
	  \item[1.] L'utilisateur téléverse un document et choisi de le chiffrer,
          il spécifie les personnes autorisées;
	\end{itemize}
	}
	{\begin{itemize}
        \item[]
		\item[2.] Le "proxy" chiffre le document avec une clef 
        de chiffrement, récupère les clefs publiques
        des personnes autorisées et chiffre la clef de chiffrement
        avec ces clefs;
		\item[3.] le "proxy" envoi ensuite une concaténation de 
        du document chiffré et des clefs chiffrées aux serveurs de Facebook.
	\end{itemize}
	}
	{}
\flots
    {\begin{itemize}
    \item[]
    \item[1.] Si l'utilisateur spécifie un document qui n'est pas une image
    et choisi de ne pas le chiffrer, cela sera refusé par Facebook;
    \item[2.] Si une des personnes choisie n'a pas de clef publique,
        elle ne pourra pas déchiffrer le message;
    \end{itemize}
    }
	{}    
\\*

\fiche
	{Déchiffrement d'un document}
	{Utilisateur}
	{L'utilisateur déchiffre un document d'un de ses amis}
	{}
    % TODO à discuter
	{L'utilisateur est arrivé sur une page contenant un document chiffré}
	{L'utilisateur a déchiffré un message}
    {}
	{\begin{itemize}
        \item[]
		\item[1.] Le "proxy" tente de déchiffrer la clef de chiffrement
            du document avec la clef publique de l'utilisateur;
		\item[2.] le "proxy" déchiffre tout le document avec la clef
            de chiffrement, la télécharge dans le cas ou ce n'est pas
            une image.
	\end{itemize}
	}
	{}
\flots
    {}
    {\begin{itemize}
    \item[]
    \item[1.] Si l'utilisateur ne fait pas partie des personnes
        autorisées, il ne pourra pas déchiffrer le document.
    \end{itemize}
    }
	{}    
\\*

\subsection{Gérer les liens d'amitiés}
Afin d'améliorer l'ergonomie des opérations de chiffrement, 
l'utilisateur aura la possibilité d'organiser ses amis en
différents groupes. \\*

% TODO à discuter
\fiche
	{}
	{}
	{}
	{}
	{}
	{}
	{\begin{itemize}
	    \item[]
		\item[1.] toto
	\end{itemize}
	}
	{\begin{itemize}
        \item[]
		\item[2.] tutu
	\end{itemize}
	}
	{}
\flots
    {}
	{}    
\\*

\subsection{Chiffrer/déchiffrer un commentaire}
De même que pour les messages de statut (de mur), l'utilisateur
peut chiffrer un commentaire ou au contraire en déchiffrer,
s'il fait partie des personnes autorisées.

\fiche
	{Chiffrement d'un commentaire}
	{Utilisateur}
	{L'utilisateur chiffre un commentaire}
	{}
	{L'utilisateur souhaite chiffrer un commentaire}
	{L'utilisateur a chiffré un commentaire, lisible que des personnes autorisées}
	{\begin{itemize}
	    \item[]
	  \item[1.] L'utilisateur saisie un commentaire et choisi de le chiffrer,
          il spécifie les personnes autorisées à le déchiffrer;
	\end{itemize}
	}
	{\begin{itemize}
        \item[]
		\item[2.] Le "proxy" chiffre le commentaire avec une clef 
        de chiffrement, récupère les clefs publiques
        des personnes autorisées et chiffre la clef de chiffrement
        avec ces clefs;
		\item[3.] le "proxy" envoi ensuite une concaténation de 
        du commentaire chiffré et des clefs chiffrées aux serveurs de Facebook.
	\end{itemize}
	}
	{}
\flots
    {\begin{itemize}
    \item[]
    \item[2.] Si une des personnes choisie n'a pas de clef publique,
        elle ne pourra pas déchiffrer le message;
    \end{itemize}
    }
	{}    
\\*

\fiche
	{Déchiffrement d'un commentaire}
	{Utilisateur}
	{L'utilisateur déchiffre un commentaire d'un de ses amis, présent sur la page}
	{}
    % TODO à discuter
	{L'utilisateur est arrivé sur une page contenant un commentaire chiffré}
	{L'utilisateur a déchiffré un commentaire}
    {}
	{\begin{itemize}
        \item[]
		\item[1.] Le "proxy" tente de déchiffrer la clef de chiffrement
            du commentaire avec la clef publique de l'utilisateur;
		\item[2.] le "proxy" déchiffre le commentaire avec la clef
            de chiffrement et l'affiche
	\end{itemize}
	}
	{}
\flots
    {}
    {\begin{itemize}
    \item[]
    \item[1.] Si l'utilisateur ne fait pas partie des personnes
        autorisées, il ne pourra pas déchiffrer le commentaire.
    \end{itemize}
    }
	{}    

\pagebreak
\subsection{Exigences fonctionnelles détaillées}

TODO

\newcounter{FNcount}

\begin{longtable}{|p{2cm}|p{10cm}|p{2.5cm}|}

% Header for the first page of the table
\multicolumn{1}{|>{\color{white}\columncolor{blue}}l|}{\bfseries{Reference}} & \multicolumn{1}{|>{\color{white}\columncolor{blue}}l|}{\bfseries{Fonctionalité}} & \multicolumn{1}{|>{\color{white}\columncolor{blue}}l|}{\bfseries{Priorité}}
\endfirsthead
% Header for next pages of the table
\multicolumn{1}{|>{\color{white}\columncolor{blue}}l|}{\bfseries{Reference}} & \multicolumn{1}{|>{\color{white}\columncolor{blue}}l|}{\bfseries{Fonctionalité}} & \multicolumn{1}{|>{\color{white}\columncolor{blue}}l|}{\bfseries{Priorité}}
\endhead

%This is the footer for all pages except the last page of the table...
\endfoot
%This is the footer for the last page of the table...
\endlastfoot

\hline
\addtocounter{FNcount}{10}
F-FN-\arabic{FNcount} & L'utilisateur ne donne son mot de passe qu'au SAUD &
\cellcolor{green!50}Indispensable \\
\hline
\addtocounter{FNcount}{10}
F-FN-\arabic{FNcount} & L'authentification sur un service externe ne peut se
faire qu'à partir d'un emplacement donnée &
\cellcolor{blue!50}Secondaire \\
\hline
\addtocounter{FNcount}{10}
F-FN-\arabic{FNcount} & Plusieurs authenfication simultanées sont impossible, 
le SAUD déconnecte la session en cours en cas d'une autre demande
d'authenfication &
\cellcolor{green!50}Indispensable \\
\hline
\addtocounter{FNcount}{10}
F-FN-\arabic{FNcount} & À chaque tentative de connexion d'un client sur un
service externe, celui-ci vérifie l'authenticité du SAUD associé au client &
\cellcolor{green!50}Indispensable \\
\hline
\addtocounter{FNcount}{10}
F-FN-\arabic{FNcount} & Les échanges et dialogues entre service externe, SAUD
et utilisateur sont sécurisés &
\cellcolor{green!50}Indispensable \\
\hline
\end{longtable}


\begin{longtable}{|p{2cm}|p{10cm}|p{2.5cm}|}

% Header for the first page of the table
\multicolumn{1}{|>{\color{white}\columncolor{blue}}l|}{\bfseries{Reference}} &
\multicolumn{1}{|>{\color{white}\columncolor{blue}}l|}{\bfseries{Fonctionalité}}
& \multicolumn{1}{|>{\color{white}\columncolor{blue}}l|}{\bfseries{Priorité}}
\endfirsthead
% Header for next pages of the table
\multicolumn{1}{|>{\color{white}\columncolor{blue}}l|}{\bfseries{Reference}} &
\multicolumn{1}{|>{\color{white}\columncolor{blue}}l|}{\bfseries{Fonctionalité}}
& \multicolumn{1}{|>{\color{white}\columncolor{blue}}l|}{\bfseries{Priorité}}
\endhead

%This is the footer for all pages except the last page of the table...
\endfoot
%This is the footer for the last page of the table...
\endlastfoot

\hline
\addtocounter{FNcount}{10}
F-FN-\arabic{FNcount} & Le SAUD est accessible au travers d'une page web &
\cellcolor{green!50}Indispensable \\
\hline
\addtocounter{FNcount}{10}
F-FN-\arabic{FNcount} & La page d'accueil du site propose un formulaire
d'authentification
ainsi que le lien/bouton "Creer un compte"  & \cellcolor{green!50}Indispensable
\\
\hline
\addtocounter{FNcount}{10}
F-FN-\arabic{FNcount} & Un visiteur peut créer un compte utilisateur &
\cellcolor{green!50}Indispensable \\
\hline
\addtocounter{FNcount}{10}
F-FN-\arabic{FNcount} & Un login unique, un mot de passe, une question secrête
ainsi que sa réponse et une adresse mail valide est essentiel pour créer un
compte utilisateur
 & \cellcolor{green!50}Indispensable \\
\hline
\addtocounter{FNcount}{10}
F-FN-\arabic{FNcount} & Quand un visiteur créé un compte d'utilisateur, un mail
lui est envoyé
et sans accusé de reception, le compte n'est pas créé &
\cellcolor{green!50}Indispensable \\
\hline
\addtocounter{FNcount}{10}
F-FN-\arabic{FNcount} & Si un utilisateur veut suprimer son compte, il doit
entrer son mot de passe
 & \cellcolor{blue!50}Secondaire \\
\hline
\addtocounter{FNcount}{10}
F-FN-\arabic{FNcount} & Un utilisateur peut se déconnecter du SAUD en cliquant
sur un lien & \cellcolor{red!50}Important \\
\hline
\addtocounter{FNcount}{10}
F-FN-\arabic{FNcount} & Un utilisateur peut modifier son mot de passe &
\cellcolor{blue!50}Secondaire \\
\hline
\addtocounter{FNcount}{10}
F-FN-\arabic{FNcount} & Un utilisateur peut changer sa question secrête
(intitulé et réponse) & \cellcolor{blue!50}Secondaire\\
\hline
\addtocounter{FNcount}{10}
F-FN-\arabic{FNcount} & Un utilisateur peut modifier son adresse mail &
\cellcolor{green!50}Indispensable \\
\hline
\addtocounter{FNcount}{10}
F-FN-\arabic{FNcount} & Un utilisateur peut se connecter à un site quelconque
 en donnant son adresse SAUD dans le champs "identifiant" du site &
\cellcolor{green!50}Indispensable \\
\hline
\addtocounter{FNcount}{10}
F-FN-\arabic{FNcount} & Un utilisateur peut créer un compte sur un site
quelconque
 en utilisant l'authentification externe & \cellcolor{green!50}Indispensable \\
\hline
\addtocounter{FNcount}{10}
F-FN-\arabic{FNcount} & Lorsqu'un utilisateur se connecte depuis une autre
adresse IP pour la première fois, il doit répondre à sa question secrête et 
rentrer son adresse mail & \cellcolor{green!50}Indispensable \\
\hline
\addtocounter{FNcount}{10}
F-FN-\arabic{FNcount} & L'utilisateur peut modifier son mot de passe
peut entrer un nouveau en suivant la procédure adéquate &
\cellcolor{green!50}Indispensable \\
\hline
\addtocounter{FNcount}{10}
F-FN-\arabic{FNcount} & L'administrateur peut supprimer le compte de n'importe
quel utilisateur & \cellcolor{blue!50}Secondaire \\
\hline
\addtocounter{FNcount}{10}
F-FN-\arabic{FNcount} & L'administrateur n'aura jamais accès aux mot de passes
des utilisateurs  & \cellcolor{blue!50}Secondaire \\
\hline
\addtocounter{FNcount}{10}
F-FN-\arabic{FNcount} & Il est possible de télécharger un paquetage
d'installation d'un SAUD peronnel & \cellcolor{blue!50}Secondaire \\
\hline
\addtocounter{FNcount}{10}
F-FN-\arabic{FNcount} & Log de la dernière connexion à chaque nouvelle connexion
& \cellcolor{blue!50}Secondaire \\
\hline
\end{longtable}
\pagebreak

\section{Exigences opérationnelles}

\newcounter{FOcount}

\begin{longtable}{|p{2cm}|p{10cm}|p{2.5cm}|}

% Header for the first page of the table
\multicolumn{1}{|>{\color{white}\columncolor{blue}}l|}{\bfseries{Reference}} & \multicolumn{1}{|>{\color{white}\columncolor{blue}}l|}{\bfseries{Fonctionalité}} & \multicolumn{1}{|>{\color{white}\columncolor{blue}}l|}{\bfseries{Priorité}}
\endfirsthead
% Header for next pages of the table
\multicolumn{1}{|>{\color{white}\columncolor{blue}}l|}{\bfseries{Reference}} & \multicolumn{1}{|>{\color{white}\columncolor{blue}}l|}{\bfseries{Fonctionalité}} & \multicolumn{1}{|>{\color{white}\columncolor{blue}}l|}{\bfseries{Priorité}}
\endhead

%This is the footer for all pages except the last page of the table...
\endfoot
%This is the footer for the last page of the table...
\endlastfoot

\hline
\addtocounter{FOcount}{10}
F-FO-\arabic{FOcount} & Le chiffrement n'est pas trop long (> 2s) & \cellcolor{green!50}Indispensable \\
\hline
\end{longtable}

\section{Exigences d'interface}

\newcounter{FIcount}

\begin{longtable}{|p{2cm}|p{10cm}|p{2.5cm}|}

% Header for the first page of the table
\multicolumn{1}{|>{\color{white}\columncolor{blue}}l|}{\bfseries{Reference}} & \multicolumn{1}{|>{\color{white}\columncolor{blue}}l|}{\bfseries{Fonctionalité}} & \multicolumn{1}{|>{\color{white}\columncolor{blue}}l|}{\bfseries{Priorité}}
\endfirsthead
% Header for next pages of the table
\multicolumn{1}{|>{\color{white}\columncolor{blue}}l|}{\bfseries{Reference}} & \multicolumn{1}{|>{\color{white}\columncolor{blue}}l|}{\bfseries{Fonctionalité}} & \multicolumn{1}{|>{\color{white}\columncolor{blue}}l|}{\bfseries{Priorité}}
\endhead

%This is the footer for all pages except the last page of the table...
\endfoot
%This is the footer for the last page of the table...
\endlastfoot

\hline
\addtocounter{FIcount}{10}
F-FI-\arabic{FIcount} & Notre système d'interface avec \emph{Mozilla Firefox}
& \cellcolor{green!50}Indispensable \\
\hline
\end{longtable}

\section{Exigences de qualité}

\newcounter{FQcount}

\begin{longtable}{|p{2cm}|p{10cm}|p{2.5cm}|}

% Header for the first page of the table
\multicolumn{1}{|>{\color{white}\columncolor{blue}}l|}{\bfseries{Reference}} & \multicolumn{1}{|>{\color{white}\columncolor{blue}}l|}{\bfseries{Fonctionalité}} & \multicolumn{1}{|>{\color{white}\columncolor{blue}}l|}{\bfseries{Priorité}}
\endfirsthead
% Header for next pages of the table
\multicolumn{1}{|>{\color{white}\columncolor{blue}}l|}{\bfseries{Reference}} & \multicolumn{1}{|>{\color{white}\columncolor{blue}}l|}{\bfseries{Fonctionalité}} & \multicolumn{1}{|>{\color{white}\columncolor{blue}}l|}{\bfseries{Priorité}}
\endhead

%This is the footer for all pages except the last page of the table...
\endfoot
%This is the footer for the last page of the table...
\endlastfoot

\hline
\addtocounter{FQcount}{10}
F-FQ-\arabic{FQcount} & La système sera livré pour le 04 mars 2013 & \cellcolor{green!50}Indispensable \\
\hline
\addtocounter{FQcount}{10}
F-FQ-\arabic{FQcount} & Un manuel d'utilisation sera livré avec le système
& \cellcolor{green!50}Indispensable \\
\hline
\end{longtable}

\section{Exigences de réalisation}

\newcounter{FRcount}

\begin{longtable}{|p{2cm}|p{10cm}|p{2.5cm}|}

% Header for the first page of the table
\multicolumn{1}{|>{\color{white}\columncolor{blue}}l|}{\bfseries{Reference}} & \multicolumn{1}{|>{\color{white}\columncolor{blue}}l|}{\bfseries{Fonctionalité}} & \multicolumn{1}{|>{\color{white}\columncolor{blue}}l|}{\bfseries{Priorité}}
\endfirsthead
% Header for next pages of the table
\multicolumn{1}{|>{\color{white}\columncolor{blue}}l|}{\bfseries{Reference}} & \multicolumn{1}{|>{\color{white}\columncolor{blue}}l|}{\bfseries{Fonctionalité}} & \multicolumn{1}{|>{\color{white}\columncolor{blue}}l|}{\bfseries{Priorité}}
\endhead

%This is the footer for all pages except the last page of the table...
\endfoot
%This is the footer for the last page of the table...
\endlastfoot

\hline
\addtocounter{FRcount}{10}
F-FR-\arabic{FRcount} & TODO & \cellcolor{green!50}Indispensable \\
\hline
\end{longtable}


\end{document}
