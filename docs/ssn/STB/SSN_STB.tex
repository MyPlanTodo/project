%!TEX TS-program = xelatex
%!TEX encoding = UTF-8 Unicode

\documentclass[a4paper,11pt,french]{article}

%Import des packages utilisés pour le document
\usepackage[latin1]{inputenc}
\usepackage[french]{babel}
\usepackage{chngpage}
\usepackage[colorlinks=true,linkcolor=black,urlcolor=blue]{hyperref}
\usepackage{graphicx, amssymb, color, listings}
\usepackage{fontspec,xltxtra,xunicode,color}
\usepackage{tabularx, longtable}
\usepackage[table]{xcolor}
\usepackage{fancyhdr}
\usepackage{tikz}
\usetikzlibrary{shapes}
\usepackage{lastpage}

\definecolor{gris}{rgb}{0.95, 0.95, 0.95}

%Redéfinition des marges
\addtolength{\hoffset}{-2cm}
\addtolength{\textwidth}{4cm}
\addtolength{\topmargin}{-2cm}
\addtolength{\textheight}{1cm}
\addtolength{\headsep}{0.8cm} 
\addtolength{\footskip}{1cm}


%Import page de garde et structures pour la gestion de projet
\usepackage{res/structures} 

%Variables
\def\matiere{Conduite de Projet}
\def\filiere{Master 2 SSI}
\def\projectDesc{Smart Social Network}
\def\projectName{\emph{SSN}~}
\def\completeName{\projectDesc ~- \projectName}
\def\docType{Spécification technique des besoins}
\def\docDate{\today}
\def\version{0.1}
\def\author{Florian \textsc{Guilbert}}
\def\checked{}
\def\approved{}


% -- Début du document -- %
\begin{document}
%Page de garde
\makeFirstPage
\clearpage

%Tableau de mises à jour
\vspace*{1cm}
\begin{center}
\textbf{\huge{MISES À JOUR}}\\
\vspace*{3cm}
	\begin{tabularx}{16cm}{|c|c|X|}
	\hline
	\bfseries{Version} & \bfseries{Date} & \bfseries{Modifications réalisées}\\
	\hline
	0.1 & 07/01/2013 & Création\\
	\hline
	&&\\
	\hline
	&&\\
	\hline
	\end{tabularx}
\end{center}

%La table des matières
\clearpage
\tableofcontents
\clearpage

\section{Objet}
\renewcommand\labelitemi{\textbullet} %style des puces
\renewcommand\labelitemii{$\circ$} %style des puces 2e niveau
Proposer des solutions cryptographiques garantissant la protection de la vie 
privée des utilisateurs vis-à-vis d'un réseau social. Cette protection pourra
être effective par le chiffrement systématique des données sensibles. Et
le déchiffrement de ces données ne serait possible que par des personnes
considérées explicitement par l'utilisateur.

% TODO préciser
Le projet prendra la forme d'une extension pour le navigateur 
\emph{Mozilla Firefox}
s'interfaçant avec une carte à puce pour effectuer certaines tâches de 
chiffrement.

Il ne sera pas nécessaire de créer un compte, notre projet pourra fonctionner 
comme un patch sur un compte déjà existant.

\paragraph{}
Le réseau social étudié sera \emph{Facebook} à moins que lors du développement
du projet des problèmes spécifiques à ce réseau social soient rencontrés. 
Par conséquent, la terminologie utilisée correspond à celle de \emph{Facebook} 
(statut, mur, ...).

\section{Documents applicables et de référence}
\begin{itemize}
    \item Manuel d'utilisation;
    \item proxy-encryption.pdf, le sujet du projet.
\end{itemize}


\section{Terminologie et sigles utilisés}
\begin{description}
	\item[SN :] Social Network, représente le réseau social que nous avons 
    choisi comme support
    pour ce projet;
    \item[FaceCrypt :] application Java gérant les traitements lourds 
    (chiffrement) de l'extension et étant en relation avec la carte à puce.
    \item[Extension :] programme incorporé dans le navigateur permettant
    de manipuler les pages de \emph{Facebook}.
    \item[SoftCard :] Application effectuant le relais entre la carte
        et FaceCrypt.
\end{description}

\section{Exigences fonctionnelles}

\subsection{Présentation de la mission du produit logiciel}
\newcounter{FGcount}
\begin{tabularx}{16cm}{|c|X|l|c|}
\hline
\rowcolor{blue}~{\color{white}\bfseries{Reference}}&~{\color{white}\bfseries{Fonctionalité Globale}}&~{\color{white}\bfseries{Acteur}}&~{\color{white}\bfseries{Priorité}}\\
\hline
\addtocounter{FGcount}{10}
F-Gl-\arabic{FGcount} & Chiffrer/déchiffrer un statut & Utilisateur & \cellcolor{green!50}Indispensable \\
\hline
\addtocounter{FGcount}{10}
F-Gl-\arabic{FGcount} & Chiffrer/déchiffrer un document & Utilisateur & \cellcolor{green!50}Indispensable \\
\hline
\addtocounter{FGcount}{10}
F-Gl-\arabic{FGcount} & Gérer les liens d'amitiés & Utilisateur & \cellcolor{red!20}Important \\
\hline
\addtocounter{FGcount}{10}
F-Gl-\arabic{FGcount} & Chiffrer/déchiffrer un commentaire & Utilisateur & \cellcolor{blue!50}Secondaire\\
\hline
\end{tabularx}

\subsection{Préconditions}
Pour tous les cas d'utilisation décrits ci-dessous, nous supposons que 
l'utilisateur est déjà authentifié sur le réseau social (Facebook), il
a donc déjà inséré sa carte dans le lecteur.

\subsection{Chiffrer/déchiffrer un statut}
Un utilisateur peut lorsqu'il souhaite écrire un message
sur son mur le chiffrer. Il choisit, dans ce cas, les amis
qui peuvent déchiffrer ce message.

Inversement, lorsqu'un de ses amis poste (sur son mur) un message chiffré,
l'utilisateur peut tenter de le déchiffrer. Si, l'utilisateur
fait partie des personnes autorisées, il peut lire le message.\\

\fiche
	{Chiffrement d'un message sur son mur}
	{Utilisateur}
	{L'utilisateur chiffre un message qui sera affiché sur le mur}
	{}
	{L'utilisateur souhaite poster un message sur son mur}
	{L'utilisateur a posté un message chiffré sur son mur, lisible que des personnes autorisées}
	{\begin{itemize}
	    \item[]
	  \item[1.] L'utilisateur saisie un message et choisi de le chiffrer,
          il spécifie les personnes autorisées;
      \item[3.] L'utilisateur précise des listes d'amis ou des amis, 
          qui pourront lire son message;
	\end{itemize}
	}
	{\begin{itemize}
        \item[]
        \item[2.] L'extension demande à l'utilisateur quels amis vont-être
            autorisés à déchiffrer le message
		\item[4.] L'extension récupère le message avant l'envoi sur le serveur 
            de Facebook et l'envoi à FaceCrypt qui va le chiffrer avec une clef 
        de chiffrement, récupèrer les clefs publiques des personnes autorisées 
        et chiffre la clef de chiffrement avec ces clefs;
		\item[3.] FaceCrypt envoi ensuite une concaténation de 
        ce message et des clefs chiffrées à l'extension qui enverra le tout
        chiffré sur les serveurs de Facebook.
	\end{itemize}
    }
	{}
\flots
    {}
    {\begin{itemize}
    \item[]
    \item[2.] Si une des personnes choisie n'a pas de clef publique,
        elle ne pourra pas déchiffrer le message;
    \end{itemize}
    }
    {}
\\*

\fiche
	{Déchiffrement d'un message sur un mur}
	{Utilisateur}
	{L'utilisateur déchiffre un message du mur d'un de ses amis}
	{}
    % TODO à discuter
	{L'utilisateur souhaite déchiffrer un message}
	{L'utilisateur a déchiffré un message, ou pas}
    {\begin{itemize}
        \item[]
        \item[1.] L'utilisateur appuie sur le bouton pour 
            déchiffrer un message;
    \end{itemize}
    }
	{\begin{itemize}
        \item[]
		\item[1.] L'extension récupère le message et l'envoi à 
            FaceCrypt;
        \item[2.] FaceCrypt envoie la concaténation des clefs à la carte
            pour qu'elle tente de déchiffrer avec la clef privée de 
            l'utilisateur la clef de chiffrement du message;
		\item[3.] FaceCrypt reçoit la clef de chiffrement de message;
        \item[4.] FaceCrypt déchiffre tout le message avec la clef
            de chiffrement et envoi le message à l'extension.
        \item[5.] L'extension affiche le résultat;
	\end{itemize}
	}
	{}
\flots
    {}
    {\begin{itemize}
    \item[]
    \item[1.] Si l'utilisateur ne fait pas partie des personnes
        autorisées, il ne pourra pas déchiffrer le message.
    \end{itemize}
    }
	{}    
\\*

\subsection{Chiffrer/déchiffrer un document}
Un utilisateur peut choisir utiliser l'option de téléversement
d'image du réseau social pour téléverser un document (image, 
fichier texte, ...) chiffré. Celui-ci sera considéré par une image
par le réseau social.\\*

\fiche
	{Chiffrement d'un document}
	{Utilisateur}
	{L'utilisateur chiffre un document qui sera interpréter comme une image 
        par Facebook}
	{}
	{L'utilisateur souhaite téléverser un document}
	{L'utilisateur a téléverser un document, lisible que des personnes 
        autorisées}
	{\begin{itemize}
	    \item[]
	  \item[1.] L'utilisateur téléverse un document et choisi de le chiffrer,
          il spécifie les personnes autorisées;
	\end{itemize}
	}
	{\begin{itemize}
        \item[]
		\item[2.] FaceCrypt chiffre le document avec une clef 
        de chiffrement, récupère les clefs publiques
        des personnes autorisées et chiffre la clef de chiffrement
        avec ces clefs;
		\item[3.] FaceCrypt envoi ensuite une concaténation de 
        du document chiffré et des clefs chiffrées aux serveurs de Facebook.
	\end{itemize}
	}
	{}
\flots
    {\begin{itemize}
    \item[]
    \item[1.] Si l'utilisateur spécifie un document qui n'est pas une image
    et choisi de ne pas le chiffrer, cela sera refusé par Facebook;
    \item[2.] Si une des personnes choisie n'a pas de clef publique,
        elle ne pourra pas déchiffrer le message;
    \end{itemize}
    }
	{}    
\\*

\fiche
	{Déchiffrement d'un document}
	{Utilisateur}
	{L'utilisateur déchiffre un document d'un de ses amis}
	{}
    % TODO à discuter
	{L'utilisateur est arrivé sur une page contenant un document chiffré}
	{L'utilisateur a déchiffré un message}
    {}
	{\begin{itemize}
        \item[]
		\item[1.] FaceCrypt tente de déchiffrer la clef de chiffrement
            du document avec la clef publique de l'utilisateur;
		\item[2.] FaceCrypt déchiffre tout le document avec la clef
            de chiffrement, la télécharge dans le cas ou ce n'est pas
            une image.
	\end{itemize}
	}
	{}
\flots
    {}
    {\begin{itemize}
    \item[]
    \item[1.] Si l'utilisateur ne fait pas partie des personnes
        autorisées, il ne pourra pas déchiffrer le document.
    \end{itemize}
    }
	{}    
\\*

\subsection{Gérer les liens d'amitiés}
Afin d'améliorer l'ergonomie des opérations de chiffrement, 
l'utilisateur aura la possibilité d'organiser ses amis en
différents groupes. \\*

(Cas d'utilisation encore à déterminer)

\fiche
	{Création d'une liste d'amis}
	{Utilisateur}
    {L'utilisateur crée une liste d'amis}
    % TODO à discuter
    {}
    {L'utilisateur souhaite créer une liste d'amis}
	{L'utilisateur a créé une liste d'amis}
    {\begin{itemize}
        \item[]
        \item[1.] L'utilisateur appuie sur le "bouton" création
            d'une liste;
        \item[3.] L'utilisateur entre le nom de sa liste et valide;
        
    \end{itemize}}
	{\begin{itemize}
        \item[]
		\item[2.] FaceCrypt ouvre une popup pour inviter l'utilisateur
            à choisir un nom pour sa liste;
            du document avec la clef publique de l'utilisateur;
		\item[4.] FaceCrypt créé une liste d'amis vide;
	\end{itemize}
	}
	{}
\flots
    {\begin{itemize}
    \item[]
    \item[1.] Si l'utilisateur met un nom trop long à sa liste
    \end{itemize}
    }
    {}
\\*

\subsection{Chiffrer/déchiffrer un commentaire}
De même que pour les messages de statut (de mur), l'utilisateur
peut chiffrer un commentaire ou au contraire en déchiffrer,
s'il fait partie des personnes autorisées.

\fiche
	{Chiffrement d'un commentaire}
	{Utilisateur}
	{L'utilisateur chiffre un commentaire}
	{}
	{L'utilisateur souhaite chiffrer un commentaire}
	{L'utilisateur a chiffré un commentaire, lisible que des personnes 
        autorisées}
	{\begin{itemize}
	    \item[]
	  \item[1.] L'utilisateur saisie un commentaire et choisi de le chiffrer,
          il spécifie les personnes autorisées à le déchiffrer;
	\end{itemize}
	}
	{\begin{itemize}
        \item[]
		\item[2.] FaceCrypt chiffre le commentaire avec une clef 
        de chiffrement, récupère les clefs publiques
        des personnes autorisées et chiffre la clef de chiffrement
        avec ces clefs;
		\item[3.] FaceCrypt envoi ensuite une concaténation de 
        du commentaire chiffré et des clefs chiffrées aux serveurs de Facebook.
	\end{itemize}
	}
	{}
\flots
    {\begin{itemize}
    \item[]
    \item[2.] Si une des personnes choisie n'a pas de clef publique,
        elle ne pourra pas déchiffrer le message;
    \end{itemize}
    }
	{}    
\\*

\fiche
	{Déchiffrement d'un commentaire}
	{Utilisateur}
	{L'utilisateur déchiffre un commentaire d'un de ses amis, présent sur la
        page}
	{}
    % TODO à discuter
	{L'utilisateur est arrivé sur une page contenant un commentaire chiffré}
	{L'utilisateur a déchiffré un commentaire}
    {}
	{\begin{itemize}
        \item[]
		\item[1.] FaceCrypt tente de déchiffrer la clef de chiffrement
            du commentaire avec la clef publique de l'utilisateur;
		\item[2.] FaceCrypt déchiffre le commentaire avec la clef
            de chiffrement et l'affiche
	\end{itemize}
	}
	{}
\flots
    {}
    {\begin{itemize}
    \item[]
    \item[1.] Si l'utilisateur ne fait pas partie des personnes
        autorisées, il ne pourra pas déchiffrer le commentaire.
    \end{itemize}
    }
	{}    

\pagebreak
\subsection{Exigences fonctionnelles détaillées}


\newcounter{FNcount}

\begin{longtable}{|p{2cm}|p{10cm}|p{2.5cm}|}

% Header for the first page of the table
\multicolumn{1}{|>{\color{white}\columncolor{blue}}l|}{\bfseries{Reference}} &
\multicolumn{1}{|>{\color{white}\columncolor{blue}}l|}{\bfseries{Fonctionalité}}
& \multicolumn{1}{|>{\color{white}\columncolor{blue}}l|}{\bfseries{Priorité}}
\endfirsthead
% Header for next pages of the table
\multicolumn{1}{|>{\color{white}\columncolor{blue}}l|}{\bfseries{Reference}} &
\multicolumn{1}{|>{\color{white}\columncolor{blue}}l|}{\bfseries{Fonctionalité}}
& \multicolumn{1}{|>{\color{white}\columncolor{blue}}l|}{\bfseries{Priorité}}
\endhead

%This is the footer for all pages except the last page of the table...
\endfoot
%This is the footer for the last page of the table...
\endlastfoot

\hline
\addtocounter{FNcount}{10}
F-FN-\arabic{FNcount} & L'utilisateur peut chiffrer un message sur son mur &
\cellcolor{green!50}Indispensable \\
\hline
\addtocounter{FNcount}{10}
F-FN-\arabic{FNcount} & L'utilisateur peut tenter de déchiffrer un message d'un mur de ses amis, 
en appuyant sur un bouton & \cellcolor{green!50}Indispensable\\
\hline
\addtocounter{FNcount}{10}
F-FN-\arabic{FNcount} & L'utilisateur peut chiffrer un document &
\cellcolor{green!50}Indispensable \\
\hline
\addtocounter{FNcount}{10}
F-FN-\arabic{FNcount} & L'utilisateur peut tenter déchiffrer un document, 
en appuyant sur le bouton & \cellcolor{green!50}Indispensable\\
\hline
\addtocounter{FNcount}{10}
F-FN-\arabic{FNcount} & L'utilisateur peut chiffrer un commentaire &
\cellcolor{blue!50}Secondaire \\
\hline
\addtocounter{FNcount}{10}
F-FN-\arabic{FNcount} & L'utilisateur peut déchiffrer un commentaire, 
en appuyant sur le bouton & \cellcolor{blue!50}Secondaire\\
\hline
\addtocounter{FNcount}{10}
F-FN-\arabic{FNcount} & L'utilisateur peut créer une liste d'amis & \cellcolor{green!50}Indispensable \\
\hline 
\addtocounter{FNcount}{10}
F-FN-\arabic{FNcount} & L'utilisateur peut effacer une liste d'amis & \cellcolor{green!50}Indispensable \\
\hline 
\addtocounter{FNcount}{10}
F-FN-\arabic{FNcount} & L'utilisateur peut ajouter un ami dans une liste & \cellcolor{green!50}Indispensable \\
\hline
\addtocounter{FNcount}{10}
F-FN-\arabic{FNcount} & L'utilisateur peut retirer un ami d'une liste &
\cellcolor{green!50}Indispensable \\
\hline
\end{longtable}
\pagebreak

\section{Exigences opérationnelles}

\newcounter{FOcount}

\begin{longtable}{|p{2cm}|p{10cm}|p{2.5cm}|}

% Header for the first page of the table
\multicolumn{1}{|>{\color{white}\columncolor{blue}}l|}{\bfseries{Reference}} & \multicolumn{1}{|>{\color{white}\columncolor{blue}}l|}{\bfseries{Fonctionalité}} & \multicolumn{1}{|>{\color{white}\columncolor{blue}}l|}{\bfseries{Priorité}}
\endfirsthead
% Header for next pages of the table
\multicolumn{1}{|>{\color{white}\columncolor{blue}}l|}{\bfseries{Reference}} & \multicolumn{1}{|>{\color{white}\columncolor{blue}}l|}{\bfseries{Fonctionalité}} & \multicolumn{1}{|>{\color{white}\columncolor{blue}}l|}{\bfseries{Priorité}}
\endhead

%This is the footer for all pages except the last page of the table...
\endfoot
%This is the footer for the last page of the table...
\endlastfoot

\hline
\addtocounter{FOcount}{10}
F-FO-\arabic{FOcount} & Le chiffrement n'est pas trop long & \cellcolor{green!50}Indispensable \\
\hline
\end{longtable}

\section{Exigences d'interface}

\newcounter{FIcount}

\begin{longtable}{|p{2cm}|p{10cm}|p{2.5cm}|}

% Header for the first page of the table
\multicolumn{1}{|>{\color{white}\columncolor{blue}}l|}{\bfseries{Reference}} & \multicolumn{1}{|>{\color{white}\columncolor{blue}}l|}{\bfseries{Fonctionalité}} & \multicolumn{1}{|>{\color{white}\columncolor{blue}}l|}{\bfseries{Priorité}}
\endfirsthead
% Header for next pages of the table
\multicolumn{1}{|>{\color{white}\columncolor{blue}}l|}{\bfseries{Reference}} & \multicolumn{1}{|>{\color{white}\columncolor{blue}}l|}{\bfseries{Fonctionalité}} & \multicolumn{1}{|>{\color{white}\columncolor{blue}}l|}{\bfseries{Priorité}}
\endhead

%This is the footer for all pages except the last page of the table...
\endfoot
%This is the footer for the last page of the table...
\endlastfoot

\hline
\addtocounter{FIcount}{10}
F-FI-\arabic{FIcount} & Notre système s'interface avec \emph{Mozilla Firefox}
& \cellcolor{green!50}Indispensable \\
\hline
\addtocounter{FIcount}{10}
F-FI-\arabic{FIcount} & Notre système fonctionnera comme un \emph{patch} : il pourra fonctionner sur d'un compte déjà existant
& \cellcolor{green!50}Indispensable \\
\hline
\addtocounter{FIcount}{10}
F-FI-\arabic{FIcount} & Un bouton permet à l'utilisateur de déchiffrer un message qui lui apparaît chiffré
& \cellcolor{green!50}Indispensable \\
\hline
\end{longtable}

\section{Exigences de qualité}

\newcounter{FQcount}

\begin{longtable}{|p{2cm}|p{10cm}|p{2.5cm}|}

% Header for the first page of the table
\multicolumn{1}{|>{\color{white}\columncolor{blue}}l|}{\bfseries{Reference}} & \multicolumn{1}{|>{\color{white}\columncolor{blue}}l|}{\bfseries{Fonctionalité}} & \multicolumn{1}{|>{\color{white}\columncolor{blue}}l|}{\bfseries{Priorité}}
\endfirsthead
% Header for next pages of the table
\multicolumn{1}{|>{\color{white}\columncolor{blue}}l|}{\bfseries{Reference}} & \multicolumn{1}{|>{\color{white}\columncolor{blue}}l|}{\bfseries{Fonctionalité}} & \multicolumn{1}{|>{\color{white}\columncolor{blue}}l|}{\bfseries{Priorité}}
\endhead

%This is the footer for all pages except the last page of the table...
\endfoot
%This is the footer for the last page of the table...
\endlastfoot

\hline
\addtocounter{FQcount}{10}
F-FQ-\arabic{FQcount} & La système sera livré pour le 04 mars 2013 & \cellcolor{green!50}Indispensable \\
\hline
\addtocounter{FQcount}{10}
F-FQ-\arabic{FQcount} & Un manuel d'utilisation sera livré avec le système
& \cellcolor{green!50}Indispensable \\
\hline
\addtocounter{FQcount}{10}
F-FQ-\arabic{FQcount} & Les mots de passes utilisés par l'utilisateurs doivent avoir une taille conséquente pour améliorer leur sécurité
& \cellcolor{green!50}Indispensable \\
\hline
\end{longtable}

\section{Exigences de réalisation}

\newcounter{FRcount}

\begin{longtable}{|p{2cm}|p{10cm}|p{2.5cm}|}

% Header for the first page of the table
\multicolumn{1}{|>{\color{white}\columncolor{blue}}l|}{\bfseries{Reference}} & \multicolumn{1}{|>{\color{white}\columncolor{blue}}l|}{\bfseries{Fonctionalité}} & \multicolumn{1}{|>{\color{white}\columncolor{blue}}l|}{\bfseries{Priorité}}
\endfirsthead
% Header for next pages of the table
\multicolumn{1}{|>{\color{white}\columncolor{blue}}l|}{\bfseries{Reference}} & \multicolumn{1}{|>{\color{white}\columncolor{blue}}l|}{\bfseries{Fonctionalité}} & \multicolumn{1}{|>{\color{white}\columncolor{blue}}l|}{\bfseries{Priorité}}
\endhead

%This is the footer for all pages except the last page of the table...
\endfoot
%This is the footer for the last page of the table...
\endlastfoot

\hline
\addtocounter{FRcount}{10}
F-FR-\arabic{FRcount} & Seul le mot de passe nécessite de ne jamais être transmis en clair & \cellcolor{green!50}Indispensable \\
\hline
\end{longtable}


\end{document}

\end{document}
