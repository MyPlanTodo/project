%!TEX TS-program = xelatex
%!TEX encoding = UTF-8 Unicode

\documentclass[a4paper,11pt,french]{article}

%Import des packages utilisés pour le document
\usepackage[french]{babel}
\usepackage{chngpage}
\usepackage[colorlinks=true,linkcolor=black,urlcolor=blue]{hyperref}
\usepackage{graphicx, amssymb, color, listings}
\usepackage{fontspec,xltxtra,xunicode,color}
\usepackage{tabularx, longtable}
\usepackage[table]{xcolor}
\usepackage{fancyhdr}
\usepackage{tikz}
\usetikzlibrary{shapes}
\usepackage{lastpage}

\definecolor{gris}{rgb}{0.95, 0.95, 0.95}

%Redéfinition des marges
\addtolength{\hoffset}{-2cm}
\addtolength{\textwidth}{4cm}
\addtolength{\topmargin}{-2cm}
\addtolength{\textheight}{1cm}
\addtolength{\headsep}{0.8cm} 
\addtolength{\footskip}{1cm}


%Import page de garde et structures pour la gestion de projet
\usepackage{res/structures} 

%Variables
\def\matiere{Conduite de Projet}
\def\filiere{Master 2 SSI}
\def\projectDesc{Smart Social Network}
\def\projectName{\emph{SSN}~}
\def\completeName{\projectDesc ~- \projectName}
\def\docType{Spécification technique des besoins}
\def\docDate{\today}
\def\version{1.0}
\def\author{Florian \textsc{Guilbert}}
\def\checked{Baptiste \textsc{Dolbeau}}
\def\approved{}


% -- Début du document -- %
\begin{document}
%Page de garde
\makeFirstPage
\clearpage

%Tableau de mises à jour
\vspace*{1cm}
\begin{center}
\textbf{\huge{MISES À JOUR}}\\
\vspace*{3cm}
	\begin{tabularx}{16cm}{|c|c|X|}
	\hline
	\bfseries{Version} & \bfseries{Date} & \bfseries{Modifications réalisées}\\
	\hline
	0.1 & 07/01/2013 & Création\\
	\hline
    1.0 & 29/01/2013 & Relecture par Baptiste \textsc{Dolbeau}\\
	\hline
	&&\\
	\hline
	\end{tabularx}
\end{center}

%La table des matières
\clearpage
\tableofcontents
\clearpage

\section{Objet}
\renewcommand\labelitemi{\textbullet} %style des puces
\renewcommand\labelitemii{$\circ$} %style des puces 2e niveau
Proposer des solutions cryptographiques garantissant la protection de la vie 
privée des utilisateurs vis-à-vis d'un réseau social. Cette protection pourra
être effective par le chiffrement systématique des données sensibles. Et
le déchiffrement de ces données ne sera possible que par des personnes
considérées explicitement par l'utilisateur.


Le projet prendra la forme d'une extension pour le navigateur 
\emph{Mozilla Firefox}
s'interfaçant avec une carte à puce pour effectuer certaines tâches de 
chiffrement.

Il ne sera pas nécessaire de créer un compte, notre projet pourra fonctionner 
comme patch sur un compte déjà existant.

\paragraph{}
Le réseau social étudié sera \emph{Facebook} à moins que des problèmes spécifiques
à ce réseau social soient rencontrés lors du développement du projet.
Par conséquent, la terminologie utilisée correspondra à celle de \emph{Facebook} 
(statut, mur, ...).

\section{Documents applicables et de référence}
\begin{itemize}
    \item Manuel d'utilisation.
    \item proxy-encryption.pdf, le sujet du projet.
\end{itemize}


\section{Terminologie et sigles utilisés}
\begin{description}
	\item[SN :] Social Network, représente le réseau social que nous 
	avons choisi comme support pour ce projet.
    \item[FaceCrypt :] Application Java gérant les traitements lourds 
    (chiffrement/déchiffrement) de l'extension et étant en relation avec
	la carte à puce.
    \item[Extension :] Programme incorporé dans le navigateur permettant
    de manipuler les pages de \emph{Facebook}.
    \item[SoftCard :] Application effectuant le relais entre la carte
        et FaceCrypt.
\end{description}

\section{Exigences fonctionnelles}

\subsection{Présentation de la mission du produit logiciel}
\newcounter{FGcount}
\begin{tabularx}{16cm}{|c|X|l|c|}
\hline
\rowcolor{blue}~{\color{white}\bfseries{Référence}}&~{\color{white}\bfseries{Fonctionalité Globale}}&~{\color{white}\bfseries{Acteur}}&~{\color{white}\bfseries{Priorité}}\\
\hline
\addtocounter{FGcount}{10}
F-Gl-\arabic{FGcount} & Chiffrer/déchiffrer un statut & Utilisateur & \cellcolor{green!50}Indispensable \\
\hline
\addtocounter{FGcount}{10}
F-Gl-\arabic{FGcount} & Chiffrer/déchiffrer un document & Utilisateur & \cellcolor{green!50}Indispensable \\
\hline
\addtocounter{FGcount}{10}
F-Gl-\arabic{FGcount} & Gérer les liens d'amitiés & Utilisateur & \cellcolor{red!20}Important \\
\hline
\addtocounter{FGcount}{10}
F-Gl-\arabic{FGcount} & Chiffrer/déchiffrer un commentaire & Utilisateur & \cellcolor{blue!50}Secondaire\\
\hline
\end{tabularx}

\subsection{Préconditions}
Pour tous les cas d'utilisation décrits ci-dessous, nous supposons que 
l'utilisateur est déjà authentifié sur le réseau social (Facebook). Il
a donc déjà inséré sa carte dans le lecteur.

\subsection{Chiffrer/déchiffrer un statut}
Un utilisateur peut, lorsqu'il le souhaite, écrire un message
sur son mur et le chiffrer. Il choisit dans ce cas les amis
qui peuvent déchiffrer ce message.

Inversement, lorsqu'un de ses amis poste (sur son mur) un message chiffré,
l'utilisateur peut tenter de le déchiffrer. Si l'utilisateur
fait partie des personnes autorisées, il pourra lire le message.\\

\fiche
	{Chiffrement d'un message sur son mur}
	{Utilisateur}
	{L'utilisateur chiffre un message qui sera affiché sur le mur}
	{}
	{L'utilisateur souhaite poster un message sur son mur}
	{L'utilisateur a posté un message chiffré sur son mur, lisible uniquement par les personnes autorisées}
	{\begin{itemize}
	    \item[]
	  \item[1.] L'utilisateur saisi un message et choisit de le chiffrer,
          il spécifie les personnes autorisées.
      \item[3.] L'utilisateur précise des listes d'amis ou des amis, 
          qui pourront lire son message.
	\end{itemize}
	}
	{\begin{itemize}
        \item[]
        \item[2.] L'extension demande à l'utilisateur quels amis vont être
            autorisés à déchiffrer le message.
		\item[4.] L'extension récupère le message avant son envoi sur le serveur 
            de Facebook et l'envoie à FaceCrypt qui va le chiffrer avec une clef 
        de chiffrement, récupèrer les clefs publiques des personnes autorisées 
        et chiffrer la clef de chiffrement avec ces clefs.
		\item[5.] FaceCrypt envoie ensuite une concaténation de 
        ce message et des clefs chiffrées à l'extension qui enverra le tout,
        chiffré, sur les serveurs de Facebook.
	\end{itemize}
    }
	{}
\flots
    {}
    {\begin{itemize}
    \item[]
    \item[2.] Si une des personnes choisie n'a pas de clef publique,
        elle ne pourra pas déchiffrer le message.
    \end{itemize}
    }
    {}
\\*

\fiche
	{Déchiffrement d'un message sur un mur}
	{Utilisateur}
	{L'utilisateur déchiffre un message du mur d'un de ses amis}
	{}
	{L'utilisateur souhaite déchiffrer un message}
	{L'utilisateur a déchiffré un message, ou pas}
    {\begin{itemize}
        \item[]
        \item[1.] L'utilisateur appuie sur le bouton pour 
            déchiffrer le message.
    \end{itemize}
    }
	{\begin{itemize}
        \item[]
		\item[2.] L'extension récupère le message et l'envoie à 
            FaceCrypt.
        \item[3.] FaceCrypt déchiffre tout le message avec la clef
            de chiffrement et envoie le message à l'extension.
        \item[4.] L'extension affiche le résultat.
	\end{itemize}
	}
	{}
\flots
    {}
    {\begin{itemize}
    \item[]
    \item[1.] Si l'utilisateur ne fait pas partie des personnes
        autorisées, il ne pourra pas déchiffrer le message.
    \end{itemize}
    }
	{}    
\\*

\subsection{Chiffrer/déchiffrer un document}
Un utilisateur peut choisir d'utiliser l'option de téléversement
d'image du réseau social pour téléverser un document (image, 
fichier texte, ...) chiffré. Celui-ci sera considéré comme une image
par le réseau social.\\*

\fiche
	{Chiffrement d'un document}
	{Utilisateur}
	{L'utilisateur chiffre un document qui sera interprété comme une image 
        par Facebook}
	{}
	{L'utilisateur souhaite téléverser un document}
	{L'utilisateur a téléversé un document, lisible uniquement par les personnes 
        autorisées}
	{\begin{itemize}
	    \item[]
	  \item[1.] L'utilisateur téléverse un document et choisit de le chiffrer.
      \item[3.] L'utilisateur précise des listes d'amis ou des amis, 
          qui pourront lire son message.
	\end{itemize}
	}
	{\begin{itemize}
        \item[]
		\item[2.] FaceCrypt chiffre le document avec une clef 
        de chiffrement, récupère les clefs publiques
        des personnes autorisées et chiffre la clef de chiffrement
        avec ces clefs.
		\item[4.] L'extension récupère le commentaire avant son envoi sur le 
            serveur de Facebook et l'envoie à FaceCrypt qui va chiffrer le 
            document avec une clef de chiffrement, récupérer  les clefs publiques
            des personnes autorisées et chiffrer la clef de chiffrement avec
            ces clefs.
		\item[5.] FaceCrypt envoie ensuite une concaténation de 
        ce message et des clefs chiffrées à l'extension qui enverra le tout,
        chiffré, aux serveurs de Facebook.
	\end{itemize}
	}
	{}
\flots
    {\begin{itemize}
    \item[]
    \item[1.] Si l'utilisateur spécifie un document qui n'est pas une image
    et choisit de ne pas le chiffrer, cela sera refusé par Facebook.
    \item[2.] Si une des personnes choisie n'a pas de clef publique,
        elle ne pourra pas déchiffrer le message.
    \end{itemize}
    }
	{}    
\\*

\fiche
	{Déchiffrement d'un document}
	{Utilisateur}
	{L'utilisateur déchiffre un document d'un de ses amis}
	{}
	{L'utilisateur souhaite déchiffrer un message}
	{L'utilisateur a déchiffré un message}
    {\begin{itemize}
        \item[]
        \item[1.] L'utilisateur appuie sur le bouton pour déchiffrer
            le document
    \end{itemize}
    }
	{\begin{itemize}
        \item[]
		\item[2.] L'extension récupère le message et l'envoi à 
            FaceCrypt.
		\item[3.] FaceCrypt déchiffre tout le document avec la clef
            de chiffrement, le télécharge dans le cas ou ce n'est pas
            une image, sinon le renvoie à l'extension.
        \item[4.] L'extension reçoit l'image et l'affiche.
	\end{itemize}
	}
	{}
\flots
    {}
    {\begin{itemize}
    \item[]
    \item[1.] Si l'utilisateur ne fait pas partie des personnes
        autorisées, il ne pourra pas déchiffrer le document.
    \end{itemize}
    }
	{}    
\\*

\subsection{Gérer les liens d'amitiés}
Afin d'améliorer l'ergonomie des opérations de chiffrement, 
l'utilisateur aura la possibilité d'organiser ses amis en
différents groupes. \\*

\fiche
	{Création d'une liste d'amis}
	{Utilisateur}
    {L'utilisateur crée une liste d'amis}
    {}
    {L'utilisateur souhaite créer une liste d'amis}
	{L'utilisateur a créé une liste d'amis}
    {\begin{itemize}
        \item[]        
        \item[1.] L'utilisateur appuie sur le bouton "Gestion des listes".
        \item[3.] L'utilisateur appuie sur le bouton "Ajouter une liste".
        \item[3.] L'utilisateur entre le nom de sa liste et valide.
    \end{itemize}}
	{\begin{itemize}
        \item[]		
        \item[2.] L'extension ouvre la fenêtre de gestion de liste.
	\item[4.] L'extension ouvre une popup pour inviter l'utilisateur
    		à choisir un nom pour sa liste.
	\item[5.] L'extension envoie une requête à FaceCrypt de création de 
		liste et actualise ses listes.
	\item[6.] FaceCrypt crée la liste dans sa base.
	\end{itemize}
	}
	{}
\flots
    {\begin{itemize}
    \item[]
    \item[1.] Si l'utilisateur met un nom trop long (> 128) à sa liste.
    \end{itemize}
    }
    {}
\\*

\fiche
	{Suppression d'une liste d'amis}
	{Utilisateur}
    {L'utilisateur supprime une liste existante d'amis}
    {}
    {L'utilisateur souhaite supprimer une liste d'amis}
	{L'utilisateur a supprimé une liste d'amis}
    {\begin{itemize}
        \item[]
        \item[1.] L'utilisateur appuie sur le bouton "Gestion des listes".
        \item[3.] L'utilisateur sélectionne une liste et appuie sur le bouton
        "Supprimer".
    \end{itemize}}
	{\begin{itemize}
        \item[]
		\item[2.] L'extension ouvre la fenêtre de gestion de liste.
		\item[4.] L'extension envoie la requête de suppression à FaceCrypt
		et réactualiste les listes.
		\item[5.] FaceCrypt supprime la liste de sa base.
	\end{itemize}
	}
	{}
\flots
    {}
    {}
\\*

\fiche
	{Ajout d'un ami dans une liste}
	{Utilisateur}
    {L'utilisateur ajoute un ami dans une de ses listes}
    {La liste d'amis doit déjà exister}
    {L'utilisateur souhaite ajouter un ami dans une des liste d'amis}
	{L'utilisateur a ajouté un ami dans une de ses liste d'amis}
    {\begin{itemize}
        \item[]
        \item[1.] L'utilisateur appuie sur le bouton "Gestion des listes".
        \item[3.] L'utilisateur sélectionne une liste et clique sur le bouton 
        "Gérer les amis".
        \item[5] L'utilisateur coche l'ami voulu et valide.
    \end{itemize}}
	{\begin{itemize}
        \item[]
        \item[2.] L'extension ouvre la fenêtre de gestion de liste.
		\item[4.] L'extension ouvre une popup contenant tous les amis de 
		l'utilisateur.
		\item[6.] L'extension envoie une requête d'actualisation de liste à
		 FaceCrypt et actualise sa liste.
		\item[7.] FaceCrypt ajoute un lien entre le(s) ami(s) et la liste
		dans sa base.
	\end{itemize}
	}
	{}
\flots
    {\begin{itemize}
    \item[]
    \item[1.] Si l'utilisateur choisit un ami qui ne dispose pas de notre 
    système, celui ne peut pas être ajouté.
    \end{itemize}
    }
    {}
\\*

\fiche
	{Suppression d'un ami dans une liste}
	{Utilisateur}
    {L'utilisateur supprime un ami d'une de ses listes}
    {La liste d'amis doit déjà exister, et l'ami doit y être présent}
    {L'utilisateur souhaite supprimer un ami dans une des liste d'amis}
	{L'utilisateur a supprimé un ami d'une de ses liste d'amis}
    {\begin{itemize}
        \item[]
        \item[1.] L'utilisateur appuie sur le bouton "Gestion des listes".
        \item[3.] L'utilisateur sélectionne une liste et clique sur le bouton 
        "Gérer les amis".
        \item[5] L'utilisateur décoche l'ami voulu et valide.
    \end{itemize}}
	{\begin{itemize}
        \item[]
        \item[2.] L'extension ouvre la fenêtre de gestion de liste.
		\item[4.] L'extension ouvre une popup contenant tous les amis de 
		l'utilisateur.
		\item[6.] L'extension envoie une requête d'actualisation de la liste à 
		FaceCrypt et actualise la liste.
		\item[7.] FaceCrypt supprime un lien entre le(s) ami(s) et la liste
		dans sa base.
	\end{itemize}
	}
	{}
\flots
    {}
    {}
\\*

\subsection{Chiffrer/déchiffrer un commentaire}
De même que pour les messages de statut (de mur), l'utilisateur
peut chiffrer ses commentaires ou au contraire en déchiffrer (
s'il fait partie des personnes autorisées).

\fiche
	{Chiffrement d'un commentaire}
	{Utilisateur}
	{L'utilisateur chiffre un commentaire}
	{}
	{L'utilisateur souhaite chiffrer un commentaire}
	{L'utilisateur a chiffré un commentaire, lisible uniquement par les 
	personnes autorisées}
	{\begin{itemize}
	    \item[]
	  \item[1.] L'utilisateur saisit un commentaire et choisit de le chiffrer.
      \item[3.] L'utilisateur précise les listes d'amis, ou les amis, 
          qui pourront lire son message.
	\end{itemize}
	}
	{\begin{itemize}
        \item[]
        \item[2.] L'extension demande à l'utilisateur quels amis vont-être
            autorisés à déchiffrer le message.
		\item[4.] L'extension récupère le commentaire avant son envoi sur le 
            serveur de Facebook et l'envoie à FaceCrypt qui va chiffrer le 
            commentaire avec une clef de chiffrement, récupèrer les clefs 
            publiques des personnes autorisées et chiffrer la clef de chiffrement
        avec ces clefs.
		\item[5.] FaceCrypt envoie ensuite une concaténation de 
        ce message et des clefs chiffrées à l'extension qui enverra le tout,
        chiffré, aux serveurs de Facebook.
	\end{itemize}
	}
	{}
\flots
    {\begin{itemize}
    \item[]
    \item[2.] Si une des personnes choisie n'a pas de clef publique,
        elle ne pourra pas déchiffrer le message.
    \end{itemize}
    }
	{}    
\\*

\fiche
	{Déchiffrement d'un commentaire}
	{Utilisateur}
	{L'utilisateur déchiffre un commentaire d'un de ses amis, présent sur la
        page}
	{}
	{L'utilisateur souhaite déchiffrer un commentaire}
	{L'utilisateur a déchiffré un commentaire}
    {\begin{itemize}
        \item[]
        \item[1.] L'utilisateur appuie sur le bouton pour 
            déchiffrer un commentaire.
    \end{itemize}
    }
	{\begin{itemize}
        \item[]
		\item[2.] L'extension récupère le commentaire et l'envoie à 
            FaceCrypt.
        \item[3.] FaceCrypt déchiffre tout le commentaire avec la clef
            de chiffrement et envoie le message à l'extension.
        \item[4.] L'extension affiche le résultat.
	\end{itemize}
	}
	{}
\flots
    {}
    {\begin{itemize}
    \item[]
    \item[1.] Si l'utilisateur ne fait pas partie des personnes
        autorisées, il ne pourra pas déchiffrer le commentaire.
    \end{itemize}
    }
	{}    

\pagebreak
\subsection{Exigences fonctionnelles détaillées}


\newcounter{FNcount}

\begin{longtable}{|p{2cm}|p{10cm}|p{2.5cm}|}

% Header for the first page of the table
\multicolumn{1}{|>{\color{white}\columncolor{blue}}l|}{\bfseries{Référence}} &
\multicolumn{1}{|>{\color{white}\columncolor{blue}}l|}{\bfseries{Fonctionalité}}
& \multicolumn{1}{|>{\color{white}\columncolor{blue}}l|}{\bfseries{Priorité}}
\endfirsthead
% Header for next pages of the table
\multicolumn{1}{|>{\color{white}\columncolor{blue}}l|}{\bfseries{Référence}} &
\multicolumn{1}{|>{\color{white}\columncolor{blue}}l|}{\bfseries{Fonctionalité}}
& \multicolumn{1}{|>{\color{white}\columncolor{blue}}l|}{\bfseries{Priorité}}
\endhead

%This is the footer for all pages except the last page of the table...
\endfoot
%This is the footer for the last page of the table...
\endlastfoot

\hline
\addtocounter{FNcount}{10}
F-FN-\arabic{FNcount} & L'utilisateur peut chiffrer un message sur son mur &
\cellcolor{green!50}Indispensable \\
\hline
\addtocounter{FNcount}{10}
F-FN-\arabic{FNcount} & L'utilisateur peut tenter de déchiffrer un message du 
mur d'un de ses amis, en appuyant sur un bouton & \cellcolor{green!50}Indispensable\\
\hline
\addtocounter{FNcount}{10}
F-FN-\arabic{FNcount} & L'utilisateur peut chiffrer un document &
\cellcolor{green!50}Indispensable \\
\hline
\addtocounter{FNcount}{10}
F-FN-\arabic{FNcount} & L'utilisateur peut tenter de déchiffrer un document,
en appuyant sur un bouton & \cellcolor{green!50}Indispensable\\
\hline
\addtocounter{FNcount}{10}
F-FN-\arabic{FNcount} & L'utilisateur peut chiffrer un commentaire &
\cellcolor{blue!50}Secondaire \\
\hline
\addtocounter{FNcount}{10}
F-FN-\arabic{FNcount} & L'utilisateur peut déchiffrer un commentaire, 
en appuyant sur un bouton & \cellcolor{blue!50}Secondaire\\
\hline
\addtocounter{FNcount}{10}
F-FN-\arabic{FNcount} & L'utilisateur peut créer une liste d'amis & \cellcolor{green!50}Indispensable \\
\hline 
\addtocounter{FNcount}{10}
F-FN-\arabic{FNcount} & L'utilisateur peut effacer une liste d'amis & \cellcolor{green!50}Indispensable \\
\hline 
\addtocounter{FNcount}{10}
F-FN-\arabic{FNcount} & L'utilisateur peut ajouter un ami dans une liste & \cellcolor{green!50}Indispensable \\
\hline
\addtocounter{FNcount}{10}
F-FN-\arabic{FNcount} & L'utilisateur peut retirer un ami d'une liste & \cellcolor{green!50}Indispensable \\
\hline
\end{longtable}
\pagebreak

\section{Exigences opérationnelles}

\newcounter{FOcount}

\begin{longtable}{|p{2cm}|p{10cm}|p{2.5cm}|}

% Header for the first page of the table
\multicolumn{1}{|>{\color{white}\columncolor{blue}}l|}{\bfseries{Référence}} & \multicolumn{1}{|>{\color{white}\columncolor{blue}}l|}{\bfseries{Fonctionalité}} & \multicolumn{1}{|>{\color{white}\columncolor{blue}}l|}{\bfseries{Priorité}}
\endfirsthead
% Header for next pages of the table
\multicolumn{1}{|>{\color{white}\columncolor{blue}}l|}{\bfseries{Référence}} & \multicolumn{1}{|>{\color{white}\columncolor{blue}}l|}{\bfseries{Fonctionalité}} & \multicolumn{1}{|>{\color{white}\columncolor{blue}}l|}{\bfseries{Priorité}}
\endhead

%This is the footer for all pages except the last page of the table...
\endfoot
%This is the footer for the last page of the table...
\endlastfoot

\hline
\addtocounter{FOcount}{10}
F-FO-\arabic{FOcount} & Le chiffrement n'est pas trop long & \cellcolor{green!50}Indispensable \\
\hline
\end{longtable}

\section{Exigences d'interface}

\newcounter{FIcount}

\begin{longtable}{|p{2cm}|p{10cm}|p{2.5cm}|}

% Header for the first page of the table
\multicolumn{1}{|>{\color{white}\columncolor{blue}}l|}{\bfseries{Référence}} & \multicolumn{1}{|>{\color{white}\columncolor{blue}}l|}{\bfseries{Fonctionalité}} & \multicolumn{1}{|>{\color{white}\columncolor{blue}}l|}{\bfseries{Priorité}}
\endfirsthead
% Header for next pages of the table
\multicolumn{1}{|>{\color{white}\columncolor{blue}}l|}{\bfseries{Référence}} & \multicolumn{1}{|>{\color{white}\columncolor{blue}}l|}{\bfseries{Fonctionalité}} & \multicolumn{1}{|>{\color{white}\columncolor{blue}}l|}{\bfseries{Priorité}}
\endhead

%This is the footer for all pages except the last page of the table...
\endfoot
%This is the footer for the last page of the table...
\endlastfoot

\hline
\addtocounter{FIcount}{10}
F-FI-\arabic{FIcount} & Notre système s'interface avec \emph{Mozilla Firefox}
& \cellcolor{green!50}Indispensable \\
\hline
\addtocounter{FIcount}{10}
F-FI-\arabic{FIcount} & Notre système s'utilisera comme un \emph{patch} : il pourra fonctionner sur un compte déjà existant
& \cellcolor{green!50}Indispensable \\
\hline
\addtocounter{FIcount}{10}
F-FI-\arabic{FIcount} & Un bouton permet à l'utilisateur de déchiffrer un message qui lui apparaît chiffré
& \cellcolor{green!50}Indispensable \\
\hline
\end{longtable}

\section{Exigences de qualité}

\newcounter{FQcount}

\begin{longtable}{|p{2cm}|p{10cm}|p{2.5cm}|}

% Header for the first page of the table
\multicolumn{1}{|>{\color{white}\columncolor{blue}}l|}{\bfseries{Référence}} & \multicolumn{1}{|>{\color{white}\columncolor{blue}}l|}{\bfseries{Fonctionalité}} & \multicolumn{1}{|>{\color{white}\columncolor{blue}}l|}{\bfseries{Priorité}}
\endfirsthead
% Header for next pages of the table
\multicolumn{1}{|>{\color{white}\columncolor{blue}}l|}{\bfseries{Référence}} & \multicolumn{1}{|>{\color{white}\columncolor{blue}}l|}{\bfseries{Fonctionalité}} & \multicolumn{1}{|>{\color{white}\columncolor{blue}}l|}{\bfseries{Priorité}}
\endhead

%This is the footer for all pages except the last page of the table...
\endfoot
%This is the footer for the last page of the table...
\endlastfoot

\hline
\addtocounter{FQcount}{10}
F-FQ-\arabic{FQcount} & La système sera livré pour le 01 mars 2013 & \cellcolor{green!50}Indispensable \\
\hline
\addtocounter{FQcount}{10}
F-FQ-\arabic{FQcount} & Un manuel d'utilisation sera livré avec le système
& \cellcolor{green!50}Indispensable \\
\hline
\addtocounter{FQcount}{10}
F-FQ-\arabic{FQcount} & Les mots de passes utilisés par l'utilisateurs doivent avoir une taille conséquente pour améliorer leur sécurité
& \cellcolor{green!50}Indispensable \\
\hline
\end{longtable}

\section{Exigences de réalisation}

\newcounter{FRcount}

\begin{longtable}{|p{2cm}|p{10cm}|p{2.5cm}|}

% Header for the first page of the table
\multicolumn{1}{|>{\color{white}\columncolor{blue}}l|}{\bfseries{Référence}} & \multicolumn{1}{|>{\color{white}\columncolor{blue}}l|}{\bfseries{Fonctionalité}} & \multicolumn{1}{|>{\color{white}\columncolor{blue}}l|}{\bfseries{Priorité}}
\endfirsthead
% Header for next pages of the table
\multicolumn{1}{|>{\color{white}\columncolor{blue}}l|}{\bfseries{Référence}} & \multicolumn{1}{|>{\color{white}\columncolor{blue}}l|}{\bfseries{Fonctionalité}} & \multicolumn{1}{|>{\color{white}\columncolor{blue}}l|}{\bfseries{Priorité}}
\endhead

%This is the footer for all pages except the last page of the table...
\endfoot
%This is the footer for the last page of the table...
\endlastfoot

\hline
\addtocounter{FRcount}{10}
F-FR-\arabic{FRcount} & Seul le mot de passe nécessite de ne jamais être transmis en clair & \cellcolor{green!50}Indispensable \\
\hline
\end{longtable}


\end{document}
