%!TEX TS-program = xelatex
%!TEX encoding = UTF-8 Unicode

\documentclass[a4paper,11pt,french]{article}

%Import des packages utilisés pour le document
\usepackage[french]{babel}
\usepackage{chngpage}
\usepackage[colorlinks=true,linkcolor=black,urlcolor=blue]{hyperref}
\usepackage{graphicx, amssymb, color, listings}
\usepackage{fontspec,xltxtra,xunicode,color}
\usepackage{tabularx, longtable}
\usepackage[table]{xcolor}
\usepackage{fancyhdr}
\usepackage{tikz}
\usetikzlibrary{shapes}
\usepackage{lastpage}

\definecolor{gris}{rgb}{0.95, 0.95, 0.95}

%Redéfinition des marges
\addtolength{\hoffset}{-2cm}
\addtolength{\textwidth}{4cm}
\addtolength{\topmargin}{-2cm}
\addtolength{\textheight}{1cm}
\addtolength{\headsep}{0.8cm} 
\addtolength{\footskip}{1cm}


%Import page de garde et structures pour la gestion de projet
\usepackage{res/structures} 

%Variables
\def\matiere{Conduite de Projet}
\def\filiere{Master 2 SSI}
\def\projectDesc{Smart Social Network}
\def\projectName{\emph{SSN}~}
\def\completeName{\projectDesc ~- \projectName}
\def\docType{Spécification technique de besoins}
\def\docDate{\today}
\def\version{1.4}
\def\author{Giovanni \textsc{Huet}, Romain \textsc{Pignard}}
\def\checked{Florian \textsc{Guilbert}, Emmanuel \textsc{Mocquet}}
\def\approved{}


% -- Début du document -- %
\begin{document}
%Page de garde
\makeFirstPage
\clearpage

%Tableau de mises à jour
\vspace*{1cm}
\begin{center}
\textbf{\huge{MISES À JOUR}}\\
\vspace*{3cm}
	\begin{tabularx}{16cm}{|c|c|X|}
	\hline
	\bfseries{Version} & \bfseries{Date} & \bfseries{Modifications réalisées}\\
	\hline
	0.1 & 26/11/2013 & Création\\
	\hline
    0.2 & 02/01/2013 & Ajout des cas d'utilisation\\
	\hline
    0.3 & 30/01/2013 & Modifications mineures\\
	\hline
    1.0 & 31/01/2013 & Relecture \\
	\hline
    1.1 & 06/02/2013 & Corrections après rdv client \\
	\hline
    1.2 & 06/02/2013 & Corrections \\
	\hline
    1.3 & 07/02/2013 & Corrections (modifications C10) \\
	\hline
    1.4 & 11/02/2013 & Modifications (priorité F-FI-30) \\
	\hline
	&&\\
	\hline
	\end{tabularx}
\end{center}

%La table des matières
\clearpage
\tableofcontents
\clearpage

\section{Objet}
\renewcommand\labelitemi{\textbullet} %style des puces
\renewcommand\labelitemii{$\circ$} %style des puces 2e niveau
Ce projet propose la mise en place de solutions cryptographiques pour 
sécuriser les données qu’un utilisateur place sur un réseau social au moyen 
d’authentifications fortes.

\paragraph{}
Il s’agirait donc ici de développer une extension pour le logiciel Mozilla 
Firefox permettant à l’utilisateur de gérer le chiffrement de ses données sur 
le réseau social Facebook. Cette extension utilisera
une application Java pour assurer les traitements lourds. Pour gérer 
l’authentification forte, cette application dialoguera avec une carte à puce 
qui contiendra les données sensibles de l’utilisateur (login/mot de passe), 
clef privée, ...

Le dialogue avec cette carte à puce se fera par l’intermédiaire d’un 
client Java. 

Ce projet est une composition de deux sous-projets :
\begin{itemize}
    \item Étude et mise en \oe{}uvre de solutions d’authentifications et de signatures 
        par cartes à puce, proposé par Magali \textsc{Bardet} ;
\item Solutions cryptographiques pour les réseaux sociaux, proposé par Ayoub 
    \textsc{Otmani} ;
\end{itemize}

\paragraph{}
Dans ce document, nous présentons le sous-projet SC (pour SmartCards) utilisé 
par l'entité FaceCrypt du sous-projet SSN (Secure Social Network), cette
utilisation est adaptable à d'autres situations.

\section{Documents applicables et de référence}
\begin{itemize}
    \item Manuel d'utilisation;
    \item Tutoriel d'installation;
    \item cartes-a-puce.pdf, le sujet du projet.
\end{itemize}


\section{Terminologie et sigles utilisés}
\begin{description}
    \item[CdR :] Cahier de Recettes;
    \item[AdR :] Analyse des Risques;
    \item[DAL :] Document d'Architecture Logicielle;
    \item[PdD :] Plan de développement;
    \item[STB :] Spécification Technique de Besoins;
    \item[SC :] \emph{SmartCard}, relatif au sous-projet sur les cartes à puce;
    \item[SSN :] \emph{Secure Social Network}, relatif au sous-projet sur 
        Facebook;
    \item[FaceCrypt :] Application Java gérant les traitements lourds 
    (chiffrement/déchiffrement) de l'extension et étant en relation avec
	la carte à puce;
    \item[IHM :] Interface Homme-Machine, (interface graphique); 
    \item[Utilisateur :] entité (humain ou programme) interagissant avec ce 
        sous-projet;
    \item[Système :] ce sous-projet;
    \item[Sécurisé :] ce terme sous-entend un chiffrement des données
        et une vérification de leur intégrité;
    \item[SoftCard :] application effectuant le relais entre la carte
        et FaceCrypt;
    \item[Extension :] programme incorporé dans le navigateur;
    \item[Aléatoire :] les résultats suivent une distribution de
        probabilité uniforme;
    \item[Pseudo-aléatoire :] les résultats sont indistingables en temps polynomial 
        d'une distribution de probabilité uniforme;
    \item[PRNG :] (Pseudo Random Number Generator) générateur de nombres
        pseudo-aléatoires;
    \item[RNG :] (Random Number Generator) générateur de nombres
        aléatoires;
    \item[PIN :] (Personal Identification Number) code servant à authentifier
        l'utilisateur;
    \item[PUK :] (Personal Unlock Key) code servant à débloquer la carte quand
        trop de codes PIN erronés ont été entrés.
\end{description}

\clearpage

\section{Exigences fonctionnelles}

\subsection{Présentation de la mission du produit logiciel}
\newcounter{FGcount}
\begin{tabularx}{16cm}{|c|X|l|c|}
\hline
\rowcolor{blue}~{\color{white}\bfseries{Référence}}&~{\color{white}\bfseries{Fonctionalité Globale}}&~{\color{white}\bfseries{Acteur}}&~{\color{white}\bfseries{Priorité}}\\
\hline
\addtocounter{FGcount}{10}
F-Gl-\arabic{FGcount} & Génération de nombres aléatoires & SmartCard, SoftCard, Utilisateur & \cellcolor{green!50}Indispensable \\
\hline
\addtocounter{FGcount}{10}
F-Gl-\arabic{FGcount} & Déblocage de la carte (\emph{via} authentification par code PIN et PUK) &
SmartCard, SoftCard, Utilisateur & \cellcolor{green!50}Indispensable \\
\hline
\addtocounter{FGcount}{10}
F-Gl-\arabic{FGcount} & Transmission de données & SmartCard, SoftCard, Utilisateur & \cellcolor{green!50}Indispensable \\
\hline
\addtocounter{FGcount}{10}
F-Gl-\arabic{FGcount} & Chiffrement/déchiffrement de données & SmartCard, SoftCard, Utilisateur & \cellcolor{green!50}Indispensable\\
\hline
\addtocounter{FGcount}{10}
F-Gl-\arabic{FGcount} & Signature/Vérification de données & SmartCard, SoftCard, Utilisateur & \cellcolor{green!50}Indispensable\\
\hline
\addtocounter{FGcount}{10}
F-Gl-\arabic{FGcount} & Administration des cartes & Administrateur & \cellcolor{blue!50}Secondaire\\
\hline
\end{tabularx}

\subsection{Communications sécurisées}
La communication sécurisée est une précondition pour tous les prochains cas 
d'utilisation. Les composants établissent des secrets cryptographiques pour 
des tunnels sécurisés après s’être mutuellement authentifiés.


\subsection{Génération de nombres aléatoire}
SoftCard doit être capable d'utiliser le générateur de nombres aléatoire de
SmartCard.\\

\fiche
	{Génération de nombres aléatoire}
	{SmartCard, SoftCard}
    {SmartCard génère un nombre aléatoire de la taille demandée}
	{}
    {Demande de SoftCard}
    {SmartCard renvoie un nombre aléatoire à SoftCard}
	{\begin{itemize}
	    \item[]
	  \item[1.] SoftCard demande à SmartCard un nombre aléatoire de longueur 
          donnée.
      \item[3.] SoftCard récupère le nombre.
	\end{itemize}
	}
	{\begin{itemize}
        \item[]
        \item[2.] SmartCard génère un nombre grâce au RNG intégré et le renvoie
            à SoftCard.
	\end{itemize}
    }
	{}
\flots
    {}
    {}
    {}
\\*

\subsection{Déblocage de la carte}
Pour utiliser une SmartCard, l'utilisateur devra entrer son code PIN afin de 
\og{}débloquer \fg{} celle-ci.\\

\fiche
    {Authentification d'un utilisateur/Déblocage de la carte}
	{Utilisateur, SmartCard, SoftCard}
	{Le système authentifie l'utilisateur grâce au code PIN contenu sur SmartCard.}
	{L'utilisateur a une carte valide et connaît le code PIN, SoftCard et SmartCard sont authentifiés.}
    {SoftCard a besoin d'utiliser SmartCard.}
    {L'authentification a réussi ou a échoué.}
    {\begin{itemize}
        \item[]
        \item[1.] L'utilisateur insère la carte et demande à SoftCard 
            d'utiliser SmartCard
        \item[3.] L'utilisateur tape le code PIN
    \end{itemize}
    }
	{\begin{itemize}
        \item[]
		\item[2.] SoftCard demande le code PIN à l'utilisateur
        \item[4.] SoftCard envoie le code PIN à SmartCard
        \item[5.] SmartCard vérifie le code PIN et passe en état 
            \og{}débloqué \fg{} pendant 30 minutes s'il est correct
        \item[6.] SoftCard informe l’utilisateur du résultat
	\end{itemize}
	}
	{}
\flots
{\begin{itemize}
        \item[]
        \item[7.] SoftCard redemande le code PIN à l'utilisateur.
\end{itemize}}
    {\begin{itemize}
    \item[]
    \item[7 bis.] L'utilisateur a tapé 3 mauvais codes. La carte se verrouille.
    \end{itemize}
    }
	{}    
\\*

Si l'utilisateur a entré plusieurs mauvais codes PIN, la carte se bloque et il 
peut la débloquer avec le code PUK.\\

\fiche
{Déblocage de la carte par code PUK}
	{Utilisateur, SmartCard, SoftCard}
    {L'utilisateur débloque sa carte avec le code PUK}
	{L'utilisateur a une carte valide mais bloquée, il connaît le code PUK, 
    SoftCard et SmartCard sont authentifiés}
    {SoftCard a besoin d'utiliser SmartCard}
    {La carte est déverrouillée ou inutilisable}
	{\begin{itemize}
	    \item[]
	  \item[1.] L'utilisateur insère la carte et demande à SoftCard 
          d'utiliser SmartCard
      \item[3.] L'utilisateur tape le code PUK
	\end{itemize}
	}
	{\begin{itemize}
        \item[]
		\item[2.] SoftCard indique que le code PIN est verrouillé et que la 
            carte doit être débloquée par le code PUK
		\item[4.] SoftCard envoie le code PUK à SmartCard
		\item[5.] SmartCard vérifie le code PUK, génère un nouveau code PIN 
            aléatoire et le renvoie à SoftCard.
        \item[6.] SoftCard informe l’utilisateur de son nouveau code PIN
	\end{itemize}
	}
	{}
\flots
{\begin{itemize}
        \item[]
        \item[7.] SoftCard redemande le code PUK à l'utilisateur.
\end{itemize}}
    {\begin{itemize}
    \item[]
    \item[7 bis.] L'utilisateur a tapé 3 mauvais codes PUK. La carte se verrouille
        définitivement et doit être remplacée.
    \end{itemize}
    }
	{}    
\\*

\subsection{Transmission de données}
SmartCard contient des données propres à l'utilisateur, elle doit alors 
permettre la transmission de ces  données. Ici, c'est FaceCrypt qui souhaite 
récupérer les données de l'utilisateur.\\

\fiche
{Transmission login/mot de passe au SocialNetwork}
	{SmartCard, SoftCard, FaceCrypt, Social Network}
    {SmartCard transmet le couple login/mdp à FaceCrypt}
	{SmartCard est débloquée avec le bon code PIN, SoftCard et FaceCrypt sont 
    authentifiés. SoftCard et SmartCard sont authentifiés}
    {L'utilisateur veut se connecter sur SocialNetwork}
    {L'utilisateur est connecté auprès de SocialNetwork.}
    {\begin{itemize}
        \item[]
        \item[1.] FaceCrypt demande à SoftCard le login/mdp du Social Network
        \item[5.] FaceCrypt envoie au Social Network le login/mdp.
    \end{itemize}
    }
	{\begin{itemize}
        \item[]
		\item[2.] SoftCard demande à SmartCard le login/mdp du Social Network.
		\item[3.] SmartCard envoie le login/mdp à SoftCard
        \item[4.] SoftCard envoie à FaceCrypt le login/mdp du Social Network.
	\end{itemize}
	}
	{}
\flots
    {}
    {\begin{itemize}
    \item[]
    \item[1.] Authentification invalide
    \end{itemize}
    }
	{}    
\\*

\subsection{Chiffrement/Déchiffrement}
La carte procède au chiffrement et au déchiffrement de la clef de chiffrement 
symétrique pour chaque message avec les clés asymétriques adéquates.

Pour le chiffrement, SoftCard utilise la clef publique du destinataire. 
Pour le déchiffrement, SmartCard utilise la clef privée stockée en mémoire sur
la carte.

Cet exemple concerne FaceCrypt mais est aisément adaptable à tout autre système.

\fiche
{Déchiffrement de données}
	{SmartCard, SoftCard, FaceCrypt}
    {SmartCard déchiffre des données, envoyées par FaceCrypt avec la clé privée
    de chiffrement stockée sur la carte.}
    {SmartCard est débloquée avec le bon code PIN, Authentification entre 
    SoftCard et FaceCrypt.  SoftCard et SmartCard sont authentifiés}
    {Demande de FaceCrypt}
    {SmartCard renvoie un résultat du déchiffrement à SoftCard qui transmet à 
    FaceCrypt}
    {\begin{itemize}
        \item[]        
        \item[1.] FaceCrypt envoie des données chiffrées à SoftCard
        \item[5.] FaceCrypt récupère les données déchiffrées
    \end{itemize}}
	{\begin{itemize}
        \item[]		
        \item[2.] SoftCard transmet les données chiffrées à SmartCard
	\item[3.] SmartCard déchiffre les données avec la clé privée de 
        chiffrement stockée et renvoie le résultat du déchiffrement à SoftCard
	\item[4.] SoftCard transmet le résultat à FaceCrypt.
	\end{itemize}
	}
	{}
\flots
    {}
    {Une erreur de déchiffrement provoque l'envoi d'un message d'erreur, ne
    permettant pas de connaître sa cause}
\\*

La demande de chiffrement de données vient de l'utilisateur ou de toute autre 
application tierce, dans le cas d'une utilisation plus générale.\\

\fiche
{Chiffrement de données}
	{Utilisateur, SoftCard}
    {SoftCard chiffre les donnés avec la clef publique du destinataire}
    {}
    {Demande de l'utilisateur}
    {SoftCard renvoie les données chiffrées}
    {\begin{itemize}
        \item[]
        \item[1.] L'utilisateur transmet des données à chiffrer à SoftCard,
            ainsi que la clef publique du destinataire;
    \end{itemize}}
	{\begin{itemize}
        \item[]
        \item[2.] SoftCard reçoit les données en clair et les chiffre 
        avec la clef publique du destinataire;
		\item[3.] SoftCard renvoie les données chiffrées à l'utilisateur.
	\end{itemize}
	}
	{}
\flots
    {}
    {}
\\*

\subsection{Signature/Vérification de données}

\fiche
{Signature de données}
	{SoftCard, SmartCard}
    {SmartCard utilise la clef privée pour signer des données fournies par 
    SoftCard}
    { SmartCard et SoftCard sont mutuellement authentifiées et l'utilisateur a
    déverrouillé la carte}
    {Demande de l'utilisateur}
    {Les données sont signées ou l'utilisateur annule}
    {\begin{itemize}
        \item[]
        \item[1.] SoftCard envoie les données qu'il faut signer à SmartCard.
        \item[3.] SoftCard récupère la signature.
    \end{itemize}}
	{\begin{itemize}
        \item[]
        \item[2.] SmartCard signe les données et envoie la signature a SoftCard 
	\end{itemize}
	}
	{}
\flots
    {}
    {}
\\*

\fiche
{Vérification de données}
	{SoftCard}
    {SoftCard utilise la clef publique pour vérifier des données signées}
    {}
    {Demande de l'utilisateur}
    {Les données sont vérifiées, ou non}
    {\begin{itemize}
        \item[]
        \item[2.] SoftCard vérifie les données et renvoie "vrai" si la signature
            et son équivalent en clair correspondent; faux sinon.
    \end{itemize}}
	{\begin{itemize}
        \item[]
	\end{itemize}
	}
	{}
\flots
    {}
    {}
\\*


\subsection{Administration des cartes}

\fiche
{Initialisation d'une carte}
	{Utilisateur, SmartCard, SoftCard, FaceCrypt}
    {L'utilisateur paramètre la carte avec ses identifiants}
    {SmartCard et SoftCard sont mutuellement authentifiées et l'utilisateur a
    déverrouillé la carte}
    {Demande de l'utilisateur}
    {Les applets ont été installées sur la carte. Les clefs ont été 
        initialisées, l'identifiant et le mot de passe ont été stockés}
    {\begin{itemize}
        \item[]
        \item[1.] L'utilisateur entre ses identifiants Facebook et choisit ou 
        non de générer un nouveau mot de passe. 
        \item[4.] FaceCrypt enregistre le nouveau mot de passe et envoie la
        confirmation de modification à SoftCard.
    \end{itemize}}
	{\begin{itemize}
        \item[]
        \item[2.] Un utilitaire installe les applets et les clefs de chiffrement
        et de signature sur la carte, génère un nouveau mot de passe si besoin,
        stocke temporairement les identifiants Facebook (anciens et nouveaux)
        et crée un code PIN, qu'il transmet à l'utilisateur.
        \item[3.] SoftCard envoie le nouveau mot de passe à FaceCrypt.
        \item[5.] SoftCard transmet la confirmation à SmartCard qui stocke
            définitivement le mot de passe et efface l'ancien.
	\end{itemize}
	}
	{}
\flots
    {}
    {Si le compte n'existe pas, une erreur est générée.}
\\*

\fiche
{Modification du mot de passe}
	{Utilisateur, SmartCard, SoftCard, FaceCrypt}
    {L'utilisateur souhaite changer son mot de passe Facebook}
    {SmartCard et SoftCard sont mutuellement authentifiées et l'utilisateur a
    déverrouillé la carte}
    {Demande de l'utilisateur}
    {Le mot de passe a été modifié et enregistré sur Facebook et sur la carte.}
    {\begin{itemize}
        \item[]
        \item[1.] L'utilisateur choisit l'option permettant de générer un 
            nouveau mot de passe.
        \item[3.] FaceCrypt enregistre la modification et envoie une confirmation
            du changement de mot de passe à SoftCard.
    \end{itemize}}
	{\begin{itemize}
        \item[]
        \item[2.] SoftCard transmet le mot de passe à SmartCard qui le stocke 
            temporairement. Il notifie ensuite FaceCrypt de la modification.
        \item[4.] SoftCard transmet la confirmation à SmartCard qui stocke 
            définitivement le mot de passe et efface l'ancien.
	\end{itemize}
	}
	{}
\flots
    {}
    {Si le compte n'existe pas, une erreur est générée.}
\\*


\section{Exigences opérationnelles}

\newcounter{FOcount}

\begin{longtable}{|p{2cm}|p{10cm}|p{2.5cm}|}

% Header for the first page of the table
\multicolumn{1}{|>{\color{white}\columncolor{blue}}l|}{\bfseries{Référence}} & \multicolumn{1}{|>{\color{white}\columncolor{blue}}l|}{\bfseries{Fonctionalité}} & \multicolumn{1}{|>{\color{white}\columncolor{blue}}l|}{\bfseries{Priorité}}
\endfirsthead
% Header for next pages of the table
\multicolumn{1}{|>{\color{white}\columncolor{blue}}l|}{\bfseries{Référence}} & \multicolumn{1}{|>{\color{white}\columncolor{blue}}l|}{\bfseries{Fonctionalité}} & \multicolumn{1}{|>{\color{white}\columncolor{blue}}l|}{\bfseries{Priorité}}
\endhead

%This is the footer for all pages except the last page of the table...
\endfoot
%This is the footer for the last page of the table...
\endlastfoot

\hline
\addtocounter{FOcount}{10}
F-FO-\arabic{FOcount} & Le système fonctionne & \cellcolor{green!50}Indispensable \\
\hline
\addtocounter{FOcount}{10}
F-FO-\arabic{FOcount} & La caractéristique aléatoire d'un nombre généré 
(par le générateur aléatoire) est vérifiable & \cellcolor{green!50}Indispensable \\
\hline
\end{longtable}

\section{Exigences d'interface}

\newcounter{FIcount}

\begin{longtable}{|p{2cm}|p{10cm}|p{2.5cm}|}

% Header for the first page of the table
\multicolumn{1}{|>{\color{white}\columncolor{blue}}l|}{\bfseries{Référence}} & \multicolumn{1}{|>{\color{white}\columncolor{blue}}l|}{\bfseries{Fonctionalité}} & \multicolumn{1}{|>{\color{white}\columncolor{blue}}l|}{\bfseries{Priorité}}
\endfirsthead
% Header for next pages of the table
\multicolumn{1}{|>{\color{white}\columncolor{blue}}l|}{\bfseries{Référence}} & \multicolumn{1}{|>{\color{white}\columncolor{blue}}l|}{\bfseries{Fonctionalité}} & \multicolumn{1}{|>{\color{white}\columncolor{blue}}l|}{\bfseries{Priorité}}
\endhead

%This is the footer for all pages except the last page of the table...
\endfoot
%This is the footer for the last page of the table...
\endlastfoot

\hline
\addtocounter{FIcount}{10}
F-FI-\arabic{FIcount} & SoftCard communique de manière sécurisée avec FaceCrypt
& \cellcolor{green!50}Indispensable \\
\hline
\addtocounter{FIcount}{10}
F-FI-\arabic{FIcount} & SoftCard présente une interface pour demander le code 
PIN & \cellcolor{green!50}Indispensable \\
\hline
\addtocounter{FIcount}{10}
F-FI-\arabic{FIcount} & SoftCard communique de manière sécurisée avec SmartCard
& \cellcolor{green!50}Indispensable \\
\hline
\end{longtable}

\section{Exigences de qualité}

\newcounter{FQcount}

\begin{longtable}{|p{2cm}|p{10cm}|p{2.5cm}|}

% Header for the first page of the table
\multicolumn{1}{|>{\color{white}\columncolor{blue}}l|}{\bfseries{Référence}} & \multicolumn{1}{|>{\color{white}\columncolor{blue}}l|}{\bfseries{Fonctionalité}} & \multicolumn{1}{|>{\color{white}\columncolor{blue}}l|}{\bfseries{Priorité}}
\endfirsthead
% Header for next pages of the table
\multicolumn{1}{|>{\color{white}\columncolor{blue}}l|}{\bfseries{Référence}} & \multicolumn{1}{|>{\color{white}\columncolor{blue}}l|}{\bfseries{Fonctionalité}} & \multicolumn{1}{|>{\color{white}\columncolor{blue}}l|}{\bfseries{Priorité}}
\endhead

%This is the footer for all pages except the last page of the table...
\endfoot
%This is the footer for the last page of the table...
\endlastfoot

\hline
\addtocounter{FQcount}{10}
F-FQ-\arabic{FQcount} & Le système sera livré pour le 01 mars 2013 & \cellcolor{green!50}Indispensable \\
\hline
\addtocounter{FQcount}{10}
F-FQ-\arabic{FQcount} & Une documentation de développement est fournie
& \cellcolor{green!50}Indispensable \\
\hline
\addtocounter{FQcount}{10}
F-FQ-\arabic{FQcount} & Le système est adaptable
& \cellcolor{red!20}Important \\
\hline
\addtocounter{FQcount}{10}
F-FQ-\arabic{FQcount} & L'utilisation d'une fonction cryptographique ne doit pas ralentir le système
& \cellcolor{green!50}Indispensable \\
\hline
\addtocounter{FQcount}{10}
F-FQ-\arabic{FQcount} & Authentification mutelle entre toutes les entités (SmartCard, SoftCard et FaceCrypt)
& \cellcolor{green!50}Indispensable \\
\hline
\end{longtable}

\section{Exigences de réalisation}

\newcounter{FRcount}

\begin{longtable}{|p{2cm}|p{10cm}|p{2.5cm}|}

% Header for the first page of the table
\multicolumn{1}{|>{\color{white}\columncolor{blue}}l|}{\bfseries{Référence}} & \multicolumn{1}{|>{\color{white}\columncolor{blue}}l|}{\bfseries{Fonctionalité}} & \multicolumn{1}{|>{\color{white}\columncolor{blue}}l|}{\bfseries{Priorité}}
\endfirsthead
% Header for next pages of the table
\multicolumn{1}{|>{\color{white}\columncolor{blue}}l|}{\bfseries{Référence}} & \multicolumn{1}{|>{\color{white}\columncolor{blue}}l|}{\bfseries{Fonctionalité}} & \multicolumn{1}{|>{\color{white}\columncolor{blue}}l|}{\bfseries{Priorité}}
\endhead

%This is the footer for all pages except the last page of the table...
\endfoot
%This is the footer for the last page of the table...
\endlastfoot

\hline
\addtocounter{FRcount}{10}
F-FR-\arabic{FRcount} & Un SDK et un manuel sont fournis & \cellcolor{green!50}Indispensable \\
\hline
\end{longtable}


\end{document}
