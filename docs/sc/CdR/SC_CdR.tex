
%!TEX encoding = UTF-8 Unicode

\documentclass[a4paper,11pt,french]{article}

%Import des packages utilisés pour le document
\usepackage[french]{babel}
\usepackage{chngpage}
\usepackage[colorlinks=true,linkcolor=black,urlcolor=blue]{hyperref}
\usepackage{graphicx, amssymb, color, listings}
\usepackage{fontspec,xltxtra,xunicode,color}
\usepackage{tabularx, longtable}
\usepackage[table]{xcolor}
\usepackage{fancyhdr}
\usepackage{tikz}
\usetikzlibrary{shapes}
\usepackage{lastpage}

\definecolor{gris}{rgb}{0.95, 0.95, 0.95}

%Redéfinition des marges
\addtolength{\hoffset}{-2cm}
\addtolength{\textwidth}{4cm}
\addtolength{\topmargin}{-2cm}
\addtolength{\textheight}{1cm}
\addtolength{\headsep}{0.8cm}
\addtolength{\footskip}{1cm}


%Import page de garde et structures pour la gestion de projet
\usepackage{res/structures}

%Variables
\def\matiere{Conduite de Projet}
\def\filiere{Master 2 SSI}
\def\projectDesc{Smart Social Network}
\def\projectName{\emph{SSN}~}
\def\completeName{\projectDesc ~- \projectName}
\def\docType{Cahier de recettes}
\def\docDate{\today}
\def\version{0.1}
\def\author{Giovanni \textsc{Huet}, Romain \textsc{Pignard}}
\def\checked{Baptiste \textsc{Dolbeau}}
\def\approved{}


% -- Début du document -- %
\begin{document}
%Page de garde
\makeFirstPage
\clearpage

%Tableau de mises à jour
\vspace*{1cm}
\begin{center}
\textbf{\huge{MISES À JOUR}}\\
\vspace*{3cm}
\begin{tabularx}{16cm}{|c|c|X|}
\hline
\bfseries{Version} & \bfseries{Date} & \bfseries{Modifications réalisées}\\
\hline
0.1 & 05/02/2013 & Création\\
\hline
0.2 & 24/02/2013 & Vérification\\
\hline
\end{tabularx}
\end{center}

%La table des matières
\clearpage
\tableofcontents
\clearpage

\section{Introduction}

Ce document est un support pour la validation du logiciel lors de la recette
auprès du client. Il est consacré à la définition des moyens et des
procédures utilisés pour assurer la recette du produit développé. La recette
est un procédé permettant d’assurer la conformité du logiciel à la
spécification déjà définie. Nous allons recenser dans ce document les
objectifs de tests de validation et les moyens nécessaires pour les atteindre
en précisant :
\begin{itemize}
	\item Les pré-conditions à satisfaire;
	\item Les moyens matériels requis (plate-forme de tests);
	\item La logique de leur déroulement (étapes successives).
\end{itemize}
Notre logiciel peut être divisé en une liste de constituants qui seront
testés à tour de rôle. L'ensemble des opérations devra être transparent
vis à vis de l'utilisateur. Les différents cas d’utilisation prélevés
de la spécification technique de besoin sont les suivants :

\subsection*{Génération de nombres aléatoires}
La carte à puce doit pouvoir générer des nombres aléatoires de manière
sécurisée : c'est à dire non prévisible.

\subsection*{Déblocage de la carte (via authentification par code PIN ou via PUK)}
Afin d'utiliser la carte, il nous faut nous authentifier à l'aide d'un code PIN.
Si l'utilisateur échoue à l'authentification par code PIN suite à un certain
nombre de tentatives, la carte sera verrouillée. Pour la déverrouiller nous
utiliserons le code PUK.

\subsection*{Transmission de données}
La carte doit pouvoir transmettre des données stockées à SoftCard telles que
le login, le mot de passe, la clef publique, etc...

\subsection*{Chiffrement/Déchiffrement de données}
Sur la carte sont stockées la clef publique et la clef privée préalablement
générées (Crypto-système assymétrique de type RSA). Ces clefs nous
permettront de chiffrer et de déchiffrer des données reçues ou à envoyer.

\subsection*{Signature/Vérification de données}
Par le biais de la carte, nous serons en mesure de signer des données avec
notre clef privée afin d'assurer la non répudiation. Nous devons également
pouvoir vérifier l'auteur des données. Nous utiliserons pour cela la clef
publique. 

\subsection*{Administration des cartes}
On devra également pouvoir administrer les cartes : effectuer des opérations
telles que la réinitialisation du code PIN, son attribution, etc...


\section{Documents applicables et de références}
\begin{itemize}
	\item SC\_STB : Le document renfermant les spécifications techniques de Besoin;
	\item SC\_DaL : Le document contenant l'architecture logicielle;
	\item Les comptes rendu de réunion du projet;
	\item Le sujet du projet : "cartes-a-puce.pdf".
\end{itemize}

\section{Terminologie et sigles utilisés}
\begin{description}
    \item[CdR :] Cahier de Recettes;
    \item[AdR :] Analyse des Risques;
    \item[DAL :] Document d'Architecture Logicielle;
    \item[PdD :] Plan de développement;
    \item[STB :] Spécification Technique de Besoin;
    \item[SC :] \emph{SmartCard}, relatif au sous-projet sur les cartes à puce;
    \item[SSN :] \emph{Secure Social Network}, relatif au sous-projet sur les
	réseaux sociaux;
    \item[FaceCrypt :] Application Java gérant les traitements lourds
	(chiffrement/déchiffrement) de l'extension et relais entre l'Extension
	et \emph{SmartCard};
    \item[IHM :] Interface Homme-Machine, (interface graphique);
    \item[Utilisateur :] Entité (humain ou programme) intéragissant avec ce
        sous-projet;
    \item[Système :] Ce sous-projet;
    \item[SoftCard :] Application effectuant le relais entre la carte et
	FaceCrypt;
    \item[Extension :] Programme incorporé dans le navigateur;
    \item[Aléatoire :] Indistinguable en temps polynomial, distribution de
	probabilité uniforme;
    \item[PRNG :] \emph{Pseudo Random Number Generator} - Générateur de
	nombres pseudo-aléatoires;
    \item[PIN :] \emph{Personal Identification Number} - Code servant à
	authentifier l'utilisateur;
    \item[PUK :] \emph{Personal Unlock Key} - Code servant à débloquer la
	carte quand trop de codes PIN erronés ont été entrés.
\end{description}

\section{Environnement de tests}
L'ensemble des tests se sont effectués sur des machines ayant les
caractéristiques suivantes :
\begin{itemize} 
	\item Système d'exploitation : Ubuntu 12.04;
	\item Processeur : Intel(R) Core(TM)2 Duo CPU E8400  @ 3.00GHz;
	\item Mémoire : 2Go RAM;
	\item Logiciel : Eclipse Platform Version: 3.8.0, Java 1.6 (Client) et Java 1.5 (Java card).
\end{itemize}


Nous utilisons également des cartes Java Card J3A (marque NXP) avec 40K
d'EEPROM et des lecteurs Omnikey 3121. Les cartes sont conformes aux standards
Java Card 2.2.2 et Global Platform 2.1.1.

\section{Responsabilité}
Afin de mener les tests dans les meilleurs conditions, une organisation au sein
du groupe a été mise en place :\\

La conception et la définition des données de tests a été réalisée par
Giovanni \textsc{Huet} et Romain \textsc{Pignard}. Après avoir exécuté
les différents tests, les responsables de ce module transmettront aux
développeurs un compte rendu contenant les résultats afin d’améliorer la
version actuelle du logiciel et d'en fournir une nouvelle à tester. Chaque
version fournie doit être testée et validée.

\section{Stratégie de tests}

La démarche utilisée pour effectuer les tests est la suivante :
\begin{itemize}
\item Mettre à la disposition de l’équipe testeur les modules développés.
\item Réalisation des tests à travers une procédure, celle ci comportera
un jeu de tests ainsi que la modalité de leur exécution.
\item Élaboration d'un compte rendu des résultats des tests qui sera
transmis aux développeurs.
\item Correction des anomalies par l'équipe développeur.
\item Des tests secondaires seront effectués pour s'assurer que toutes les
anomalies ont été corrigées.
\end{itemize}

Les tests seront réalisés par ordre de priorité. Les modules ayant une
priorité indispensable seront pris en compte dès que possible. La condition
d'arrêt des tests sera le succès de ces derniers après correction des
anomalies.

\section{Gestion des anomalies}

A chaque modification apportée (correction), nous devrons réaliser un nombre
de tests permettant de détecter les anomalies persistantes. Toute anomalie
détectée sera notée dans un rapport et ce dernier sera envoyé aux
développeurs afin qu'ils apportent les modifications nécessaires.

\section{Procédures de tests}
Pour chaque cas d’utilisation, nous décrivons une procédure de test
détaillée. Chaque procédure dispose d'un jeu de tests basé sur des
données réelles.

\newpage

\begin{figure}[!h]
\begin{tabular}{|p{1cm}|p{5cm}|p{5cm}|p{2cm}|p{2cm}|}
\hline
\multicolumn{3}{|l|}{Objet testé : Génération de nombres aléatoires} & \multicolumn{2}{|l|}{Version: 1.0} \\
\hline
\multicolumn{5}{|l|}{Objectif de test : Vérifier le comportement du générateur aléatoire de la carte à puce.} \\
\hline
\multicolumn{5}{|l|}{Procédure n°1 : Générer un nombre aléatoire.} \\
\hline
N° & Action(s) & Résultats attendus & Exigence & OK/NOK \\
\hline
1 & Générer un nombre aléatoire à l'aide des fonctions javacard disponibles. & Obtention d'un nombre aléatoire. & F-Gl-10 & OK \\
\hline
\end{tabular}
\end{figure}


\begin{figure}[!h]
\begin{tabular}{|p{1cm}|p{5cm}|p{5cm}|p{2cm}|p{2cm}|}
\hline
\multicolumn{5}{|l|}{Procédure n°2 : Evaluer le niveau de l'aléatoire.} \\
\hline
N° & Action(s) & Résultats attendus & Exigence & OK/NOK \\
\hline
1 & Générer plusieurs (millions) nombres aléatoires afin d'établir des stastitiques et vérifier le niveau de l'aléatoire. &
Générateur non prévisible (Probabilité uniforme) & F-Gl-10 & NC \\
\hline
\end{tabular}
\end{figure}



\begin{figure}[!h]
\begin{tabular}{|p{1cm}|p{5cm}|p{5cm}|p{2cm}|p{2cm}|}
\hline
\multicolumn{5}{|l|}{Procédure n°3 : Evaluer le temps d'exécution.} \\
\hline
N° & Action(s) & Résultats attendus & Exigence & OK/NOK \\
\hline
1 & Chronométrer la génération de plusieurs (milliers) nombres aléatoires pour en évaluer la moyenne. & Nous souhaitons que le temps de génération d'un nombre aléatoire soit < 300ms pour que cela soit invisible à l'utilisateur. & F-Gl-10 & OK \\
\hline
\end{tabular}
\end{figure}


\newpage

\begin{figure}[!h]
\begin{tabular}{|p{1cm}|p{5cm}|p{5cm}|p{2cm}|p{2cm}|}
\hline
\multicolumn{3}{|l|}{\begin{tabular}{l}Objet testé : Déblocage de la carte\\ (via authentification par code PIN et PUK)\end{tabular}} & \multicolumn{2}{|l|}{Version: 1.0} \\
\hline
\multicolumn{5}{|l|}{Objectif de test : Vérifier le comportement de la carte lors de plusieurs tentatives d'authentifications.} \\
\hline
\multicolumn{5}{|l|}{Procédure n°4 : Déverrouillage de la carte (via code PIN).} \\
\hline
N° & Action(s) & Résultats attendus & Exigence & OK/NOK \\
\hline
1 & Entrer le code PIN valide. & Authentification de l'utilisateur rendant la carte utilisable. & F-Gl-20 & OK \\
\hline
\end{tabular}
\end{figure}


\begin{figure}[!h]
\begin{tabular}{|p{1cm}|p{5cm}|p{5cm}|p{2cm}|p{2cm}|}
\hline
\multicolumn{5}{|l|}{Procédure n°5 : Verrouillage de la carte.} \\
\hline
N° & Action(s) & Résultats attendus & Exigence & OK/NOK \\
\hline
1 & Effectuer trois authentifications erronées à la suite. & Verrouillage de la carte. & F-Gl-20 & OK \\
\hline
\end{tabular}
\end{figure}


\begin{figure}[!h]
\begin{tabular}{|p{1cm}|p{5cm}|p{5cm}|p{2cm}|p{2cm}|}
\hline
\multicolumn{5}{|l|}{Procédure n°6 : Déverrouillage de la carte (via code PUK).} \\
\hline
N° & Action(s) & Résultats attendus & Exigence & OK/NOK \\
\hline
1 & Entrer le code PUK valide. & La carte est de nouveau opérationnelle. & F-Gl-20 & OK \\
\hline
\end{tabular}
\end{figure}


\newpage

\begin{figure}[!h]
\begin{tabular}{|p{1cm}|p{5cm}|p{5cm}|p{2cm}|p{2cm}|}
\hline
\multicolumn{3}{|l|}{Objet testé : Transmission de données} & \multicolumn{2}{|l|}{Version: 1.0} \\
\hline
\multicolumn{5}{|l|}{Objectif de test : Vérifier si les données contenues dans la carte peuvent être transmises.} \\
\hline
\multicolumn{5}{|l|}{Procédure n°7 : Transmettre des données.} \\
\hline
N° & Action(s) & Résultats attendus & Exigence & OK/NOK \\
\hline
1 & Envoyer des données de la carte à SoftCard. & Réception intégrale des données par SoftCard. & F-Gl-30 & OK \\
\hline
\end{tabular}
\end{figure}


\newpage

\begin{figure}[!h]
\begin{tabular}{|p{1cm}|p{5cm}|p{5cm}|p{2cm}|p{2cm}|}
\hline
\multicolumn{3}{|l|}{Objet testé : Communication sécurisée lecteur/carte (Tunnel)} & \multicolumn{2}{|l|}{Version: 1.0} \\
\hline
\multicolumn{5}{|l|}{Objectif de test : Vérifier les fonctions de chiffrement, d'authentification et d'intégrité.} \\
\hline
\multicolumn{5}{|l|}{Procédure n°8 : Envoi de données chiffrées.} \\
\hline
N° & Action(s) & Résultats attendus & Exigence & OK/NOK \\
\hline
1 & Envoyer des données chiffrées à l'aide d'un chiffrement sysmétrique (AES-128) & Données non compréhensibles sans la clef de déchiffrement. & F-FI-30 & OK \\
\hline
\end{tabular}
\end{figure}



\begin{figure}[!h]
\begin{tabular}{|p{1cm}|p{5cm}|p{5cm}|p{2cm}|p{2cm}|}
\hline
\multicolumn{5}{|l|}{Procédure n°9 : Envoi de données authentifiées sans altération.} \\
\hline
N° & Action(s) & Résultats attendus & Exigence & OK/NOK \\
\hline
1 & Envoyer des données authentifiées à l'aide de CBC-MAC sans altérer le contenu. & Validation de la non modification des données reçues. & F-FI-30 & OK \\
\hline
\end{tabular}
\end{figure}


\begin{figure}[!h]
\begin{tabular}{|p{1cm}|p{5cm}|p{5cm}|p{2cm}|p{2cm}|}
\hline
\multicolumn{5}{|l|}{Procédure n°10 : Envoi de données authentifiées avec altération.} \\
\hline
N° & Action(s) & Résultats attendus & Exigence & OK/NOK \\
\hline
1 & Envoyer des données authentifiées à l'aide de CBC-MAC en altérant leur contenu. & Détection de la modification des données reçues. & F-FI-30
 & OK \\
\hline
\end{tabular}
\end{figure}


\begin{figure}[!h]
\begin{tabular}{|p{1cm}|p{5cm}|p{5cm}|p{2cm}|p{2cm}|}
\hline
\multicolumn{5}{|l|}{Procédure n°11 : Déchiffrement de données reçues.} \\
\hline
N° & Action(s) & Résultats attendus & Exigence & OK/NOK \\
\hline
1 & Déchiffrement des données issues d'un cryptosystème sysmétrique à l'aide de la clef secrète partagée. & Récupération en clair des données transitant dans le tunnel. & F-FI-30 & OK \\
\hline
\end{tabular}
\end{figure}


\newpage

\begin{figure}[!h]
\begin{tabular}{|p{1cm}|p{5cm}|p{5cm}|p{2cm}|p{2cm}|}
\hline
\multicolumn{3}{|l|}{\begin{tabular}{l}Objet testé : Déchiffrement de données issues\\ d'un cryptosystème asymétrique (RSA)\end{tabular}} & \multicolumn{2}{|l|}{Version: 1.0} \\
\hline
\multicolumn{5}{|l|}{Objectif de test : Vérifier si les données reçues peuvent être déchiffrées.} \\
\hline
\multicolumn{5}{|l|}{Procédure n°12 : Déchiffrer avec la clef privée.} \\
\hline
N° & Action(s) & Résultats attendus & Exigence & OK/NOK \\
\hline
1 & Déchiffrer des données à partir de la carte avec la clef privée. & Données déchiffrées. & F-Gl-40 & OK \\
\hline
\end{tabular}
\end{figure}



\begin{figure}[!h]
\begin{tabular}{|p{1cm}|p{5cm}|p{5cm}|p{2cm}|p{2cm}|}
\hline
\multicolumn{5}{|l|}{Procédure n°13 : Déchiffrer avec une clef invalide.} \\
\hline
N° & Action(s) & Résultats attendus & Exigence & OK/NOK \\
\hline
1 & Déchiffrer des données à partir de la carte avec une clef privée invalide. & Données non déchiffrées et une erreur est détectée. & F-Gl-40 & OK \\
\hline
\end{tabular}
\end{figure}


\newpage

\begin{figure}[!h]
\begin{tabular}{|p{1cm}|p{5cm}|p{5cm}|p{2cm}|p{2cm}|}
\hline
\multicolumn{3}{|l|}{Objet testé : Signature/Vérification de données.} & \multicolumn{2}{|l|}{Version: 1.0} \\
\hline
\multicolumn{5}{|l|}{Objectif de test : Signer des données et vérifier la signature.} \\
\hline
\multicolumn{5}{|l|}{Procédure n°14 : Signer des données.} \\
\hline
N° & Action(s) & Résultats attendus & Exigence & OK/NOK \\
\hline
1 & Signer des données à partir de la carte avec la clef privée. & Données signées. & F-Gl-50 & OK \\
\hline
\end{tabular}
\end{figure}


\begin{figure}[!h]
\begin{tabular}{|p{1cm}|p{5cm}|p{5cm}|p{2cm}|p{2cm}|}
\hline
\multicolumn{5}{|l|}{Procédure n°15 : Vérification avec la clef publique associée.} \\
\hline
N° & Action(s) & Résultats attendus & Exigence & OK/NOK \\
\hline
1 & Vérifier des données signées à partir de la carte avec la clef publique. & Données vérifiées. & F-Gl-50 & OK \\
\hline
\end{tabular}
\end{figure}


\begin{figure}[!h]
\begin{tabular}{|p{1cm}|p{5cm}|p{5cm}|p{2cm}|p{2cm}|}
\hline
\multicolumn{5}{|l|}{Procédure n°16 : Vérification des données invalides.} \\
\hline
N° & Action(s) & Résultats attendus & Exigence & OK/NOK \\
\hline
1 & Vérifier des données signées d'une autre personne à partir de notre carte avec notre clef publique. & Données non vérifiées. & F-Gl-50 & OK \\
\hline
\end{tabular}
\end{figure}


\newpage

\begin{figure}[!h]
\begin{tabular}{|p{1cm}|p{5cm}|p{5cm}|p{2cm}|p{2cm}|}
\hline
\multicolumn{3}{|l|}{Objet testé : Administration des cartes.} & \multicolumn{2}{|l|}{Version: 1.0} \\
\hline
\multicolumn{5}{|l|}{Objectif de test : Vérifier si l'administration des cartes est possible.} \\
\hline
\multicolumn{5}{|l|}{Procédure n°17 : Attribuer un code PIN.} \\
\hline
N° & Action(s) & Résultats attendus & Exigence & OK/NOK \\
\hline
1 & Attribution d'un code PIN à un utilisateur. & L'utilisateur possède un code PIN qui lui est propre. & F-Gl-60 & OK \\
\hline
\end{tabular}
\end{figure}


\begin{figure}[!h]
\begin{tabular}{|p{1cm}|p{5cm}|p{5cm}|p{2cm}|p{2cm}|}
\hline
\multicolumn{5}{|l|}{Procédure n°18 : Insertion de données sur la carte.} \\
\hline
N° & Action(s) & Résultats attendus & Exigence & OK/NOK \\
\hline
1 & Stocker des données qui soient propres à l'utilisateur : login, mot de passe, etc... & La carte contient bien les données. & F-Gl-60 & OK \\
\hline
\end{tabular}
\end{figure}


\paragraph{}
\begin{tabular}{|p{1cm}|p{5cm}|p{5cm}|p{2cm}|p{2cm}|}
\hline
\multicolumn{5}{|l|}{Procédure n°19 : Modification des données sur la carte.} \\
\hline
N° & Action(s) & Résultats attendus & Exigence & OK/NOK \\
\hline
1 & Modification des données préalablement insérées : login, mot de passe, code PIN, etc... & Les données sont bien modifiées. & F-Gl-60 & OK \\
\hline
\end{tabular}


\section{Couverture de tests}
Ce tableau reprend les exigences de la STB et précise, pour chacune d’entre elles, la méthode
de vérification (démonstration / tests) et une description de celle ci.


\begin{figure}[!h]
\begin{tabular}{|p{2cm}|p{2.5cm}|p{2cm}|p{9cm}|}
\hline
Exigence & Méthode de vérification & Procédure utilisée & Commentaire \\
\hline
F-Gl-10 & Démonstration & Procédure 1 & Ce test consiste à utiliser une
fonction disponible via la librairie javacard pour générer un nombre
aléatoire. \\
\hline
F-Gl-10 & Test & Procédure 2 & Le test consiste à générer plusieurs nombres
aléatoires (millions) et les soumettre à un test stastistique pour évaluer
le niveau de l'aléatoire. \\
\hline
F-Gl-10 & Test & Procédure 3 & Le test consiste à chronométrer la génération
de plusieurs nombres aléatoires (milliers) et d'en faire la moyenne afin de
connaitre le temps d'exécution moyen pour la génération d'un nombre
aléatoire, que l'on considérera valide s'il est inférieur à 300 ms. \\
\hline
F-Gl-20 & Test & Procédure 4 & Le test consiste à débloquer la carte en
s'authentifiant auprès de celle ci via le code PIN. \\
\hline
F-Gl-20 & Test & Procédure 5 & Le test consiste à entrer 3 codes PIN
erronés afin de verrouiller la carte, et d'entrer ensuite le bon code
PIN pour vérifier le verrouillage. \\
\hline
F-Gl-20 & Test & Procédure 6 & Le test consiste à déverrouiller la carte
après avoir entré 3 codes PIN erronés en entrant le code PUK valide. \\
\hline
F-Gl-30 & Test & Procédure 7 & Le test consiste à envoyer des données
contenues sur la carte. \\
\hline
F-FI-30 & Test & Procédure 8 & Le test consiste à envoyer des données chiffrées
via une communication sécurisée par un chiffrement symétrique. \\
\hline
F-FI-30 & Test & Procédure 9 & Le test consiste à envoyer des données
authentifiées sans altérer leur contenu. Nous devons ensuite vérifier que
le contenu est bien intègre. \\
\hline 
F-FI-30 & Test & Procédure 10 & Le test consiste à envoyer des données
authentifiées en altérant leur contenu. Nous devons ensuite pouvoir détecter
la modification du contenu. \\
\hline 
F-FI-30 & Test & Procédure 11 & Le test consiste à recevoir des données au
préalable chiffrées, et vérifier si elles ont bien été déchiffrées. \\
\hline
F-Gl-40 & Test & Procédure 12 & Le test consiste à déchiffrer des données
avec la clef privée, nous devrions alors obtenir les données en clair. \\
\hline
F-Gl-40 & Test & Procédure 13 & Le test consiste à déchiffrer des données
avec une clef non valide, nous devrions alors obtenir une erreur. \\
\hline
F-Gl-50 & Test & Procédure 14 & Le test consiste à signer des données à
partir de la clef privée stockée sur la carte. \\
\hline
F-Gl-50 & Test & Procédure 15 & Le test consiste à vérifier avec la clef
publique des données au préalable signées. Les données doivent alors être
vérifiées. \\
\hline
F-Gl-50 & Test & Procédure 16 & Le test consiste à vérifier avec une clef
publique non correspondante des données au préalable signées. Les données
ne doivent alors pas être vérifiées. \\
\hline
F-Gl-60 & Démonstration & Procédure 17 & Dans cette procédure, nous
affectons à un utilisateur, donc à la carte lui appartenant, un code PIN
afin de pouvoir débloquer la carte et l'utiliser. \\
\hline
F-Gl-60 & Démonstration & Procédure 18 & Dans cette procédure, nous
insérons des données dans une carte telles qu'un mot de passe, un login, 
etc... \\
\hline
F-Gl-60 & Test & Procédure 19 & Ce test consiste à modifier les données
précédemment insérées sur la carte. \\
\hline
\end{tabular}
\end{figure}

\clearpage
\end{document}
