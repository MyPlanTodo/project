%!TEX TS-program = xelatex
%!TEX encoding = UTF-8 Unicode

\documentclass[a4paper,11pt,french]{article}

%Import des packages utilisés pour le document
\usepackage{chngpage}
\usepackage[colorlinks=true,linkcolor=black,urlcolor=blue]{hyperref}
\usepackage{graphicx, amssymb, color, listings}
\usepackage{fontspec,xltxtra,xunicode,color}
\usepackage{tabularx, longtable}
\usepackage[table]{xcolor}
\usepackage{fancyhdr}
\usepackage{tikz}
\usepackage[top=2.6cm,bottom=2.6cm,left=2.6cm,right=2.6cm]{geometry}
\usetikzlibrary{shapes}
\usepackage{lastpage}
\usepackage[french]{babel}

\definecolor{gris}{rgb}{0.95, 0.95, 0.95}

%%Redéfinition des marges
%\addtolength{\hoffset}{-2cm}
%\addtolength{\textwidth}{4cm}
%\addtolength{\topmargin}{-2cm}
%\addtolength{\textheight}{1cm}
%\addtolength{\headsep}{0.8cm}
%\addtolength{\footskip}{1cm}
%
%
%Import page de garde et structures pour la gestion de projet
%\usepackage{res/structures}
%
%Variables
%\def\matiere{Conduite de Projet}
%\def\filiere{Master 2 SSI}
%\def\projectDesc{Smart Social Network}
%\def\projectName{\emph{SSN}~}
%\def\completeName{\projectDesc ~- \projectName}
%\def\docType{Rapport}
%\def\docDate{\today}
%
%
\title{Smart Social Network\\Rapport de projet SSI }
%% -- Début du document -- %
\begin{document}
%Page de garde
%\makeFirstPage
%\clearpage
\maketitle

\tableofcontents

\section{Introduction}
\renewcommand\labelitemi{\textbullet} %style des puces
\renewcommand\labelitemii{$\circ$} %style des puces 2e niveau
De nos jours, les réseaux sociaux ont pris une grande ampleur sur internet
et accueillent chaque jour de plus en plus d'adhérents. Le principe consiste à 
y créer un profil, y insérer des données que l'utilisateur désire partagées, 
telles que des photos, des vidéos, des messages, etc. Sur ce réseau, des \og{}amis \fg{}
seront ajoutés : ils pourront alors accéder à ces informations.

\paragraph{}
La problématique soulevée était de limiter la diffusion des informations à 
certaines personnes dans nos \og{}amis \fg{}, mais surtout limiter la diffusion 
d'informations vis à vis du réseau social lui-même. En effet, lorsque nous 
partageons une donnée, celui-ci détient cette information qu'elle soit définie
comme privée ou non.

L'idée de ce projet était alors de limiter cette fuite d'information, afin de 
garantir la confidentialité des données des utilisateurs et ce, renforcé par une
authentification forte. Nous nous concentrerons sur le réseau social Facebook 
puisqu'il est le plus utilisé.


\paragraph{}
La concrétisation du projet s'est traduite par le développement d'une extension 
pour le logiciel Mozilla Firefox permettant à l’utilisateur de gérer le 
chiffrement et le déchiffrement de ses données sur le réseau social Facebook, 
les traitements lourds étant confiés à une application Java.

Concernant l’authentification forte, nous avons utilisé des cartes à puce de 
type Java Card J3A (marque NXP) avec 40 Kilo-octets (Ko) d'EEPROM, via des 
lecteurs Omnikey 3121. 

L'intérêt du projet était également d'analyser la sécurité de ces cartes à puce,
à savoir la génération de nombres aléatoires, de clefs (sysmétriques et 
asymétriques), chiffrement, déchiffrement et signature. C'est cette même carte
à puce qui contiendra à posteriori les données sensibles de l’utilisateur 
comme son identifiant, son mot de passe, sa clef privée... Le dialogue avec 
la carte se fait par l’intermédiaire d’un client Java : \og{}SoftCard \fg{}. 

\paragraph{}
Initialement prévu comme deux projets différents, un par groupe, il s'est avéré 
que nous travaillerions conjointement pour se concentrer sur un unique projet 
regroupant :
\begin{itemize}
    \item l'étude et la mise en \oe{}uvre de solutions d’authentifications et 
        de signatures par cartes à puce, proposé par Magali \textsc{Bardet} ;
    \item les solutions cryptographiques pour les réseaux sociaux, proposé par Ayoub
        \textsc{Otmani} ;
\end{itemize}

\section{Fonctionnement global}
% A compléter par ceux d'SSN
Nous allons dans cette partie décrire, dans les grandes lignes, le 
fonctionnement global du projet.

\subsection{Première utilisation}
Lorsque l'utilisateur installe l'extension et désire se connecter par la suite à Facebook,
une \og pop-up \fg{} sera affichée, lui demandant ainsi ses identifiants. A cet
instant, il pourra décider de générer un nouveau mot de passe -- aléatoire -- par
la carte. Toutes ces informations seront alors stockées sur la carte et il pourra
chiffrer des messages destinés à une liste d'amis qu'il aura remplie. 

\subsection{Chiffrement}
Les chiffrements s'effectuent en plusieurs temps. Tout d'abord, l'utilisateur 
entre son message, clique sur le bouton \og chiffrer \fg{}, sélectionne la ou les
listes d'amis concernées et choisit deux modes : l'un garantit l'anonymat mais
nécessite plus de ressources, l'autre a les propriétés inverses.

%chiffrement à compléter par SSN

\subsection{Déchiffrement}
Pour le déchiffrement, deux cas se présentent. Si la personne ayant chiffré le
message avait choisi le mode \og anonyme \fg{}, un application tentera de 
déchiffrer chacune des lignes composant le contenu chiffré avec la clef privée 
de l'utilisateur, au préalable obtenue en interrogeant la carte. Ces tentatives
s'arrêteront dès qu'aucune erreur n'aura été déclenchée, c'est-à-dire lorsque la
clef privée utilisée correspondra à la clef publique ayant servi à chiffrer la 
clef symétrique. Cette dernière sera alors utilisée pour déchiffrer le message 
posté sur Facebook.

\paragraph{}
Dans le cas où le mode \og non-anonyme \fg{} avait été sélectionné lors du
chiffrement, l'algorithme se contentera de déchiffrer la partie chiffrée
associée à l'identifiant de l'utilisateur courant.


\section{SmartCard}
Aujourd'hui nous sommes tous menés à utiliser les cartes à puce comme les cartes
bancaires, les cartes vitales. Elles sont notamment utilisées pour effectuer de 
l'authentification forte et pour contenir des informations confidentielles.

Dans cette partie du projet, nous détaillerons notre étude des solutions 
cryptographiques pouvant permettre l'authentification ou la signature, puis de 
mettre à profit ces caractéristiques pour la confidentialité liée à Facebook.


\subsection{Génération de nombres aléatoires}
La carte à puce permet de générer des nombres aléatoires, utiles dans la création
d'IV (Initialization Vector), de mots de passe, de clefs, etc. Ce générateur 
peut donc être considéré comme un point crucial dans la sécurité de l'application.
C'est pour cette raison qu'il a fallu nous assurer que les résultats suivent une
distribution uniforme. 

La librairie Javacard d'Oracle met à notre disposition deux moteurs de génération
de nombres aléatoires : l'un est un algorithme pseudo aléatoire, l'autre 
cryptographiquement sûr.

\paragraph{}
Bien entendu, pour notre projet, nous avons utilisé l'algorithme décrit comme 
\og{}cryptographiquement sûr\fg{}. Cependant, par acquis de conscience, nous avons voulu
vérifier le niveau de l'aléatoire du générateur dit sûr à l'aide d'un
outil permettant de réaliser une analyse statistique dont le compte-rendu
est disponible dans le document 
\texttt{misc/testing\_randomization/Gen\_random.pdf}

\subsection{Chiffrement/Déchiffrement}
Dans l'étude des cartes à puce, nous avons également utilisé des méthodes de 
chiffrement et de déchiffrement sysmétriques et asymétriques. Pour notre cas 
d'utilisation, nous avons choisi des clefs RSA de 1024 bits pour l'algorithme de
chiffrement asymétrique RSA-PKCS1. Quant aux opérations cryptographiques 
symétriques, des clefs AES de 128 bits ont été générées et utilisées. Ces clefs
et algorithmes ont été choisis pour leur sûreté.

Nous avons testé ces algorithmes en chiffrant et en déchiffrant, à l'aide de 
clefs préalablement générées par la carte, divers messages. 

\paragraph{}
Ici, nous avons utilisé ces algorithmes dans différents cas. Concernant
le chiffrement par clef publique, bien qu'implanté, nous n'en avons pas eu 
besoin : c'est en effet Facecrypt, de la seconde partie du projet, qui s'en 
chargera, la clef publique lui ayant été fournie. Quant au cryptosystème 
sysmétrique, nous l'avons utilisé lors de la communication avec SoftCard
via un \og{}tunnel \fg{} sécurisé par AES-128. Ce dernier a pour objectif d'apporter 
confidentialité, intégrité et authentification aux données échangées entre la 
carte et SoftCard.

\subsection{Signature/Vérification}
Nous avons étudié un autre cas où le cryptosystème asymétrique peut être utilisé
via la carte à puce: la signature et la vérification de données. Grâce à la 
signature, nous pouvons obtenir authentification et non-répudiation, puisque la
signature se fait via la clef privée de l'émetteur. Une méthode de vérification
existe également permettant de vérifier l'authenticité des données via la clef 
publique du destinataire. Un booléen nous est alors retourné selon la réponse.

\label{CodePIN}
\subsection{Code PIN/PUK}
Une carte à puce étant associée à un utilisateur, il va de soi que ce dernier 
dispose d'un secret pour pouvoir la carte protéger. Il existe pour cela un code 
PIN, affecté à chaque carte, et dont seul l'utilisateur a connaissance.

Le code PIN est défini sur deux octets, ce qui représente $2^{16}$ = 65536 solutions.
Ceci est relativement faible, notamment contre une attaque de type 
\og{}bruteforce \fg{}, mais elle est contrée ici via à un nombre d'essais limité.
Dans notre cas nous limitons le nombre d'essais à trois. En cas de blocage de la
carte, dû à un nombre de tentatives trop élevé, seul le code PUK pourra 
débloquer la carte. Tout comme le code PIN, le nombre d'essais pour entrer le 
code PUK est de trois. Si ce nombre est dépassé la carte devient inutilisable et
la réinstallation des applets est alors la seule option pour la rendre
de nouveau opérationnelle.

\paragraph{}
Durant l'exécution de l'application, dès qu'une fonctionnalité sensible de la 
carte sera sollicitée, le code PIN de l'utilisateur lui sera demandé, bloquant 
ainsi temporairement l'accès à la carte. Au final, un déchiffrement, une signature
et une modification des identifiants engendreront une telle interrogation.

\subsection{SoftCard}
Pour permettre le dialogue avec la carte à puce, il a été nécessaire de développer
une application tierce : SoftCard. Comme nous l'avons mentionné précédemment,
la partie SSN avait besoin de certaines opérations ou de certaines données. 
C'est pourquoi cette application devait aussi servir d'intermédiaire entre 
FaceCrypt, détaillé plus loin, et Smartcard.

Etant donné que nous disposions déjà d'un environnement de développement en Java
pour les applets de SmartCard, SoftCard a été développé dans ce même langage. Il
permettait en outre de disposer des mêmes API que FaceCrypt.

Ainsi, SoftCard a été pensé comme un serveur : pour chaque requête reçue de 
FaceCrypt, une action est déclenchée. Celle-ci est traitée puis transmise à
SmartCard. Comme nous l'avons expliqué dans la partie \ref{CodePIN}, s'il s'agit
d'une opération \og sensible \fg, le code PIN est demandé à l'utilisateur.

\paragraph{}
Au final, les opérations supportées par SoftCard sont les suivantes:  
\begin{itemize}
    \item génération d'un nombre aléatoire;
    \item obtention de la clef publique
    \item déchiffrement de données;
    \item signature de données;
    \item enregistrement, modification et récupération des identifiants Facebook.
\end{itemize}

\paragraph{}
Afin de garantir la sécurité des communications entre SoftCard et SmartCard,
nous avons aussi implanté un tunnel entre ces deux entités. Pour plus 
d'informations, l'annexe \ref{TunnelSS} décrit notre raisonnement et son
fonctionnement.

\subsection{Difficultés rencontrées}
\paragraph{}
Une première catégorie de difficulté rencontrée était l'utilisation des fonctions
cryptographiques sur SoftCard et SmartCard.

Tout d'abord, malgré une bonne documentation JavaCard, certaines cartes 
n'implantent pas toutes les méthodes. Il nous a donc fallu nous assurer 
qu'elles existaient réellement, en testant parfois \og à l'aveuglette 
\fg{}. En outre, nous avons du faire de telle sorte que FaceCrypt et SoftCard 
utilisent les mêmes algorithmes cryptographiques, notamment au niveau des 
cryptosystèmes asymétriques.

% Taille des données dans l'APDU
\paragraph{}
Une seconde source de problèmes est provenue du fait que les informations que nous
ne pouvions pas envoyer à la carte sont de taille limitée. 

% Tunnel padding

% Taille des données dans l'APDU
% Les fonctions crypto (compatibilité avec Java, celles non implantées)
% padding

\subsection{Améliorations possibles}
Actuellement certaines améliorations sont possibles, notamment au niveau
de la gestion du code PIN et de celle de l'arrachage de la carte. En effet,
le code PIN n'est demandé que lorsque des informations sensibles sont
réquisitionnées auprès de la carte. On pourrait ainsi imaginer un déblocage
de la carte dès sa connexion au terminal. En outre, si aucune erreur n'est gérée,
lorsqu'une requête est envoyée à la carte après qu'elle ait subi une reconnexion
physique \og à chaud \fg{}, une erreur sera générée. Dans notre cas, étant donné
le temps qu'il nous restait lorsque nous nous sommes penchés sur le problème, 
nous avons ajouté une surveillance d'une levée d'exception, en forçant une 
reconnexion à la carte. Si à ce moment la carte n'est toujours pas insérée,
une erreur est soulevée, terminant ainsi le processus auquel s'est connecté
FaceCrypt. Une amélioration serait donc de se mettre en attente d'une connection
correcte avec la carte.

% Amélioration robustesse code


\clearpage

\section{Secure Social Network}
Ce projet avait initialement pour objectif d'étudier les différents 
procédés cryptographiques que nous pouvions utiliser pour sécuriser la vie
privée des utilisateurs vis-à-vis d'un réseau social. C'est Facebook qui a 
été choisi car plus pertinent étant donné son ampleur. 

\paragraph{}
Après la fusion des deux projets, il a été jugé intéressant d'utiliser 
la carte à puce comme outils d'authentification forte pour réaliser des 
opérations comme la génération de nombres aléatoires, le déchiffrement 
avec la clef privée par exemple.

Pour les autres besoins de chiffrement moins sensibles -- comme chiffrer
en utilisant un algorithme symétrique et une clef générée par la carte, ou
chiffrer avec une clef publique -- nous avons développé une application
Java \og{}FaceCrypt \fg{} qui permet de réaliser les opérations de chiffrement
plus rapidement que la carte.

Afin de proposer une solution de chiffrement au sein de Facebook, nous avons 
aussi conçu une extension pour \emph{Mozilla Firefox}, SSNExt, servant 
d'interface entre ce réseau social et l'utilisateur.

\subsection{FaceCrypt}
Comme mentionné précédemment, FaceCrypt est une application gérant une partie du
chiffrement du projet et faisant le relais entre l'extension Firefox et le 
composant SoftCard. 

Elle agit comme un serveur pour l'extension et comme un client pour SoftCard, 
c'est à dire que SSNExt va envoyer des requêtes à FaceCrypt qui va pour 
sa part traiter ces requêtes et les envoyer à SoftCard si besoin.

\paragraph{}
FaceCrypt fonctionne donc comme un démon et ne dispose donc pas d'interface
graphique, jugée inutile. Ce composant a été écrit en Java dans un premier 
temps afin de maximiser l'interopérabilité avec SoftCard (lui aussi écrit
Java). En effet, nous pensions pouvoir disposer des mêmes algorithmes 
que sur la carte. De plus, ce langage est le seul, avec le C, que tout le
groupe maitrisaît. Nous l'avons préféré au C pour les raisons cités ci-dessus
mais aussi pour le fait qu'il soit plus haut niveau et orienté objet.

\paragraph{}
FaceCrypt est divisé en trois parties : 

\subsubsection{Cryptographie}
Une partie majeure de FaceCrypt consiste en un certain nombre de modules 
permettant d'utiliser la cryptographie symétrique, asymétrique et des 
fonctions de hachage. Plusieurs algorithmes peuvent être utilisés mais
pour ce projet, notre choix s'est porté sur AES (CBC avec une clef de 256) et 
RSA, avec des clefs de 1024 bits.


\subsubsection{Communications sécurisées} % Pour Baptiste

\subsubsection{Base de données}
Pour gérer ses amis pouvant déchiffrer les messages qu'il poste, un utilisateur
peut les placer dans des listes. Celles-ci sont gérées par FaceCrypt et 
l'extension \emph{via} une base de données \texttt{sqlite}\footnote{Il s'agit
d'une base de données contenue dans un seul fichier pouvant être utilisé dans 
des programmes sans avoir à embarquer une base de données traditionnelle suivant
un modèle client-serveur}. 

FaceCrypt gère une base (donc un fichier) pour chaque utilisateur. Une base 
contient trois tables permettant de lier des amis à une liste, la table des 
amis contient aussi leur clef publique.

\subsection{SSNExt} % Maxence ou Zako

\subsection{Difficultés rencontrées}
Lors du développement de ce sous-projet, nous avons rencontré des difficultés
à de multiples endroits :
\begin{itemize}
    \item les manipulations de la page Facebook : un certain nombre d'éléments 
    que nous pensions triviaux à réaliser nous a finalement posé beaucoup de 
    problèmes, ceci dû au fait de la minutie des développeurs de Facebook à
    empêcher les utilisateurs de \og{}scripter \fg{} leurs actions sur le réseau
    social.
    \item les communications sécurisées entre FaceCrypt et SSNExt. % à finir.
\end{itemize}

\subsection{Améliorations possibles}
% signature de documents


\clearpage

\appendix

\section{Fonctionnement détaillé du tunnel entre la carte}
\label{TunnelSS}


\subsection{Objectifs}
Le tunnel entre la carte et le logiciel qui contr\^ole le lecteur sert à 
protéger les communications avec la carte. Contrairement aux autres liens entre 
logiciels, il n'a pas été possible d'utiliser un protocole de sécurisation tel 
que TLS car la carte ne dispose pas de pile TCP/IP. En implanter une étant hors
de notre domaine de compétences, nous avons utilisé les connaissances acquises 
en cours en cryptographie pour ajouter une couche de sécurité sur la liaison de 
données déjà présente. 

Les objectifs cryptographiques réalisés par le tunnel sont les suivants : 
\begin{description}
    \item[Confidentialité :] Les données ne peuvent pas être lues par une 
        personne non autorisée.
    \item[Intégrité :] Les données n'ont pas été modifiées durant leur 
        transport.
    \item[Authentification :] Les données ont été envoyées par une entité qui 
        connaît le secret.
\end{description}
Les deux derniers objectifs sont réalisables conjointement sans diminuer la 
sécurité du système alors que le premier nécessite une clé séparée \cite[section
2.5.1]{RGS ANSSI}. Dans la suite, le second objectif désigne à la fois 
l'intégrité et l'authentification. 


\subsection{Choix techniques}
Le choix d'AES nous a paru être le plus pertinent car c'est l'algorithme de 
référence. Il a remporté le concours lancé par le NIST après de nombreuses 
études par la communauté. Il a été approuvé par la NSA avec des clés de 128
bits pour protéger des données classifiées au niveau SECRET dans \cite{NSA}. 
L'annexe B1 du RGS publié par l'ANSSI affirme également que cette longueur est
satisfaisante \cite{RGS ANSSI}. 


Nous avons utilisé AES avec des clés (différentes) de 128 bits pour les deux 
objectifs. Le mode CBC permettant de faire du chiffrement et de 
l'authentification-intégrité, nous l'avons utilisé pour les deux. Pour conjuguer
performance et sécurité, nous avons implanté une version modifiée de CBC-MAC qui
intègre la taille du message en début de message. Cette modification garantit la
sécurité lorsque les messages ont une taille variable, comme indiqué dans 
\cite{CBC-MAC}.


\paragraph{}
L'établissement du tunnel se fait selon un protocole challenge-réponse 
réciproque pour assurer l'authentification mutuelle de la carte et du logiciel 
avec laquelle elle communique. Une clé de session est envoyée chiffrée par la 
carte pour assurer la confidentialité des données.

\subsection{Etablissement du tunnel}
L'authentification mutuelle peut être résumée comme ceci : 
\begin{itemize}
    \item génération d'un nonce client;
    \item envoi du nonce client à la carte;
    \item la carte récupère le nonce client, génère un nonce carte, un IV et une clé
        de session;
    \item la carte envoie la clé, le nonce client et le nonce carte chiffrés avec la
        cle partagée et l'IV;
    \item le client extrait l'IV et déchiffre le message avec la clé partagée;
    \item le client vérifie si le nonce client est présent pour authentifier la 
        carte, extrait la clé de session et récupère le nonce carte;
    \item le client génère un IV et renvoie le nonce carte avec la clé de session 
        établie;
    \item la carte récupère le nonce carte en déchiffrant le message reçu avec 
        la clé de session et vérifie s'il est identique à celui envoyé pour 
        authentifier le client.
\end{itemize}

C'est le constructeur de l'objet Java \og{}Tunnel \fg{} qui s'occupe de tout,
côté client.

\subsection{Communications dans le tunnel}
Une fois les entités mutuellement authentifiées et la clé de session échangée, 
les communications dans le tunnel peuvent se faire. 


L'objet \og{}Tunnel \fg{} présente une fonction d'envoi semblable à celle 
présente pour les envois sans tunnel. Il se charge de l'envoi vers la carte en
découpant préalablement le message en fragments de taille inférieure à la capacité 
du lien.


Cet objet présente également une fonction d'exécution qui permet de lancer les
fonctions présentes sur la carte selon les paramètres présents dans les données.

Le format général à l'entrée du tunnel est celui-ci : 

\paragraph{}
\begin{tabular}{|c |c|c|c|c|c|}
\hline
AID & Instruction & Paramètre 1  & Paramètre 2 & Longueur des données & Données \\
\hline
\end{tabular}
\paragraph{}
Ce format est identique à celui des APDU Javacard, la seule différence étant le 
r\^ole du premier octet qui nous sert à différencier les applets.  Ce 
fonctionnement est inspiré du modèle TCP/IP dans lequel le numéro de port 
indique vers quel programme envoyer les données. 


\paragraph{}
Afin de ne pas pouvoir prévoir le contenu du dernier bloc, la taille des 
fragments varie d'un message à l'autre. En effet, découper le message en 
fragments de taille fixe à chaque fois fait que le padding sera toujours 
identique et ceci nous a semblé une divulgation inutile d'information, 
le pire des cas étant le dernier bloc ne contenant que des 16. Nous avons décidé
d'envoyer entre 4 et 5 blocs AES, i.e. entre 64 et 80 octets pour chaque 
fragment, la valeur exacte étant définie aléatoirement pour chaque envoi de
message. Ainsi, le même message envoyé successivement pourra être découpé en
blocs de 68 octets la première fois et en blocs de 75 octets la seconde fois.
C'est un compromis entre performance, sécurité et facilité d'implémentation.

Le message est donc découpé en plusieurs fragments de même taille et en un 
dernier de taille inférieure pour le reste. Chaque fragment est ensuite complété
par un padding selon la norme PKCS7.

\paragraph{}
L'envoi dans le tunnel d'un fragment se fait comme ceci : 
\begin{itemize}
    \item génération d'un IV de transmission;
    \item chiffrement du fragment avec la clé de session et l'IV;
    \item concaténation de l'IV et des données chiffrées;
    \item calcul du CBC-MAC avec la taille totale ajoutée au début;
    \item concaténation de l'IV, des données chiffrées et du MAC;
    \item envoi physique vers la carte.
\end{itemize} 
\begin{thebibliography}{9}


\bibitem[CBC-MAC]{CBC-MAC}
    Mihir Bellare and Joe Kilian and Phillip Rogaway,
  \emph{The Security of the Cipher Block Chaining Message Authentication Code},
    2000

\bibitem[NSA]{NSA}
CNSS Secretariat, National Security Agency,
\emph{National Policy on the Use of the Advanced Encryption Standard (AES) to 
Protect National Security Systems and National Security Information},
Juin 2003

\bibitem[RGS ANSSI]{RGS ANSSI}
Agence nationale de la sécurité des systèmes d’information,
\emph{Annexe B1 - Référentiel Général de Sécurité}, 26 janvier 2010
\end{thebibliography}


\end{document}
