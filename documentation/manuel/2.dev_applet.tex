\documentclass[a4paper,11pt,french]{article}

\usepackage[utf8]{inputenc}
\usepackage[french]{babel}
\usepackage[colorlinks=true, linkcolor=black, urlcolor=blue]{hyperref}

\title{Installation du lecteur de cartes}

\begin{document}
\maketitle

\section{Environnement}
Une façon relativement simple de développer des applets pour des cartes
Javacard est d'utiliser Eclipse, disponible dans les dépôts. 

Le plugin Eclipse JCDE, disponible sur le site de SourceForge, permet de 
faciliter le développement de ces applets. Son installation se fait en copiant
tous les fichiers de l'archive dans le répertoire "plugins" d'Eclipse. (Par
défaut : \emph{/usr/lib/eclipse/plugins})


\section{Le JDK}
Oracle fournit un SDK Javacard qu'il est possible télécharger sur son site.
La dernière version du SDK Javacard pour GNU/Linux est la 2.2.2. Cependant,
cette version utilisant le JDK 1.5, il faut la télécharger et modifier la 
variable "\$JAVA\_HOME" :
\begin{verbatim}
JAVA_HOME=<chemin vers le jdk 1.5>

\end{verbatim}

La variable "\$JC\_HOME" indique le chemin vers le JDK :
\begin{verbatim}
JC_HOME=<chemin vers le javacard_dk 2.2.2>

\end{verbatim}

Enfin, la variable "PATH" doit être modifiée pour qu'il soit possible, par
exemple, de tester une applet via un simulateur.
\begin{verbatim}
PATH=$PATH:$JAVA_HOME/bin:$JC_HOME/bin

\end{verbatim}

\section{Création d'un projet}
Sur Eclipse, le projet est créé en choisissant le type "Javacard". Cependant,
à ce stage, le chemin vers le dossier du SDK Javacard n'a pas encore été donné
à Eclipse et une erreur se déclenche. Elle est corrigeable dans Window > 
Preferences > Java Card.

De plus, il peut être nécessaire de spécifier quel JDK doit utiliser Eclipse
pour compiler le projet : Project > Properties > Java Compiler puis "Compiler
compilance level: 1.5".\\

Un exemple d'applet est disponible dans la branche "sc" : lib/samples/gen\_random.java\\

Avant d'effectuer la simulation, il faut spécifier l'AID (l'identifiant
unique) du package et de l'applet : Java Card Tools > Set Package AID 
puis Set Applet AID.\\


Etant donné que les dialogues se font par envois d'APDUs et qu'avant
l'interrogation de l'applet, il faut la installer et sélectionner, il est
nécessaire d'envoyer de telles instructions. JCDE met à disposition des outils
permettant de générer des scripts contenant ces APDUs. Cette génération se
fait en utilisant "Java Card Tools > Generate Script".\\

Dans le cas d'une simulation de communication, il est possible d'utiliser
JCWDE (créant une VM sur une carte (simulée)) et l'outil "apdutool" qui va simuler
l'envoi d'ADPUs à cette VM. Pour l'installation et la sélection de l'applet,
seront envoyés ceux générés précédemments (dans create-<nomApplet>.txt et
select-<nomApplet>.txt).\\


Concernant l'application cliente, un exemple est disponible :
samples/client\_gen\_random.java. Notons que contrairement à l'applet, il est
possible d'utiliser une version de Java en version supérieure à 1.5.\\

L'installation de cette applet se fera en suivant les instructions du fichier
"3.installation\_applet.pdf"\\

\end{document}
