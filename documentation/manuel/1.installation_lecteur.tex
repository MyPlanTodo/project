\documentclass[a4paper,11pt,french]{article}

\usepackage[utf8]{inputenc}
\usepackage[french]{babel}
\usepackage[colorlinks=true, linkcolor=black, urlcolor=blue]{hyperref}

\title{Installation du lecteur de cartes}

\begin{document}
\maketitle

\section{Téléchargement}
Une étape indispensable et préalable est l'installation de quelques librairies.
\begin{verbatim}
apt-get install libudev-dev libcurl4-openssl-dev pcscd pcsc-tools
\end{verbatim}

pcsc-lite : \url{http://pcsclite.alioth.debian.org/}\\
globalplatform \& gppcscconnectionplugin : \url{http://sourceforge.net/projects/globalplatform/files/GlobalPlatform%20Library/GlobalPlatform%20Library%206.0.0/}\\
gpshell : \url{http://sourceforge.net/projects/globalplatform/files/latest/download?source=files}

\section{Installation}
L'installation se fait en entrant les lignes :
\begin{verbatim}
./configure --prefix=/usr
make && make install
\end{verbatim}

Il est enfin possible de tester la reconnaissance du lecteur et de la carte
via l'utilitaire "pcsc\_scan".

\end{document}
