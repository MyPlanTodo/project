\documentclass[a4paper,11pt,french]{article}

\usepackage[utf8]{inputenc}
\usepackage[french]{babel}
\usepackage[colorlinks=true, linkcolor=black, urlcolor=blue]{hyperref}

\title{Installation du lecteur de cartes}
\author{Emmanuel \textsc{Mocquet}}

\begin{document}
\maketitle

\section{Installation}
L'installation des librairies et outils nécessaires à la reconnaissance et
à l'utilisation de la carte se fait via le script "install" situé 
dans le répertoire \texttt{/lib}.

Il est ensuite possible de vérifier que le système est bien configuré et prêt
à utiliser la carte en utilisant l'outil "pcsc\_scan". Si la carte est détectée,
diverses informations seront affichées; sinon, le système entrera dans une boucle
infinie de tentative de détection.

\section{Développement d'applets}

\subsection{Note}
La plupart des fichiers mentionnés dans cette partie sont disponibles dans le 
répertoire \texttt{/lib}.

\subsection{Environnement}
Une façon relativement simple de développer des applets pour des cartes
Javacard est d'utiliser l'IDE Eclipse, disponible dans les dépôts. 

Le plugin Eclipse JCDE permet de faciliter le développement de ces applets. 
Son installation se fait en copiant tous les fichiers JAR du répertoire \texttt{lib/JCDE}
dans le répertoire "plugins" d'Eclipse (par défaut : \emph{/usr/lib/eclipse/plugins})


\subsection{Le JDK}
Oracle fournit un SDK Javacard téléchargeable sur son site.
La dernière version du SDK Javacard pour GNU/Linux est la 2.2.2. Cette version 
utilisant le JDK 1.5, il sera nécessaire de configurer le projet sous Eclipse en
conséquence.

\subsection{Création et configuration d'un projet}
Sur Eclipse, le projet est créé en choisissant le type "Javacard". Cependant,
à ce stade, le chemin vers le dossier du SDK Javacard n'a pas encore été donné
à Eclipse et une erreur sera déclenchée. Elle est corrigeable en entrant ce chemin
dans Window $\rightarrow$ Preferences $\rightarrow$ Java Card.

De plus, il peut être nécessaire de spécifier quel JDK doit utiliser Eclipse
pour compiler le projet : Project $\rightarrow$ Properties $\rightarrow$ Java 
Compiler puis choisir "Compiler compilance level: 1.5".\\

Des exemples d'applets commentés sont disponibles dans le répertoire \texttt{lib/samples}
et devraient permettre de se faire une idée de la structure générale d'une 
applet. Pour se renseigner plus en détails sur le développement d'applets, 
ce tutoriel est de bonne qualité, bien qu'orienté Windows : 
\url{http://julienb.developpez.com/tutoriels/java/introjavacard/}

\subsection{Préparation de l'installation sur carte}
\subsubsection{Etablissement d'AID}
Avant de déployer l'applet sur sa carte, il faut donner une identifiant unique
au package et à ses applets (un AID), en faisant en sorte que l'AID de chaque applet
corresponde à celui du package (de 5 octets) complété par au minimum octet. 
Par exemple si l'AID du package est 
\begin{verbatim}
0x01 0x02 0x03 0x04 0x05 0x06 0x07 0x08 0x09 0x00
\end{verbatim}
un AID d'applet pourra être :
\begin{verbatim}
0x01 0x02 0x03 0x04 0x05 0x06 0x07 0x08 0x09 0x00 0x01
\end{verbatim}
Ces AIDs sont modifiables en cliquant droit sur le package ou l'applet en question
puis en choisissant Java Card Tools $\rightarrow$ Set Package AID/Set Applet AID. \\

\subsubsection{Conversion du package}
Avant toute utilisation d'applet, il faut bien évidemment passer par une phase
d'installation. Le SDK Java Card met à disposition un outil permettant de convertir
le package à installer en un fichier \texttt{.cap}.

Avec Eclipse/JCDE, cet outil est disponible en cliquant droit sur le package,
en ouvrant le sous-menu "Java Card Tools" puis en sélectionnant "Convert".


\subsection{Application cliente}
Concernant l'application cliente, et contrairement aux applets installés sur 
carte, il est possible d'utiliser une version de Java en version supérieure à 1.5.

Comme pour les applets, des exemples de clients sont disponibles dans le répertoire
\texttt{lib/samples}.
%%\subsubsection{Communication}
%%Les dialogues entre une carte et le terminal se font par l'envoi d'APDU 
%%(Application Protocol Data Unit). Il s'agit d'une chaîne d'octets structurée ainsi
%%
%%\begin{verbatim}
%%CLA : classe d'instruction
%%INS : code d'instruction
%%P1 : paramètre optionnel
%%P2 : paramètre optionnel
%%Lc : nombres d'octets de la partie Data
%%Data : données passées en paramètre
%%Le : nombres d'octets attendus en retour
%%\end{verbatim}
%%
%%Le SDK proposé par Oracle permet de s'abstraire plus ou moins de cette notion,
%%notamment grâce à l'utilisation des classes "CardChannel" et "CommandAPDU".


\section{Installation des applets sur la carte}
Prérequis : librairies GlobalPlatform, PCSC, et gpshell fonctionnels.\\

Pour tester l'interrogation de la carte (installation, suppression,
listing), il existe des scripts fournis par GPShell, situés dans \texttt{lib/gpshell-1.4.4}.
Etant donné que les cartes sont compatibles avec la norme GlobalPlatform 2.1.1,
les scripts à utiliser sont ceux dont le nom finit par "GP211.txt".
L'utilisation de ces scripts se fait en entrant la commande : 
\begin{verbatim}
gpshell fichier.txt
\end{verbatim}

Pour installer une applet, il faut fournir plusieurs directives dans le
fichier passé à GPshell:
\begin{verbatim}
// Protocole utilisé pour la norme GlobalPlatform 2.1.1
mode_211

// Commande nécessaire avant toute communication avec la carte :
// établissement d'un contexte
establish_context

// Connexion à la carte
// Si le lecteur n'est pas explicité, le premier branché est 
// sélectionné
card_connect

// Sélection du "Security Domain". Cette valeur est la même pour
// toutes les cartes suivant la norme GlobalPlatform 2.1.1.
// Le security domain est une applet de confiance qui va donner un
// jeton de session à l'utilisateur qui sera, par la suite, correctement
// authentifié auprès de la carte.
select -AID a000000003000000

// Authentification et établissement d'un tunnel avec la carte
// security 1 : tunnel avec code MAC 
// keyind 0 : numéro d'index de la clef
// keyver 0 : version de la clef (à laisser inchangé)
// Les clefs spécifiées correspondent à celles stockées par défaut
open_sc -security 1 -keyind 0 -keyver 0 -mac_key \
404142434445464748494a4b4c4d4e4f -enc_key 404142434445464748494a4b4c4d4e4f

// Suppression de notre applet (utile seulement pour une
// réinstallation)
delete -AID 0102030405060708090000                                              

// Suppression de son package
delete -AID 01020304050607080900                                                
                                                                                
// Installation en fournissant le package compilé et les
// droits sur ce package (-priv)
install -file pack.cap -priv 2
                                                                                
// Déconnexion de la carte et "libération du contexte"
card_disconnect          
release_context                                                                 
\end{verbatim}

L'applet pourra alors être utilisée avec une application Java cliente associée.
\end{document}
