\documentclass[a4paper,11pt,french]{article}

\usepackage[utf8]{inputenc}
\usepackage[french]{babel}
\usepackage[colorlinks=true, linkcolor=black, urlcolor=blue]{hyperref}

\title{Installation des applets sur la carte}

\begin{document}
\maketitle

Prérequis : les librairies GlobalPlatform, PCSC, et gpshell fonctionnels

Pour tester l'interrogation de la carte (installation, suppression,
listing), il existe des scripts fournis par GPShell. Etant donné que nos
cartes sont compatibles avec la norme GlobalPlatform 2.1.1, il ne faut utiliser
que les scripts se terminant par "GP211.txt".
L'utilisation de ces scripts se fait en entrant la ligne : 
\begin{verbatim}
gpshell fichier.txt
\end{verbatim}

Pour installer une applet, il faut fournir plusieurs directives dans le
fichier passé à GPshell:
\begin{verbatim}
// Protocole utilisé pour la norme GlobalPlatform 2.1.1
mode_211

// Commande nécessaire avant toute communication avec la carte :
// établissement d'un contexte
establish_context

// Connexion à la carte
// Si le lecteur n'est pas explicité, le premier branché est 
// sélectionné
card_connect

// Sélection du "Security Domain". Cette valeur est la même pour
// toutes les cartes suivant la norme GlobalPlatform 2.1.1.
// Le security domain est une applet de confiance qui va donner un
// jeton de session à l'utilisateur qui sera, par la suite, correctement
// authentifié
select -AID a000000003000000

// Authentification et établissement d'un tunnel avec la carte
// security 1 : tunnel avec code MAC 
// keyind 0 : numéro d'index de la clef
// keyver 0 : version de la clef (à laisser inchangé)
// Les clefs spécifiées correspondent à celles stockées par défaut
open_sc -security 1 -keyind 0 -keyver 0 -mac_key \
404142434445464748494a4b4c4d4e4f -enc_key 404142434445464748494a4b4c4d4e4f

// Suppression de notre applet (utile seulement pour une
// réinstallation)
delete -AID 0102030405060708090000                                              

// Suppression de son package
delete -AID 01020304050607080900                                                
                                                                                
// Installation en fournissant le package compilé et les
// droits sur ce package (-priv)
install -file pack.cap -priv 2
                                                                                
// Déconnexion de la carte et "libération du contexte"
card_disconnect          
release_context                                                                 

\end{verbatim}

\end{document}
