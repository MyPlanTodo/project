\documentclass[a4paper,10pt]{article}
\usepackage[utf8]{inputenc}
\usepackage{graphicx, amssymb, xcolor, listings}
\usepackage[colorlinks=true,linkcolor=black,urlcolor=blue]{hyperref}
\usepackage[top=2.6cm,bottom=2.6cm,left=2.6cm,right=2.6cm]{geometry}
\usepackage[french]{babel}

\hypersetup{breaklinks=true}
\definecolor{gris}{rgb}{0.9,0.9,0.9}

\title{Gestion des certifications}
\author{Florian \textsc{Guilbert}}

%opening
\begin{document}

\maketitle
\tableofcontents

Le projet est livré avec une PKI permettant à chaque composant de communiquer
de manière sécurisé. Il est possible d'utiliser vos propres certificats. 
Nous présentons dans ce document la marche à suivre permettant d'adapter 
nos outils pour utiliser vos certificats.

\paragraph{}
Pour la suite, et pour faciliter l'établissement de la PKI nous supposerons
que tous les certifications que nous allons utilisé sont sujettes à la 
même autorité. Et que vous disposer déjà d'un certificat racine, dans 
les exemples ci-dessous, le certificat racine sera \texttt{ca.pem} et 
il y aura un certificat intermédiaire \texttt{cassl.pem}.

\section{SoftCard}
Ce composant a besoin d'un certificat serveur que nous créons avec les 
commandes suivantes : 
\begin{verbatim}
$ openssl genrsa -out softCardServer.key -des3 2048
$ openssl req -new -key softCardServer.key -out softCardServer.csr \
    -config ./openssl.cnf
$ openssl ca -config ./openssl.cnf -name CA_ssl_default -extensions \
    SERVER_RSA_SSL -infiles cassl/softCardServer.csr
\end{verbatim}
Qui nous permettent d'obtenir un certificat pour le SoftCard que nous 
appellerons dans la suite \texttt{softCardServer.pem}.

\paragraph{}
Pour pouvoir utiliser ce certificat au sein de l'application java, nous allons
créer un PKCS\#12 que nous convertissons en \emph{Java KeyStore} : 
\begin{verbatim}
$ openssl pkcs12 -export -inkey softCardServer.key -in softCardServer.pem \ 
    -out softCardServer.p12 -name "SoftCard Server Certificate "
$ keytool -importkeystore -deststorepass motDePasse -destkeypass \ 
    motDePass -destkeystore softCardServer.jks -srckeystore \ 
    softCardServer.p12 -srcstoretype PKCS12 -srcstorepass motDePasse \ 
    -alias "Softcard Server Certificate"
\end{verbatim}

Pour finir, il nous faut un \emph{package} contenant les certificats pour 
lesquelles l'application aura confiance, ce \emph{package} est un 
\emph{trustStore}
contenant des certificats au formats \emph{DER}.
\begin{verbatim}
commande
\end{verbatim}

Il faut ensuite placer les fichiers .jks dans le répertoire \texttt{cert}
de SoftCard. Au cas où vos \emph{passphrases} sont différents de 
\og lolilol \fg{} il faut modifier le code source du fichier 
\texttt{SoftCardServer} puis recompiler l'application.

\section{FaceCrypt}
La démarche est très semblable à celle pour SoftCard sauf qu'il faut 
créer deux certificats pour FaceCrypt un client servant pour sa communication
avec SoftCard et l'autre serveur pour le dialogue avec l'extension.

Les commandes : 
\begin{verbatim}
$ openssl genrsa -out facecryptClient.key -des3 2048
$ openssl req -new -key facecryptClient.key -out facecryptClient.csr \
    -config ./openssl.cnf
$ openssl ca -config ./openssl.cnf -name CA_ssl_default -extensions \
    SERVER_RSA_SSL -infiles facecryptClient.csr
\end{verbatim}
et 
\begin{verbatim}
$ openssl genrsa -out facecryptServer.key -des3 2048
$ openssl req -new -key facecryptServer.key -out facecryptServer.csr \
    -config ./openssl.cnf
$ openssl ca -config ./openssl.cnf -name CA_ssl_default -extensions \
    CLIENT_RSA_SSL -infiles facecryptServer.csr
\end{verbatim}
créent les deux certificats.

\paragraph{}
Ensuite, comme pour SoftCard, il faut créer une enveloppe PKCS\#12 pour ces
deux certificats et les convertir en \texttt{.jks} : 
\begin{verbatim}
$ openssl pkcs12 -export -inkey facecryptClient.key -in facecryptClient.pem \ 
    -out facecryptClient.p12 -name "FaceCrypt Client Certificate"
$ keytool -importkeystore -deststorepass motDePasse -destkeypass \ 
    motDePass -destkeystore facecryptClient.jks -srckeystore \ 
    facecryptClient.p12 -srcstoretype PKCS12 -srcstorepass motDePasse \ 
    -alias "FaceCrypt Client Certificate"
$ openssl pkcs12 -export -inkey facecryptServer.key -in facecryptServer.pem \ 
    -out facecryptServer.p12 -name "FaceCrypt Server Certificate"
$ keytool -importkeystore -deststorepass motDePasse -destkeypass \ 
    motDePass -destkeystore facecryptServer.jks -srckeystore \ 
    facecryptServer.p12 -srcstoretype PKCS12 -srcstorepass motDePasse \ 
    -alias "FaceCrypt Server Certificate"
\end{verbatim}
Une fois cette étape terminée, il faut maintenant créer un \emph{trustStore}
pour indiquer les certicats dont FaceCrypt a confiance.
\begin{verbatim}
commande
\end{verbatim}

Il faut ensuite placer les fichiers .jks dans le répertoire \texttt{cert}
de FaceCrypt. Au cas où vos \emph{passphrases} sont différents de 
\og lolilol \fg{} il faut modifier le code source du fichier 
\texttt{ServerSSL} puis recompiler l'application.

\section{SSNExt}
Cette partie n'a besoin que d'un certificat : 
\begin{verbatim}
$ openssl genrsa -out cassl/extensionClient.key -des3 2048
$ openssl req -new -key extensionClient.key -out extensionClient.csr \
    -config ./openssl.cnf
$ openssl ca -config ./openssl.cnf -name CA_ssl_default \ 
    -extensions CLIENT_RSA_SSL -infiles cassl/extensionClient.csr
$ openssl pkcs12 -export inkey extensionClient.key in extensionClient.pem \
        -out extensionClient.p12 -name "Extension Client Certificate"
\end{verbatim}
Une fois cela fait, il faut importer dans le navigateur cette enveloppe 
PKCS\#12 ainsi que les certificats \texttt{cassl.pem} et \texttt{ca.pem}, 
à placer respectivement dans les onglets serveurs et autorités.
\end{document}
