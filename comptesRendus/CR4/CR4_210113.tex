\documentclass[a4paper,10pt]{article}
\usepackage[utf8]{inputenc}
\usepackage{graphicx, amssymb, xcolor, listings}
\usepackage[colorlinks=true,linkcolor=black,urlcolor=blue]{hyperref}
\usepackage{tikz}
\usepackage[top=2.6cm,bottom=2.6cm,left=2.6cm,right=2.6cm]{geometry}
\usepackage[french]{babel}

\author{Florian \textsc{Guilbert}}
\title{Compte-rendu de réunion, 4}
\date{21 janvier 2013}

\begin{document}

\maketitle

\subsubsection*{Participants : }
\begin{itemize}
    \item Magali \textsc{Bardet} (cliente);
    \item Zakaria \textsc{Addi};
    \item Baptiste \textsc{Dolbeau};
    \item Yicheng \textsc{Gao};
    \item Florian \textsc{Guilbert};
    \item Giovanni \textsc{Huet};
    \item Emmanuel \textsc{Mocquet};
    \item Maxence  \textsc{Péchoux}.
    \item Romain \textsc{Pignard}.
\end{itemize}

\subsubsection*{Absents : }
Pas d'absents.

\subsection*{Ordre du jour : présentation de la STB, planning, analyse 
des risques}

Pour tout :  préciser si c'est les clefs privées de signature ou clef publique
de chiffrement

\paragraph{1. Remarque sur la STB}

\begin{itemize}
    \item Génération nombre aléa : indistingable en temps polynomial, 
    distribution de probabilité uniforme.
    \item Tests : vérifier l'uniformité de l'aléa mettre la notion/définition 
d'aléa dans la terminologie
    \item Code PIN : description à changer par : "authentification de 
    l'utilisateur auprès de la carte par code PIN"
    \item Rendre les cas d'utilisations plus génériques.
    \item Blocage de la carte : par paramètre (5 min exemple) + 
    utilisation d'un code "PUK" pour dévérouiller la carte.
    \item Algo qui vérifie le code PIN doit être résistant aux attaques (par 
            canaux cachés) et doit être lent $\rightarrow$ à mettre dans les exigences 
        fonctionnelles
    \item "Authentification sur SocialNetwork":  remplacer par "Transmission"
précondition:  smartcard s'authentifie auprès de FaceCrypt
    \item Réfléchir au choix clef publique/clef privée ou clef partagée 
    \item C3 + C4 : à revoir, authentification qui ne concerne pas 
    l'utilisateur va dans les exigences
    \item C6 : envoi de la clef publique et l'algo de chiffrement, pas besoin
    d'un certificat ?
    \item Vérification du certificat ! Le chiffrement de la clef de message 
    fait par FaceCrypt, donc C6 à enlever.
    \item Exigence opérationnelle : changer F-FO-20 par "on doit être capable 
    de vérifier que le nombre est bien aléatoire"
    \item Exigence de qualité : notion de rapidité de chiffrement/signature à 
fournir (Indispensable) (pareil pour facebook)
\end{itemize}


\paragraph{2. Remarque sur le Plan de développement}
Dernière répétition avec Mme Bardet possible le vendredi 22 février

\paragraph{3. Analyse des risques}
Inverser la description et le facteur pour le retard, pareil pour 
l'apprentissage.

\subsection*{Prochaine réunion}
    Vendredi suivant.    
\end{document}
