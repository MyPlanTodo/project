\documentclass[a4paper,10pt]{article}
\usepackage[utf8]{inputenc}
\usepackage{graphicx, amssymb, xcolor, listings}
\usepackage[colorlinks=true,linkcolor=black,urlcolor=blue]{hyperref}
\usepackage{tikz}
\usepackage[top=2.6cm,bottom=2.6cm,left=2.6cm,right=2.6cm]{geometry}
\usepackage[french]{babel}

\author{Florian \textsc{Guilbert}}
\title{Compte-rendu de réunion, 9}
\date{13 février 2013}

\begin{document}

\maketitle

\subsubsection*{Participants : }
\begin{itemize}
    \item Ayoub \textsc{Bardet} (client);
    \item Zakaria \textsc{Addi};
    \item Baptiste \textsc{Dolbeau};
    \item Florian \textsc{Guilbert};
    \item Emmanuel \textsc{Mocquet};
    \item Maxence  \textsc{Péchoux}.
    \item Romain \textsc{Pignard}.
\end{itemize}

\subsubsection*{Absents : }
\begin{itemize}
    \item Yicheng \textsc{Gao};
    \item Giovanni \textsc{Huet};
\end{itemize}

\subsection*{Ordre du jour : Point sur l'avancement pour chaque membre de l'équipe}


\paragraph{Zakaria Addi :} (Travaille à 100\% sur l'extension)\\
\begin{itemize}
    \item Accompli pour le moment : création d'une extension firefox, modification
        du DOM avec JQuery, communication sécurisé Extension/FaceCrypt (SSL).
    \item Prochains objectifs : développement de classe implantant le protocole
        de communication entre FaceCrypt et l'extension (Tunnel), intégrer la
        partie communication dans l'extension.
\end{itemize}

\paragraph{Baptiste Dolbeau :} (Travaille à 100\% sur FaceCrypt)
\begin{itemize}
\item Accompli pour le moment : Chiffrement symétrique (test des algorithmes
        disponibles), mise en place d'un système client/serveur (SSL), 
        modifications de ce système pour communiquer avec SoftCard et 
        l'extension, création des certifications pour des authentifications
        bidirectionnelles entre les parties.
\item Prochains objectifs : mettre en place la communication avec l'extension
        via des objets JSon. 
\end{itemize}

\paragraph{Yicheng Gao :} (Travaille à 100\% sur le projet SmartCard)
\begin{itemize}
    \item Accompli pour le moment : La mise en place de l'environnement de 
        développement de JAVAcard, compiler et charger une applet Javacard sur 
        une carte physique ainsi l'’application cliente pour un simulateur et 
        une vrais carte, les documentations d'applet Cypher
\item Prochains objectifs :  Test sur l'applet GenRadom pour les bits 
        différents (en ce moment le test concerne 2 bits) et le compte-rendu 
        complet.
\end{itemize}

\paragraph{Giovanni Huet : } (Travaille à 50-50 entre SmartCard et SoftCard) 
\begin{itemize}
    \item Accompli pour le moment : développement d'une applet réalisant la 
        signature/vérification, un jeu de test permettant de vérifier le bon 
        fonctionnement et le temps requis :
            \begin{itemize}
            \item du chiffrement rsa
            \item du déchiffrement rsa
            \item de la signature rsa
            \item de la vérification correspondante
            \item de la génération d'un nombre aléatoire
            \end{itemize}
\item Prochains objectifs : rédaction du cahier de recette (à vérifier et 
        compléter)
\end{itemize}
    
\paragraph{Emmanuel Mocquet : } (Travaille à 60\% sur SoftCard et 40\% sur SmartCard)
\begin{itemize}
    \item Accompli pour le moment : manuel d'installation de l'environnement de 
        developpement, classe Cypher permettant de chiffrer/déchiffrer et
        d'obtenir la clef publique, tunnel avec FaceCrypt.
\item Prochains objectifs : mise à jour du manuel, intégration de SoftCard,
        administration, test sur le déverrouillage par code PIN (timing).
\end{itemize}

\paragraph{Maxence Péchoux : } (Travaille à 95\% sur l'extension et 5\% sur FaceCrypt)
\begin{itemize}
    \item Accompli pour le moment : création d'une extension firefox, modification
        du DOM avec JQuery, interception d'un Post (message de statut), 
        insertion/récupération clef publique.intégration de l'extension finale 
        (en cours).
\item Prochains objectifs : Terminer l'intégration, créer FBExt qui dialogue
        avec la classe Tunnel, commenter et documenter le code.
\end{itemize}

\paragraph{Romain Pignard : } (Travaille à 90\% sur SmartCard et 10\% sur SoftCard)
\begin{itemize}
    \item Accompli pour le moment : applets echo (pour les tests), génération 
        des nombres aléatoires, vérification de PIN, tunnel sécurisé 
        (authentification, confidentialité et intégrité) entre la carte et le 
        lecteur, développement de classe outils (concaténation, 
        extraction IV, ...), un peu de test.
\item Prochains objectifs : Intégration des applets avec SoftCard, test de 
        performance et amélioration si possible, poursuite de la documentation.
\end{itemize}

\paragraph{Florian Guilbert : } (Travaille à 50-50 entre FaceCrypt et l'extension)
\begin{itemize}
    \item Accompli pour le moment : Sur FaceCrypt, développement de classes 
        de chiffrement asymétrique, de hachage et une classe permettant de 
        manipuler la base de données contenant la liste des amis. Création de
        cette même classe en JavaScript pour l'extension. 
\item Prochains objectifs : développer l'IHM de l'extension pour gérer les 
        listes d'amis et étudier l'upload d'images sur Facebook.
\end{itemize}

\subsection*{Prochaine réunion}
    mercredi 20 février 10h
    
\end{document}
