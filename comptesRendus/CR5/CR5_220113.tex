\documentclass[a4paper,10pt]{article}
\usepackage[utf8]{inputenc}
\usepackage{graphicx, amssymb, xcolor, listings}
\usepackage[colorlinks=true,linkcolor=black,urlcolor=blue]{hyperref}
\usepackage{tikz}
\usepackage[top=2.6cm,bottom=2.6cm,left=2.6cm,right=2.6cm]{geometry}
\usepackage[french]{babel}

\author{Florian \textsc{Guilbert}}
\title{Compte-rendu de réunion, 5}
\date{22 janvier 2013}

\begin{document}

\maketitle

\subsubsection*{Participants : }
\begin{itemize}
    \item Ayoub \textsc{Otmani} (client);
    \item Zakaria \textsc{Addi};
    \item Baptiste \textsc{Dolbeau};
    \item Yicheng \textsc{Gao};
    \item Florian \textsc{Guilbert};
    \item Giovanni \textsc{Huet};
    \item Emmanuel \textsc{Mocquet};
    \item Maxence  \textsc{Péchoux}.
    \item Romain \textsc{Pignard}.
\end{itemize}

\subsubsection*{Absents : }
Pas d'absents.

\subsection*{Ordre du jour : présentation de la STB, planning, problème de la 
    distribution des clefs publiques.}


\paragraph{1. Remarque sur la STB}

\begin{itemize}
\item Génération des clefs de sessions par la carte à puces, pas FaceCrypt.
\item Besoin d'une configuration qui spécifie si on déchiffre à la volée ou si 
l'utilisateur doit demander le déchiffrement explicitement, par défaut,
un bouton sera présent pour que l'utilisateur demande le déchiffrement.
\item Construction de nos propres listes d'amis (pas celles de Facebook)
\item Dans les "exigences interface" -> ajouter "bouton" pour déchiffrer
\item Rajouter que la taille des mots de passe doit être importante
\item Hypothèse : les données transitant entre l'extension et FaceCrypt sont en 
claires, dans ce cas, besoin de chiffrer mdp, via crypt.js
\end{itemize}

\paragraph{2. Distribution des clefs}
\begin{itemize}
\item Stockage des clefs publiques : pseudo-pgp; clef stockée sur la page perso 
des utilisateurs.
\item Récupération : clef donnée dès qu'on ajoute "l'ami" à notre liste
\item Implanter le modèle x509 si nous avons le temps.
\end{itemize}

\subsection*{Prochaine réunion}
    Lundi 28 16h 
\end{document}
