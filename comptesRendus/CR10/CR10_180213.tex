\documentclass[a4paper,10pt]{article}
\usepackage[utf8]{inputenc}
\usepackage{graphicx, amssymb, xcolor, listings}
\usepackage[colorlinks=true,linkcolor=black,urlcolor=blue]{hyperref}
\usepackage{tikz}
\usepackage[top=2.6cm,bottom=2.6cm,left=2.6cm,right=2.6cm]{geometry}
\usepackage[french]{babel}

\author{Florian \textsc{Guilbert}}
\title{Compte-rendu de réunion, 10}
\date{18 février 2012}

\begin{document}

\maketitle

\subsubsection*{Participants : }
\begin{itemize}
    \item Magali \textsc{Bardet} (cliente);
    \item Zakaria \textsc{Addi};
    \item Baptiste \textsc{Dolbeau};
    \item Yicheng \textsc{Gao};
    \item Florian \textsc{Guilbert};
    \item Giovanni \textsc{Huet};
    \item Emmanuel \textsc{Mocquet};
    \item Maxence  \textsc{Péchoux}.
    \item Romain \textsc{Pignard}.
\end{itemize}

\subsubsection*{Absents : }
Pas d'absents

\subsection*{Ordre du jour : Revue avec le client}

Le client souhaite ajouter une nouvelle fonctionnalité dans le projet : 
\paragraph{}
Possibilité de lancer une connexion ssh sur une machine distante avec un
système OTP (One Time Passwd, cf https://tools.ietf.org/html/rfc2289),
en faisant générer les OTP par la carte à puce (la carte affiche le
résultat, et l'utilisateur est chargé de copier le code dans son
terminal ou autre).

\paragraph{Installation du système SmartCard} Le client a tenté d'installer
le système, mais ça a été un peu laborieux, nous allons donc faire un
script d'installations pour faciliter le déploiement.

\subsection*{Prochaine réunion}
01/02/13 : Livraison projet.

\end{document}
