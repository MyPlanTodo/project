\documentclass[a4paper,10pt]{article}
\usepackage[utf8]{inputenc}
\usepackage[colorlinks=true,linkcolor=black,urlcolor=blue]{hyperref}
\usepackage[top=2.6cm,bottom=2.6cm,left=2.6cm,right=2.6cm]{geometry}
\usepackage[french]{babel}

\usepackage{lib/tikz-uml}

\author{Florian \textsc{Guilbert}}
\title{Compte-rendu de réunion, 1}
\date{30 novembre 2012}

\begin{document}

\maketitle

\subsubsection*{Participants : }
\begin{itemize}
    \item Magali \textsc{Bardet} (cliente);
    \item Ayoub \textsc{Otmani} (client);
    \item Baptiste \textsc{Dolbeau};
    \item Florian \textsc{Guilbert};
    \item Giovanni \textsc{Huet};
    \item Maxence  \textsc{Péchoux};
    \item Romain \textsc{Pignard}.
\end{itemize}

\subsubsection*{Absents : }
\begin{itemize}
    \item Zakaria \textsc{Addi};
    \item Yicheng \textsc{Gao};
    \item Emmanuel  \textsc{Mocquet}.
\end{itemize}

\subsection*{Ordre du jour : Clarification du sujet}

\paragraph{Diagramme de séquence :}

    Les acteurs sont : 
    \begin{itemize}
        \item User : utilisateur du système;
        \item SoftCard : logiciel dialoguant avec la carte;
        \item SmartCard : carte à puce, contient la clef privée de l'utilisateur;
        \item Browser-Proxy : logiciel ouvrant une fenêtre permettant de visualiser le réseau social,
            et dialoguant avec le SoftCard;
        \item Social Network : facebook ou autre selon les possibilités.
    \end{itemize}
    
\begin{figure}[!h]
\begin{tikzpicture}[scale=1]
 \begin{umlseqdiag}
  \umlactor{User}
  \umlobject{SoftCard}
  \umlobject{SmartCard}
  \umlobject{BrowserProxy}
  \umlobject{SocialNetwork}
  \begin{umlcall}[op=DémarrageBP-connexionSN, type=synchron, return=Dechiffrement-Affichage vue]{User}{BrowserProxy}
   \begin{umlcall}[op=Authentication carte ?,type=synchron,return=Demande d'auth du Proxy et valide Auth Card]{BrowserProxy}{SoftCard}
      \begin{umlcall}[op=PIN ?,type=synchron,return=PIN !]{SoftCard}{User}
      \end{umlcall}
      \begin{umlcall}[op=PIN, type=synchron,return=OK]{SoftCard}{SmartCard}
      \end{umlcall}
    \end{umlcall}
    \begin{umlcall}[op=Authentification-clef de session-valide Auth Card, type=synchron,return=login et mot de passe]{BrowserProxy}{SoftCard}
    \end{umlcall}
    \begin{umlcall}[op=GET, type=synchron,return=données chiffrées]{BrowserProxy}{SocialNetwork}
    \end{umlcall}
    \begin{umlcall}[op=demande clef privée, type=synchron,return=clef privée]{BrowserProxy}{SoftCard}
    \end{umlcall}
  \end{umlcall}
  \end{umlseqdiag}
\end{tikzpicture}
\end{figure}

\subsubsection*{Travail pour la prochaine réunion (04/12/12)}
Étudier le schéma et trouver des questions à poser aux profs pour mardi.

\subsubsection*{Travail pour la réunion du (14/12/12)}
\begin{itemize}
    \item STB Cartes à puce Giovanni et Romain.
    \item Yicheng et Emmanuel Technologie javacard et PC/SC (DAL)
    \item STB SSN Baptiste et Florian
    \item Recherche sur protocole par rapport au Social Network, Zakaria et Maxence.
\end{itemize}

\end{document}
