\documentclass[a4paper,10pt]{article}
\usepackage[utf8]{inputenc}
\usepackage{graphicx, amssymb, xcolor, listings}
\usepackage[colorlinks=true,linkcolor=black,urlcolor=blue]{hyperref}
\usepackage{tikz}
\usepackage[top=2.6cm,bottom=2.6cm,left=2.6cm,right=2.6cm]{geometry}
\usepackage[french]{babel}

\author{Florian \textsc{Guilbert}, Emmanuel \textsc{Mocquet}}
\title{Compte-rendu de réunion, 2}
\date{04 décembre 2012}

\begin{document}

\maketitle

\subsubsection*{Participants : }
\begin{itemize}
    \item Magali \textsc{Bardet} (cliente);
    \item Zakaria \textsc{Addi};
    \item Baptiste \textsc{Dolbeau};
    \item Florian \textsc{Guilbert};
    \item Giovanni \textsc{Huet};
    \item Emmanuel \textsc{Mocquet};
    \item Romain \textsc{Pignard}.
\end{itemize}

\subsubsection*{Absents : }
\begin{itemize}
    \item Ayoub \textsc{Otmani} (client);
    \item Yicheng \textsc{Gao};
    \item Maxence  \textsc{Péchoux}.
\end{itemize}

\subsection*{Ordre du jour : Expression des besoins}

\subsubsection*{SmartCard}

\paragraph{Objectifs} 
\begin{itemize}
    \item Application fonctionnelle (trivial : Application de dialogue entre 
            lecteur et carte);
    \item Application de vérification de code PIN;
    \item Application d'authentification
    \item Stockage et "utilisation" de certificats
    \item Stockage du login/mot de passe Facebook
    \item Documentation détaillée du fonctionnement des applications et 
        de l'implantation (comme un compte-rendu de TP)
    \item Documentation, comment faire (comment l'avons-nous développé) ?
\end{itemize}
         
Conseillé de réutiliser tout ce qui existe sur carte (fonctions crypto, ...) 
         
                
\subsubsection*{Secure Social Network}

    En attente d'une réunion avec le client.

\subsection*{Prochaine réunion}
    14/12/12 $\rightarrow$ cf CR1

\end{document}
