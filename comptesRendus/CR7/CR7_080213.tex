\documentclass[a4paper,10pt]{article}
\usepackage[utf8]{inputenc}
\usepackage{graphicx, amssymb, xcolor, listings}
\usepackage[colorlinks=true,linkcolor=black,urlcolor=blue]{hyperref}
\usepackage{tikz}
\usepackage[top=2.6cm,bottom=2.6cm,left=2.6cm,right=2.6cm]{geometry}
\usepackage[french]{babel}

\author{Florian \textsc{Guilbert}}
\title{Compte-rendu de réunion, 7}
\date{08 février 2013}

\begin{document}

\maketitle

\subsubsection*{Participants : }
\begin{itemize}
    \item Magali \textsc{Bardet} (cliente);
    \item Zakaria \textsc{Addi};
    \item Baptiste \textsc{Dolbeau};
    \item Yicheng \textsc{Gao};
    \item Florian \textsc{Guilbert};
    \item Giovanni \textsc{Huet};
    \item Emmanuel \textsc{Mocquet};
    \item Maxence  \textsc{Péchoux}.
    \item Romain \textsc{Pignard}.
\end{itemize}

\subsubsection*{Absents : }
Pas d'absents.

\subsection*{Ordre du jour : Livraison de la seconde itération SC}

\paragraph{1. Remarque sur le STB}
\begin{itemize}
\item F-FI-30 : mettre la priorité de cette exigence en indispensable.
\end{itemize}

\paragraph{2. Remarque sur le tunnel entre lecteur et carte}

\begin{itemize}
\item Configurer l'établissement du tunnel avec un aléa pour authentifier les deux
parties.
\item Le code mac doit prendre tout le paquet (IV compris).
\item Le code PIN doit être redemandé au bout d'un certain nombre d'opérations
\end{itemize}

\subsection*{Prochaine réunion}
    Vendredi 15 février
    
\end{document}
