\documentclass[a4paper,10pt]{article}
\usepackage[utf8]{inputenc}
\usepackage{graphicx, amssymb, xcolor, listings}
\usepackage[colorlinks=true,linkcolor=black,urlcolor=blue]{hyperref}
\usepackage{tikz}
\usepackage[top=2.6cm,bottom=2.6cm,left=2.6cm,right=2.6cm]{geometry}
\usepackage[french]{babel}

\author{Florian \textsc{Guilbert}}
\title{Compte-rendu de réunion, 6}
\date{06 février 2013}

\begin{document}

\maketitle

\subsubsection*{Participants : }
\begin{itemize}
    \item Magali \textsc{Bardet} (cliente);
    \item Zakaria \textsc{Addi};
    \item Baptiste \textsc{Dolbeau};
    \item Yicheng \textsc{Gao};
    \item Florian \textsc{Guilbert};
    \item Giovanni \textsc{Huet};
    \item Emmanuel \textsc{Mocquet};
    \item Maxence  \textsc{Péchoux}.
    \item Romain \textsc{Pignard}.
\end{itemize}

\subsubsection*{Absents : }
Pas d'absents.

\subsection*{Ordre du jour : discussion à propos de la spécification
technique des besoins}

\paragraph{1. Remarque sur le STB}

\begin{itemize}
\item Fusion de deux sous-projets $\rightarrow$ composé de deux sous-projets
\item Il faut définir "sécurisé" dans la terminologie (chiffrement, intégrité)
Le client exige que "sécurisé" comprenne l'intégrité.
\item Donc : utilisation d'un code MAC pour le tunnel entre la carte et le PC.
Possibilité de tronquer le code MAC si nécessaire (mais justifier)
\item Définir "un générateur de nombre aléatoire"
\item La priorité va à la vérification de l'intégrité qu'à la vérification de 
l'aléatoire.
\item "La carte doit être capable de générer un nombre aléatoire" $\rightarrow$ 
"ON" doit être capable d'utiliser le générateur.
\item PRN $\rightarrow$ à remplacer par RN ("fourni par la carte")
\item nouveau use case : administration de la carte en permettant à l'utilisateur
de fournir/changer ses identifiants
\item Déchiffrement : flot d'exception : uniformiser les messages d'erreurs
\item chiffrement : avec clef secrète à la place de "privée"
\item vérification de données par softcard pas par smartcard
\item Livraison le 1er mars.
\end{itemize}

\subsection*{Prochaine réunion}
    Vendredi 8 février à 11h.    
    
\end{document}
