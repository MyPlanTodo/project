%!TEX TS-program = xelatex
%!TEX encoding = UTF-8 Unicode
\documentclass[a4paper,11pt,french]{article}

\usepackage[latin1]{inputenc}
\usepackage[french]{babel}
\usepackage{chngpage}
\usepackage[colorlinks=true, linkcolor=black, urlcolor=blue]{hyperref}
\usepackage{fontspec,xltxtra,xunicode,color}
\usepackage{tabularx}
\usepackage[table]{xcolor}
\usepackage{fancyhdr}
\usepackage{longtable}
\usepackage{lastpage} %pour compter le nombre de pages

\definecolor{gris}{rgb}{0.95, 0.95, 0.95}

\hypersetup{breaklinks=true}


\addtolength{\hoffset}{-2cm}
\addtolength{\textwidth}{4cm}
\addtolength{\topmargin}{-2cm}
\addtolength{\textheight}{4cm}
\addtolength{\headsep}{0.8cm} 
\addtolength{\footskip}{-0.3cm}

%les structures de gestion de projet
\usepackage{res/structures} 


\def\projectName{Service d'authentification unique et décentralisé}
\def\docType{Spécification Technique de Besoin}
\def\version{1.0}
\def\author{Lynda \textsc{Laceb}, Safae \textsc{Rebani}}
\def\dateStb{\today}
\def\checked{Florian \textsc{Guilbert}}
\def\approved{Florent \textsc{Nicart}}



\begin{document}
\makeFirstPage
\clearpage
\vspace*{1cm}
%Tableau de mises à jour
\begin{center}
\textbf{\huge{MISES À JOUR}}\\
\vspace*{3cm}
	\begin{tabularx}{16cm}{|c|c|X|}
	\hline
	\bfseries{Version} & \bfseries{Date} & \bfseries{Modifications réalisées}\\
	\hline
	0.1 & 30/11/2011 & Création\\
	\hline
	1.0 &  15/01/2012 & Relu par Florian \textsc{Guilbert} \\
	\hline
	1.1 &  11/03/2012 & Modification des exigences fonctionnelles (Ajout
d'exigences fonctionnelles pour le protocole)\\
	\hline
	1.2 &  29/03/2012 & Modification des exigences fonctionnelles (Log de
connexion) \\
	\hline
	\end{tabularx}
\end{center}

\clearpage
\tableofcontents
\clearpage


\section{Objet}
\renewcommand\labelitemi{\textbullet} %style des puces
\renewcommand\labelitemii{$\circ$} %style des puces 2e niveau
Proposer aux utilisateurs une application web permettant une authentification
unique de manière déportée sur internet en s'appuyant sur la mise en
place d'un protocole sécurisé.

Notre objectif est avant tout de réaliser une étude de faisabilité d'un
protocole garantissant la sécurité de ce système. S'il s'avère qu'un tel
protocole peut être entièrement sécurisé, nous devrons développé en guise de
démonstration un service d'authenficiation unique, décentralisé.

Le système doit s'appuyer sur des technologies libres.

Une API sera développée pour permettre aux sites d'implanter notre protocole. 
Cette API sera développée en PHP. L'implantation de cette API en JSP est 
optionnelle.

La possibilité d'installer un paquetage permettant d'héberger son propre SAUD 
ainsi que de proposer un panneau d'administration est optionnelle.

Il sera fourni au client une documentation expliquant le procole 
d'authentification.

\paragraph{Note importante}
Tout au long de ce document ainsi que pour les autres documents produit, nous
insistons sur la production d'un logiciel (ici une application web) proposant
une authenficiation unique décentralisée et ce de manière très sécurisé. En
effet si le site ne dialogue pas avec l'utilisateur ou bien les autres sites de
façon parfaitement sécurisé alors il n'a aucun intérêt. 

De ce fait pour la suite, nous admettons qu'un tel protocole sécurisé est
réalisable et que par conséquent nous livrerons une application web permettant
d'illustrer ce protocole. 

Dans le cas contraire il n'y aura aucun produit à l'issue de ce projet,
seulement un document expliquant en quoi le protocole est irréalisable.

\section{Documents applicables et de référence}
\begin{itemize}
\item Un manuel d'utilisation décrivant les différents possibilités d'
utilisations du logiciel;
\item Une charte de sécurité;
\item RFC du protocole.
\end{itemize}


\section{Terminologie et sigles utilisés}
\begin{description}
	\item[SAUD :] Site d'Authentification Unique et Décentralisée;
	\item[AE :] Authentification Externe, module permettant à des sites quelconques de proposer un
    service d'authenficiation unique et décentralisée;
	\item[SE :] Service externe, n'importe quel autre site qu'un SAUD;
	\item[Utilisateur :] Personne utilisant le service d'authentification unique, décentralisée 
	et donc inscrit au sein du SAUD;
	\item[Visiteur :] Personnne non encore inscrit sur le SAUD;
	\item[Administrateur :] Utilisateur spécial qui gère un SAUD;
	\item[Identifiant :] Mot permettant d'identifier l'utilisateur au sein du SAUD;
	\item[Adresse SAUD :] 	Correspond à la concaténation de l'identifiant, d'un dièze et du nom du SAUD
	(\emph{i.e.} \verb+[loginInterne:]loginExterne#saud.net+, avec
loginInterne le login du site quelconque
et loginExterne le login dans le SAUD);
	\item[API :] (Application Interface Programming), ensemble de fonctions
et de modules permettant de réaliser un but précis;
	\item[RFC :] (Requests for comments) est un document officiel décrivant les aspects techniques d'Internet
	ou de différent matériel informatique (routeurs, serveur DHCP...);
	\item[PHP et JSP :] (PHP Hypertext Preprocessor et Java Server Pages) sont des langages de programmation web coté serveur;
\end{description}


\section{Exigences fonctionnelles}

\subsection{Présentation de la mission du produit logiciel}
\newcounter{FGcount}
\begin{tabularx}{16cm}{|c|X|l|c|}
\hline
\rowcolor{blue}~{\color{white}\bfseries{Reference}}&~{\color{white}\bfseries{Fonctionalité Globale}}&~{\color{white}\bfseries{Acteur}}&~{\color{white}\bfseries{Priorité}}\\
\hline
\addtocounter{FGcount}{10}
F-Gl-\arabic{FGcount} & Gestion du compte & Utilisateur & \cellcolor{green!50}Indispensable \\
\hline
\addtocounter{FGcount}{10}
F-Gl-\arabic{FGcount} & Connexion sur un site quelconque en utilisant
l'authentification unique & Utilisateur & \cellcolor{green!50}Indispensable \\
\hline
\addtocounter{FGcount}{10}
F-Gl-\arabic{FGcount} & Installation d'un paquetage personnel pour créer son propre SAUD & Utilisateur & \cellcolor{blue!50}Secondaire\\
\hline
\addtocounter{FGcount}{10}
F-Gl-\arabic{FGcount} & Administrer le SAUD & Administrateur & \cellcolor{blue!50}Secondaire\\
\hline
\end{tabularx}

\subsection{Gestion du compte}
La gestion du compte se décline en huit parties : 
\begin{itemize}
    \item création d'un compte utilisateur;
    \item connexion au SAUD;
    \item déconnexion du SAUD;
    \item suppression du compte utilisateur;
    \item modifier l'adresse mail du compte utilisateur;
    \item modifier le mot de passe;
    \item modifier la question secrête;
    \item modification de mot de passe.
\end{itemize}
Il y a deux comptes différents : utilisateur et administrateur.\\
La création du compte administrateur se fait à l'installation du SAUD sur le serveur.
N'importe quel visiteur peut créer un compte sur un SAUD.\\

\fiche
	{Création d'un compte utilisateur}
	{Utilisateur}
	{L'utilisateur crée un compte en remplissant un formulaire}
	{L'utilisateur n'est pas déjà authentifié}
	{L'utilisateur souhaite se créer un compte sur le SAUD}
	{La création du compte s'est effectuée avec succès}
	{\begin{itemize}
	    \item[]
	  \item[1.] Clique sur le lien/bouton "Créer un compte"
	  \item[3.] Entre son pseudo, son adresse mail et son mot de passe.
	  \item[4.] Valide la création du compte en appuyant sur "Créer"
	  \item[7.] Confirme le mail reçu en suivant le lien présent dans 
	  le corps du texte.
	\end{itemize}
	}
	{\begin{itemize}
        \item[]
		\item[2.] Envoie le formulaire de création de compte.
		\item[5.] Vérifie les données transmises par l'utilisateur : 
        l'identifiant doit être unique, l'adresse mail valide et le mot
        de passe doit être conforme à la politique de mot de passe
        du site.
        \item [6.] Envoie un mail de confirmation à l'utilisateur.
        \item [8.] Insère les données de l'utilisateur dans la base de données du site.
        \item [9.] Archive l'identifiant.
	\end{itemize}
	}
	{}
\flots
    {\begin{itemize}
    \item[]
    \item[5.] Tant que le pseudo n'est pas unique, l'utilisateur est invité à en entrer un autre;
    \item[5.] Si un des champs obligatoires n'est pas rempli, l'utilisateur est invité à les remplir;
    \end{itemize}
    }
	{\begin{itemize}
    \item[]
    \item[8.] Si le mail de confirmation n'est pas renvoyé avant la fin de la session du site,
    la création du compte est avortée.
    \item[8.] Si le mail est renvoyé après la fin de la session du visiteur,
        il est invité à réitérer la création de son compte.
    \end{itemize}
    }    
\\*

\fiche
	{Authenfication au SAUD}
	{Utilisateur}
	{L'utilisateur s'authentifie au SAUD}
	{L'utilisateur n'est pas déjà authentifié}
	{L'utilisateur souhaite s'authentifier au SAUD}
	{L'authentification s'est effectuée avec succès}
	{\begin{itemize}
	    \item[]
		\item[1.] Remplit le formulaire d'authentification, entre son pseudo,
		son mot de passe et valide.
		\item[3.] Renseigne l'adresse mail, ainsi que la réponse à la question secrète.
	\end{itemize}
	}
	{\begin{itemize}
        \item[]
		\item[2.] Vérifie les données entrées par l'utilisateur, 
 si les données sont correctes mais que l'utilisateur se connecte d'un endroit
inhabituel, le système envoi un formulaire en demandant de remplir le mail et de répondre à la question secrète;
	\end{itemize}
	}
	{}
\flots
    {\begin{itemize}
    \item[]
    \item[2.] Tant que le mot de passe ou l'identifiant ne sont pas correctes, l'utilisateur 
    est invité à les rentrer de nouveau.
    \end{itemize}
    }
    {}    
\\*

\fiche
	{Déconnexion de l'utilisateur}
	{Utilisateur}
	{L'utilisateur se déconnecte du SAUD}
	{L'utilisateur est authentifié}
	{L'utilisateur souhaite se déconnecter du SAUD}
	{La déconnexion s'est effectuée avec succès}
	{\begin{itemize}
	    \item[]
		\item[1.] Clique sur le lien/bouton déconnexion.
	\end{itemize}
	}
	{\begin{itemize}
        \item[]
		\item[2.] Vide la session de l'utilisateur pour le déconnecter et 
		supprime les éventuels cookies.
	\end{itemize}
	}
	{}
\flots
    {}
	{}    
\\*

\fiche
	{Suppression d'un compte utilisateur}
	{Utilisateur}
	{L'utilisateur supprime son compte}
	{L'utilisateur possède un compte et est authentifié}
	{L'utilisateur souhaite supprimer son compte}
	{La suppression du compte s'est effectuée avec succès}
	{\begin{itemize}
		\item[1.] Clique sur le lien/bouton "Supprimer son compte"
        \item[3.] Entre son mot de passe et valide la suppression
	\end{itemize}
	}
	{\begin{itemize}
        \item[]
		\item[2.] Envoie un le formulaire de suppression de compte.
		\item[4.] Vérifie le mot de passe de l'utilisateur.
        \item [5.] Envoie un mail à l'utilisateur pour lui signifier la
suppression du compte
	\item[6.] Supprime les données présentes de l'utilisateur dans la
base
         de données.
	\end{itemize}
	}
	{}
\flots
    {\begin{itemize}
    \item[]
    \item[4.] Tant que le mot de passe est invalide, l'utilisateur doit le
retaper.
    \end{itemize}
    }
	{}
\\*

\fiche
	{Modification de l'adresse mail}
	{Utilisateur}
	{L'utilisateur modifie son adresse mail}
	{L'utilisateur possède un compte et est authentifié}
	{L'utilisateur souhaite modifier son adresse mail}
	{La modification de l'adresse mail s'est effectuée avec succès}
	{\begin{itemize}
		\item[1.] Clique sur le lien/bouton "Modifier son adresse mail"
        \item[3.] Entre sa nouvelle adresse ainsi que la réponse à sa question
et valide
	\end{itemize}
	}
	{\begin{itemize}
        \item[]
		\item[2.] Envoie un le formulaire de suppression de compte
		\item[4.] Modifie dans la base la nouvelle adresse de l'utilisateur
	\end{itemize}
	}
	{}
\flots
    {\begin{itemize}
    \item[]
    \item[5.] Tant que l'adresse mail n'est pas une adresse valide et que la
réponse à la question est incorrecte.
    \end{itemize}
    }
	{}
\\*

\fiche
	{Modification du mot de passe}
	{Utilisateur}
	{L'utilisateur modifie son mot de passe}
	{L'utilisateur possède un compte et est authentifié}
	{L'utilisateur souhaite modifier son mot de passe}
	{La modification de le mot de passe s'est effectuée avec succès}
	{\begin{itemize}
	      \item[]
		\item[1.] Clique sur le lien/bouton "Modifier son mot de passe"
        \item[3.] Entre son mot de passe actuel, ainsi que son nouveau mot de 
        passe deux fois et valide le formulaire
	\end{itemize}
	}
	{\begin{itemize}
        \item[]
		\item[2.] Envoie un le formulaire de modification de mot de passe.
		\item[4.] Vérifie le mot de passe de l'utilisateur
        \item [5.] modifie le mot de passe de l'utilisateur
	\end{itemize}
	}
	{}
\flots
    {\begin{itemize}
    \item[]
    \item[5.] Tant que le mot de passe est invalide, l'utilisateur doit le retaper.
    \end{itemize}
    }
	{}
\\*

\fiche
	{Modification de la question secrête}
	{Utilisateur}
	{L'utilisateur modifie sa question secrête}
	{L'utilisateur possède un compte et est authentifié}
	{L'utilisateur souhaite modifier sa question secrête}
	{La modification de la question secrête s'est effectuée avec succès}
	{\begin{itemize}
		\item[1.] Clique sur le lien/bouton "Modifier ma question secrête"
        \item[3.] Entre l'intitulé de question ainsi que sa réponse et valide
	\end{itemize}
	}
	{\begin{itemize}
        \item[]
		\item[2.] Envoie un le formulaire de modification de la question
		secrête
		\item[4.] Entre en base de données l'intitulé et la réponse de la 
		question s'ils sont conforme à la syntaxe attendue
	\end{itemize}
	}
	{}
\flots
    {\begin{itemize}
    \item[]
    \item[5.] Tant que l'intitulé et la réponse à la  question secrête ne 
    correspondent pas à la syntaxe attendue, l'utilisateur est invité à les 
    retaper 
    \end{itemize}
    }
	{}
\\*

\fiche
	{Modifie son mot de passe}
	{Utilisateur}
	{L'utilisateur souhaite modifié son mot de passe}
	{L'utilisateur possède un compte et n'est pas authentifié}
	{L'utilisateur souhaite modifié son mot de passe}
	{L'utilisateur a modifié son mot de passe}
	{\begin{itemize}
	    \item[]
		\item[1.] Clique sur le lien/bouton "Modification de mot de
passe"
        \item[3.] Entre son adresse mail et répond à sa question
        \item[5.] Suit le lien présent dans le mail
        \item[7.] Entre un nouveau mot de passe.
	\end{itemize}
	}
	{\begin{itemize}
        \item[]
		\item[2.] Envoie un le formulaire d'oublie de mot de passe
		\item[4.] Vérifie les données entrées et envoie un mail à l'adresse de 
		l'utilisateur
		\item[6.] Redirige l'utilisateur vers un formulaire de mot de passe
		\item[8.] Vérifie le mot de passe et entre celui-ci dans la base de 
		données.
	\end{itemize}
	}
	{}
\flots
    {\begin{itemize}
    \item[]
    \item[4.] Tant que la réponse est incorrecte, l'utilisateur est invité à la
    retaper
    \item[8.] Tant que le mot de passe ne correspond pas à la syntaxe attendue,
     l'utilisateur doit en retaper un nouveau.
    \end{itemize}
    }
	{\begin{itemize}
	\item[]
	\item[6.] Si le lien du mail n'est pas suivie avant la fin de la session, 
	l'utilisateur devra réitérer la procédure
	\item[6.] Si le lien est suivi après la fin de la session, l'utilisateur est
	redirigé vers le formulaire d'oublie de mot de passe.
	\end{itemize}
	}

\subsection{Authentification sur un site quelconque par le biais du SAUD}
Un utilisateur enregistré sur un SAUD peut créer un compte sur un site quelconque
en spécifiant l'utilisation de l'authentification externe.

Une fois le compte créer sur ce site, il peut désormais s'y connecter en s'authentifiant
par l'intermédiaire du SAUD. Un utilisateur peut créer plusieurs compte avec authenficiation
externe sur un même site.\\

\fiche
	{Création d'un compte sur un site quelconque avec AE}
	{Utilisateur}
	{L'utilisateur crée un compte en remplissant un formulaire et en cochant la mention
    "Authentification externe"}
	{L'utilisateur dispose d'un compte sur un SAUD et le site quelconque doit proposer le service d'authentification externe}
	{}
	{La création du compte s'est effectuée avec succès}
	{\begin{itemize}
        \item[]
		\item[1.] Clique sur le lien/bouton "Créer un compte"
        \item[3.] Remplit le formulaire du site hormis le champs \emph{mot de
passe} et donne pour identifiant, son adresse SAUD.
        \item[4.] Valide l'envoi formulaire.
	\end{itemize}
	}
	{\begin{itemize}
        \item[]
		\item[2.] Envoie le formulaire de création de compte.
		\item[5.] Vérifie les données du formulaire, la syntaxe de
l'identifiant doit correspondre à un identifiant SAUD
		\item[6.] Envoie une requête au SAUD pour savoir si l'utilisateur est authentifié.
        \item[7.] Créer le compte.
	\end{itemize}
	}
	{}
\flots
    {\begin{itemize}
    \item[]
    \item[5.] Si l'identifiant n'est pas conforme à la syntaxe attendue
    l'utilisateur est invité à le réentrer.
    \item[5.] S'il y a plusieurs idenfiant interne pour un identifiant externe, il y a ambiguïté,
    l'utilisateur est donc invité à préciser l'identifiant interne.
    \item[6.] Si la reponse du SAUD est négative, le site redirige l'utilisateur sur le SAUD pour 
    qu'il s'authentifie.
    \item[6.] Si l'adresse dans l'identifiant ne pointe pas vers une adresse
de SAUD valide.
    \end{itemize}
    }
    {}
\\*
\\*

\fiche
	{Connexion sur un site quelconque avec AE}
	{Utilisateur}
	{L'utilisateur se connecte au site quelconque par le biais du SAUD}
	{L'utilisateur dispose d'un compte sur un SAUD et sur le site quelconque et le site quelconque doit proposer le service d'authentification externe}
	{}
	{La connexion s'est effectuée avec succès}
	{\begin{itemize}
        \item[]
		\item[1.] Rempli uniquement le champs \emph{identifiant} avec
        son adresse SAUD et valide la connexion en laissant vide
        le champs \emph{mot de passe}.
	\end{itemize}
	}
	{\begin{itemize}
        \item[]
		\item[2.] Envoie une requête au SAUD pour savoir si l'utilisateur est authentifié.
        \item[3.] Valide la connexion de l'utilisateur.
	\end{itemize}
	}
	{}
\flots
    {}
	{\begin{itemize}
    \item[]
    \item[2.] Si la reponse du SAUD est négative : deux cas possibles : 
	\begin{itemize}
	\item[-]le login existe, donc le site redirige l'utilisateur sur le SAUD pour qu'il s'authentifie;
	\item[-]le login n'existe pas, une erreur est renvoyé au site.
	\end{itemize}
    \end{itemize}
    }

\subsection{Administration du SAUD}
À l'installation du SAUD, un  compte administrateur est créé. Ce compte permet 
de supprimer les utilisateurs.

\fiche
	{Administration du SAUD}
	{Administrateur}
	{L'administrateur peut supprimer les comptes des utilisateurs du SAUD}
	{Il y a un compte administrateur sur ce SAUD}
	{}
	{L'administration s'est bien passée}
	{\begin{itemize}
        \item[]
		\item[1.] L'administrateur clique sur "Panneau d'administration"
		\item[3.] L'administrateur supprime un ou plusieurs utilisateurs et valide
		l'envoi du formulaire
	\end{itemize}
	}
	{\begin{itemize}
        \item[]
		\item[2.] Envoie un formulaire contenant l'intégralité des utilisateurs
        \item[4.] Supprime les utilisateurs choisie par l'administrateur
	\end{itemize}
	}
	{}
\flots
    {}
	{}

\subsection{Installation d'un SAUD personnel}
Un utilisateur peut choisir d'utiliser un SAUD déjà existant ou bien
de d'installer le sien. Pour installer son SAUD, il doit télécharger
le paquetage requis et l'exécuter sur sa machine.\\

\fiche
	{Installation du paquetage d'installation du SAUD}
	{Utilisateur}
	{L'utilisateur installe son SAUD personnel sur sa machine}
	{L'utilisateur doit avoir un environnement adéquat pour installer le SAUD}
	{Téléchargement du paquetage}
	{L'installation s'est terminée}
	{\begin{itemize}
        \item[]
		\item[1.] L'utilisateur télécharge et installe le paquetage;
		\item[4.] L'utilisateur entre le mot de passe administrateur.
	\end{itemize}
	}
	{\begin{itemize}
        \item[]
		\item[2.] Installation du paquetage;
		\item[3.] Le programme d'installation invite l'utilisateur à entrer
		un mot de passe pour le compte administrateur conformément à la politique 
		des mots de passe du SAUD;
		\item[5.] Création du compte administrateur.
	\end{itemize}
	}
	{}
\flots
    {\begin{itemize}
    \item[]
    \item[2.] Si les logiciels apache, mysql et php ne sont pas installés, le 
    système demande à l'utilisateur de les installer;
    \item[3.] Tant que le mot de passe saisie par l'utilisateur est trop faible
    il doit en entrer un de nouveau.
    \end{itemize}
    }
	{}
	
\pagebreak
\subsection{Exigences fonctionnelles détaillées}

\subsubsection{Le protocole}

\newcounter{FNcount}

\begin{longtable}{|p{2cm}|p{10cm}|p{2.5cm}|}

% Header for the first page of the table
\multicolumn{1}{|>{\color{white}\columncolor{blue}}l|}{\bfseries{Reference}} & \multicolumn{1}{|>{\color{white}\columncolor{blue}}l|}{\bfseries{Fonctionalité}} & \multicolumn{1}{|>{\color{white}\columncolor{blue}}l|}{\bfseries{Priorité}}
\endfirsthead
% Header for next pages of the table
\multicolumn{1}{|>{\color{white}\columncolor{blue}}l|}{\bfseries{Reference}} & \multicolumn{1}{|>{\color{white}\columncolor{blue}}l|}{\bfseries{Fonctionalité}} & \multicolumn{1}{|>{\color{white}\columncolor{blue}}l|}{\bfseries{Priorité}}
\endhead

%This is the footer for all pages except the last page of the table...
\endfoot
%This is the footer for the last page of the table...
\endlastfoot

\hline
\addtocounter{FNcount}{10}
F-FN-\arabic{FNcount} & L'utilisateur ne donne son mot de passe qu'au SAUD &
\cellcolor{green!50}Indispensable \\
\hline
\addtocounter{FNcount}{10}
F-FN-\arabic{FNcount} & L'authentification sur un service externe ne peut se
faire qu'à partir d'un emplacement donnée &
\cellcolor{blue!50}Secondaire \\
\hline
\addtocounter{FNcount}{10}
F-FN-\arabic{FNcount} & Plusieurs authenfication simultanées sont impossible, 
le SAUD déconnecte la session en cours en cas d'une autre demande
d'authenfication &
\cellcolor{green!50}Indispensable \\
\hline
\addtocounter{FNcount}{10}
F-FN-\arabic{FNcount} & À chaque tentative de connexion d'un client sur un
service externe, celui-ci vérifie l'authenticité du SAUD associé au client &
\cellcolor{green!50}Indispensable \\
\hline
\addtocounter{FNcount}{10}
F-FN-\arabic{FNcount} & Les échanges et dialogues entre service externe, SAUD
et utilisateur sont sécurisés &
\cellcolor{green!50}Indispensable \\
\hline
\end{longtable}

\subsubsection{Le SAUD}

\begin{longtable}{|p{2cm}|p{10cm}|p{2.5cm}|}

% Header for the first page of the table
\multicolumn{1}{|>{\color{white}\columncolor{blue}}l|}{\bfseries{Reference}} &
\multicolumn{1}{|>{\color{white}\columncolor{blue}}l|}{\bfseries{Fonctionalité}}
& \multicolumn{1}{|>{\color{white}\columncolor{blue}}l|}{\bfseries{Priorité}}
\endfirsthead
% Header for next pages of the table
\multicolumn{1}{|>{\color{white}\columncolor{blue}}l|}{\bfseries{Reference}} &
\multicolumn{1}{|>{\color{white}\columncolor{blue}}l|}{\bfseries{Fonctionalité}}
& \multicolumn{1}{|>{\color{white}\columncolor{blue}}l|}{\bfseries{Priorité}}
\endhead

%This is the footer for all pages except the last page of the table...
\endfoot
%This is the footer for the last page of the table...
\endlastfoot

\hline
\addtocounter{FNcount}{10}
F-FN-\arabic{FNcount} & Le SAUD est accessible au travers d'une page web &
\cellcolor{green!50}Indispensable \\
\hline
\addtocounter{FNcount}{10}
F-FN-\arabic{FNcount} & La page d'accueil du site propose un formulaire
d'authentification
ainsi que le lien/bouton "Creer un compte"  & \cellcolor{green!50}Indispensable
\\
\hline
\addtocounter{FNcount}{10}
F-FN-\arabic{FNcount} & Un visiteur peut créer un compte utilisateur &
\cellcolor{green!50}Indispensable \\
\hline
\addtocounter{FNcount}{10}
F-FN-\arabic{FNcount} & Un login unique, un mot de passe, une question secrête
ainsi que sa réponse et une adresse mail valide est essentiel pour créer un
compte utilisateur
 & \cellcolor{green!50}Indispensable \\
\hline
\addtocounter{FNcount}{10}
F-FN-\arabic{FNcount} & Quand un visiteur créé un compte d'utilisateur, un mail
lui est envoyé
et sans accusé de reception, le compte n'est pas créé &
\cellcolor{green!50}Indispensable \\
\hline
\addtocounter{FNcount}{10}
F-FN-\arabic{FNcount} & Si un utilisateur veut suprimer son compte, il doit
entrer son mot de passe
 & \cellcolor{blue!50}Secondaire \\
\hline
\addtocounter{FNcount}{10}
F-FN-\arabic{FNcount} & Un utilisateur peut se déconnecter du SAUD en cliquant
sur un lien & \cellcolor{red!50}Important \\
\hline
\addtocounter{FNcount}{10}
F-FN-\arabic{FNcount} & Un utilisateur peut modifier son mot de passe &
\cellcolor{blue!50}Secondaire \\
\hline
\addtocounter{FNcount}{10}
F-FN-\arabic{FNcount} & Un utilisateur peut changer sa question secrête
(intitulé et réponse) & \cellcolor{blue!50}Secondaire\\
\hline
\addtocounter{FNcount}{10}
F-FN-\arabic{FNcount} & Un utilisateur peut modifier son adresse mail &
\cellcolor{green!50}Indispensable \\
\hline
\addtocounter{FNcount}{10}
F-FN-\arabic{FNcount} & Un utilisateur peut se connecter à un site quelconque
 en donnant son adresse SAUD dans le champs "identifiant" du site &
\cellcolor{green!50}Indispensable \\
\hline
\addtocounter{FNcount}{10}
F-FN-\arabic{FNcount} & Un utilisateur peut créer un compte sur un site
quelconque
 en utilisant l'authentification externe & \cellcolor{green!50}Indispensable \\
\hline
\addtocounter{FNcount}{10}
F-FN-\arabic{FNcount} & Lorsqu'un utilisateur se connecte depuis une autre
adresse IP pour la première fois, il doit répondre à sa question secrête et 
rentrer son adresse mail & \cellcolor{green!50}Indispensable \\
\hline
\addtocounter{FNcount}{10}
F-FN-\arabic{FNcount} & L'utilisateur peut modifier son mot de passe
peut entrer un nouveau en suivant la procédure adéquate &
\cellcolor{green!50}Indispensable \\
\hline
\addtocounter{FNcount}{10}
F-FN-\arabic{FNcount} & L'administrateur peut supprimer le compte de n'importe
quel utilisateur & \cellcolor{blue!50}Secondaire \\
\hline
\addtocounter{FNcount}{10}
F-FN-\arabic{FNcount} & L'administrateur n'aura jamais accès aux mot de passes
des utilisateurs  & \cellcolor{blue!50}Secondaire \\
\hline
\addtocounter{FNcount}{10}
F-FN-\arabic{FNcount} & Il est possible de télécharger un paquetage
d'installation d'un SAUD peronnel & \cellcolor{blue!50}Secondaire \\
\hline
\addtocounter{FNcount}{10}
F-FN-\arabic{FNcount} & Log de la dernière connexion à chaque nouvelle connexion
& \cellcolor{blue!50}Secondaire \\
\hline
\end{longtable}
\pagebreak

\section{Exigences opérationnelles}

\newcounter{FOcount}

\begin{longtable}{|p{2cm}|p{10cm}|p{2.5cm}|}

% Header for the first page of the table
\multicolumn{1}{|>{\color{white}\columncolor{blue}}l|}{\bfseries{Reference}} & \multicolumn{1}{|>{\color{white}\columncolor{blue}}l|}{\bfseries{Fonctionalité}} & \multicolumn{1}{|>{\color{white}\columncolor{blue}}l|}{\bfseries{Priorité}}
\endfirsthead
% Header for next pages of the table
\multicolumn{1}{|>{\color{white}\columncolor{blue}}l|}{\bfseries{Reference}} & \multicolumn{1}{|>{\color{white}\columncolor{blue}}l|}{\bfseries{Fonctionalité}} & \multicolumn{1}{|>{\color{white}\columncolor{blue}}l|}{\bfseries{Priorité}}
\endhead

%This is the footer for all pages except the last page of the table...
\endfoot
%This is the footer for the last page of the table...
\endlastfoot

\hline
\addtocounter{FOcount}{10}
F-FO-\arabic{FOcount} & La délai maximum de réponse du SAUD est de deux secondes & \cellcolor{red!50}Important \\
\hline
\addtocounter{FOcount}{10}
F-FO-\arabic{FOcount} & Le SAUD peut supporter plus de 1000 utilisateurs
 & \cellcolor{red!50}Important \\
\hline
\addtocounter{FOcount}{10}
F-FO-\arabic{FOcount} & Le SAUD peut gérer 100 connexions simultanées au maximum & \cellcolor{red!50}Important \\
\hline
\addtocounter{FOcount}{10}
F-FO-\arabic{FOcount} & \begin{itemize}
\item le SAUD sera développé en Javascript, PHP et MySQL
\item l'API sera développé en Javascript et PHP \end{itemize} & \cellcolor{blue!50}Secondaire \\
\hline
\addtocounter{FOcount}{10}
F-FO-\arabic{FOcount} & l'API sera aussi développée en JSP & \cellcolor{blue!50}Secondaire \\
\hline
\end{longtable}

\section{Exigences d'interface}

\newcounter{FIcount}

\begin{longtable}{|p{2cm}|p{10cm}|p{2.5cm}|}

% Header for the first page of the table
\multicolumn{1}{|>{\color{white}\columncolor{blue}}l|}{\bfseries{Reference}} & \multicolumn{1}{|>{\color{white}\columncolor{blue}}l|}{\bfseries{Fonctionalité}} & \multicolumn{1}{|>{\color{white}\columncolor{blue}}l|}{\bfseries{Priorité}}
\endfirsthead
% Header for next pages of the table
\multicolumn{1}{|>{\color{white}\columncolor{blue}}l|}{\bfseries{Reference}} & \multicolumn{1}{|>{\color{white}\columncolor{blue}}l|}{\bfseries{Fonctionalité}} & \multicolumn{1}{|>{\color{white}\columncolor{blue}}l|}{\bfseries{Priorité}}
\endhead

%This is the footer for all pages except the last page of the table...
\endfoot
%This is the footer for the last page of the table...
\endlastfoot

\hline
\addtocounter{FIcount}{10}
F-FI-\arabic{FIcount} & L'interface utilisateur/SAUD -> authentification
& \cellcolor{green!50}Indispensable \\
\hline
\addtocounter{FIcount}{10}
F-FI-\arabic{FIcount} & L'interface administrateur
& \cellcolor{red!50}Secondaire \\
\hline
\end{longtable}

\section{Exigences de qualité}

\newcounter{FQcount}

\begin{longtable}{|p{2cm}|p{10cm}|p{2.5cm}|}

% Header for the first page of the table
\multicolumn{1}{|>{\color{white}\columncolor{blue}}l|}{\bfseries{Reference}} & \multicolumn{1}{|>{\color{white}\columncolor{blue}}l|}{\bfseries{Fonctionalité}} & \multicolumn{1}{|>{\color{white}\columncolor{blue}}l|}{\bfseries{Priorité}}
\endfirsthead
% Header for next pages of the table
\multicolumn{1}{|>{\color{white}\columncolor{blue}}l|}{\bfseries{Reference}} & \multicolumn{1}{|>{\color{white}\columncolor{blue}}l|}{\bfseries{Fonctionalité}} & \multicolumn{1}{|>{\color{white}\columncolor{blue}}l|}{\bfseries{Priorité}}
\endhead

%This is the footer for all pages except the last page of the table...
\endfoot
%This is the footer for the last page of the table...
\endlastfoot

\hline
\addtocounter{FQcount}{10}
F-FQ-\arabic{FQcount} & La système sera livré pour le 31 mai 2012 & \cellcolor{green!50}Indispensable \\
\hline
\addtocounter{FQcount}{10}
F-FQ-\arabic{FQcount} & Un manuel d'utilisation sera livré avec le système
& \cellcolor{green!50}Indispensable \\
\hline
\addtocounter{FQcount}{10}
F-FQ-\arabic{FQcount} & La RFC du protocole sera fourni au client
& \cellcolor{green!50}Indispensable \\
\hline
\end{longtable}

\section{Exigences de réalisation}

\newcounter{FRcount}

\begin{longtable}{|p{2cm}|p{10cm}|p{2.5cm}|}

% Header for the first page of the table
\multicolumn{1}{|>{\color{white}\columncolor{blue}}l|}{\bfseries{Reference}} & \multicolumn{1}{|>{\color{white}\columncolor{blue}}l|}{\bfseries{Fonctionalité}} & \multicolumn{1}{|>{\color{white}\columncolor{blue}}l|}{\bfseries{Priorité}}
\endfirsthead
% Header for next pages of the table
\multicolumn{1}{|>{\color{white}\columncolor{blue}}l|}{\bfseries{Reference}} & \multicolumn{1}{|>{\color{white}\columncolor{blue}}l|}{\bfseries{Fonctionalité}} & \multicolumn{1}{|>{\color{white}\columncolor{blue}}l|}{\bfseries{Priorité}}
\endhead

%This is the footer for all pages except the last page of the table...
\endfoot
%This is the footer for the last page of the table...
\endlastfoot

\hline
\addtocounter{FRcount}{10}
F-FR-\arabic{FRcount} & Une API sera développée pour permettre aux sites
quelconques d'implanter le protocole d'authentification & \cellcolor{green!50}Indispensable \\
\hline
\end{longtable}


\end{document}
