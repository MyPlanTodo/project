%!TEX TS-program = xelatex
%!TEX encoding = UTF-8 Unicode
\documentclass[a4paper,11pt,french]{article}

\usepackage[latin1]{inputenc}
\usepackage[french]{babel}
\usepackage{chngpage}
\usepackage[colorlinks=true, linkcolor=black, urlcolor=blue]{hyperref}
\usepackage{fontspec,xltxtra,xunicode,color}
\usepackage{tabularx}
\usepackage[table]{xcolor}
\usepackage{fancyhdr}
\usepackage{longtable}
\usepackage{lastpage} %pour compter le nombre de pages

\definecolor{gris}{rgb}{0.95, 0.95, 0.95}

\hypersetup{breaklinks=true}


\addtolength{\hoffset}{-2cm}
\addtolength{\textwidth}{4cm}
\addtolength{\topmargin}{-2cm}
\addtolength{\textheight}{4cm}
\addtolength{\headsep}{0.8cm} 
\addtolength{\footskip}{-0.3cm}

%les structures de gestion de projet
\usepackage{res/structures} 


\def\projectName{Service d'authentification unique et décentralisé}
\def\docType{Analyse des risques}
\def\version{1.1}
\def\author{Giovanni \textsc{Huet}}
\def\dateStb{\today}
\def\checked{Lynda \textsc{Laceb}}
% \def\approved{Florent \textsc{Nicart}}



\begin{document}
\makeFirstPage
\clearpage
\vspace*{1cm}
%Tableau de mises à jour
\begin{center}
\textbf{\huge{MISES À JOUR}}\\
\vspace*{3cm}
	\begin{tabularx}{16cm}{|c|c|X|}
	\hline
	\bfseries{Version} & \bfseries{Date} & \bfseries{Modifications
réalisées}\\
	\hline
	0.1 & 30/11/2011 & Création\\
	\hline
	1.0 &  15/01/2012 & Relu par Lynda \textsc{Laceb} \\
	\hline
	1.1 &  14/03/2012 & Modification du risque 4\\
	\hline
	\end{tabularx}
\end{center}

\clearpage

\section{Risques}
\begin{flushleft}
\begin{small} 
\begin{tabularx}{18cm}{|c|c|p{2.28cm}|p{2cm}|c|c|c|c|}
\hline
Référence & Date & Description & Facteurs & Type & Probabilité & Gravité &
Criticité \\
\hline
R1 & 12-01-2012 & 
Nombre d'anomalies logicielles des versions précédentes
à corriger importante. (Anomalies apparues lors de la phase de tests)
& Code non performant. (Certains cas peuvent être mal traités)
& Technique & Faible & Importante & 8 \\
\hline
R2 & 12-01-2012 &
Retard pris dans la réalisation du projet
& Multiples projets et examens menant à un retard dans le planning de
développement ou membre de l’équipe absente pendant une certaine période
& Organisationnel & Moyenne & Importante & 9 \\
\hline
R3 & 12-01-2012 &
Gestion du projet durant les vacances 
& L'équipe ne peut pas faire le point sur l'avancement
pendant les vacances (Déplacement, non accès à internet...)
& Organisationnel & Faible & Moyenne & 5 \\
\hline
R4 & 12-01-2012 & 
Apprentissage ou remise à niveau concernant un langage de
programmation utilisé
& Langage de programmation pratiqué il y a plusieurs années, ou pas du tout pour
certain
& Technique & Faible & Moyenne & 5 \\
\hline
R5 & 12-01-2012 & Problème lors de la rédaction de certains documents à
présenter au client (M. Florent NICART)
& Une mauvaise rédaction peut engendrer une mauvaise compréhension entre le
fournisseur et le client
& Organisationnel & Faible & Faible & 4 \\
\hline
\end{tabularx}

\end{small}
\end{flushleft}
\section{Détail des risques}

\subsection*{R1}
Dans le développement des versions de notre projet (ou une partie de celui-ci),
la dernière version est en phase de test réalisée par les «clients/testeurs ».
Si la phase de tests est réalisée avec succès, le groupe pourra se concentrer
sur la nouvelle version à établir. Cependant si des anomalies surviennent, le
groupe sera divisé de telle sorte à ce qu’un groupe s’occupe pleinement de la
correction de ces anomalies et les autres, quant à eux, continueront sur
l’avancement de la nouvelle version.

\subsection*{R2}
Dans l’objectif d’éviter tout risque de retard dans le développement de notre
projet, il a été établi un plan de développement avec des délais pour chacune
des versions. Un suivi hebdomadaire de l’avancement a également été mis en place
afin de faire un bilan des actions qui deviennent prioritaires et donc une revue
ciblée du plan de développement.

\subsection*{R3}
Afin de s’assurer du bon avancement du projet à l’approche des vacances, une
distribution des différentes tâches sera effectuée juste avant le départ en
vacances. Chaque membre du groupe devra faire en sorte de respecter ses
engagements pour la fin des vacances. D’autres part des moyens de communication
pourront être mis en place (Réunion téléphonique avec un logiciel adapté
(skype…), la création d’un dossier partagé sur la liste de diffusion du site
\texttt{www.univ-rouen.fr} afin que tous les membres du groupe puisse avoir
accès aux différents documents).

\subsection*{R4}
Pour la réalisation du projet, la connaissance de différents langages de
programmation est indispensable (PHP, JavaScript, MySQ...). Un apprentissage ou
une remise à niveau, selon le cas, est donc nécessaire. Des recherches sur des
sites spécialisés, ou des livres à la bibliothèque seront faite par un ou
plusieurs membres du groupe afin de gagner du temps pour la réalisation du
projet.
\paragraph{}
14-03-2012 $\Rightarrow$ Des recherches ont été effectuées afin de trouver des
documents concernant le langage de programmation php, et partagés par les
différents membres du groupe.

\subsection*{R5}
La fin du projet se concrétise par la réalisation d’un rapport. Pour cela, une
rédaction minutieuse des documents sera demandée pour chaque membre du groupe.
Une documentation exhaustive des différentes librairies utilisées, un mode
d’emploi, etc. seront effectués. Pour éviter tout quiproquo, des réunions avec
le client pourront être envisagés.

\end{document}
