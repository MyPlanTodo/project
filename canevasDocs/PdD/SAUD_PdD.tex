%!TEX TS-program = xelatex
%!TEX encoding = UTF-8 Unicode
\documentclass[a4paper,11pt,french]{article}

\usepackage[latin1]{inputenc}
\usepackage[french]{babel}
\usepackage{chngpage}
\usepackage{graphicx, amssymb, color, listings}
\usepackage[colorlinks=true, linkcolor=black, urlcolor=blue]{hyperref}
\usepackage{fontspec,xltxtra,xunicode,color}
\usepackage{tabularx}
\usepackage[table]{xcolor}
\usepackage{fancyhdr}
\usepackage{longtable}
\usepackage{tikz}
\usepackage{pgflibraryshapes}
\usepackage{lastpage} %pour compter le nombre de pages

\definecolor{gris}{rgb}{0.95, 0.95, 0.95}

\hypersetup{breaklinks=true}


\addtolength{\hoffset}{-2cm}
\addtolength{\textwidth}{4cm}
\addtolength{\topmargin}{-2cm}
\addtolength{\textheight}{4cm}
\addtolength{\headsep}{0.8cm} 
\addtolength{\footskip}{-0.3cm}

%les structures de gestion de projet
\usepackage{res/structures} 


\def\projectName{Service d'authentification unique et décentralisé}
\def\docType{Plan de développement}
\def\version{1.0}
\def\author{Florian \textsc{Guilbert}}
\def\dateStb{\today}
\def\checked{Giovanni \textsc{Huet}}



\begin{document}
\makeFirstPage
\clearpage
\vspace*{1cm}
%Tableau de mises à jour
\begin{center}
\textbf{\huge{MISES À JOUR}}\\
\vspace*{3cm}
	\begin{tabularx}{16cm}{|c|c|X|}
	\hline
	\bfseries{Version} & \bfseries{Date} & \bfseries{Modifications réalisées}\\
	\hline
	0.1 & 30/11/2011 & Création\\
	\hline
	 1.0 & 15/01/2012 & Relecture par Giovanni \textsc{Huet}\\
	\hline
	1.1 & 29/03/2012 & Modification de l'organigramme des tâches (log de
connexion) \\
	\hline
	1.2 & 22/04/2012 & Modification du planning \\
	\hline
	\end{tabularx}
\end{center}

\clearpage
\tableofcontents
\clearpage

\section{Contexte du projet}
Ce projet propse une solution d'enregistrement décentralisée pour le web. 
En effet, la gestion des identifiants/mots de passes et l'enregistrement
auprès de service est un problème récurrent du web. 

Notre solution consiste à mettre en place un site web qui permettra 
à ses utilisateurs de les authentifier auprès de lui et ensuite garantir
leur authentification auprès de tous les autres sites, en admettant que ceux-ci
proposent un service d'authentification externe. Ce site web se basera sur un
protocole que nous mettrons au point, au préalable.

Dans cet analyse du projet, nous admettons que nous allons pouvoir trouver un
protocole sécurisé répondant aux objectifs du client, car dans le cas contraire,
il sera impossible de développer le logiciel. Il faudrait uniquement produire un
document prouvant pourquoi un tel protocole est impossible à réaliser.

Ce projet a été proposé par M. Florent \textsc{Nicart} dans le cadre de notre
formation en Sécurité des Systèmes Informatiques à l'Université de Rouen.

\paragraph{}
Étant un projet universitaire, ce travail a pour but de nous apprendre
à gérer un projet, de sa partie analyse jusqu'à sa complète réalisation. Pour
cette même raison, le projet devra être impérativement terminé avant la fin du
second semestre.

\section{Documents applicables et de références}
\begin{itemize}
 \item Intitulé du sujet du projet;
 \item spécification technique de besoin;
 \item cahier des recettes;
 \item document d'architecture logicielle;
 \item analyse des risques.
\end{itemize}

\section{Terminologie et sigles utilisés}
\begin{description}
	\item[AdR :] Analyse des Risques;
	\item[CdR :] Cahier de Recettes;
	\item[DAL :] Document d'Architecture Logicielle;
	\item[PdD :] Plan de développement;
	\item[STB :] Spécification Technique de Besoin;
	\item[SAUD :] Site d'Authentification Unique et Décentralisé;
	\item[API :] (Application Interface Programming), ensemble de fonctions
et de modules permettant de réaliser un but précis;
	\item[IHM :] Interface homme-machine, (interface graphique);
	\item[RFC :] (Requests for comments) est un document officiel décrivant
les aspects techniques d'Internet
	ou de différents matériels informatiques (routeurs, serveur DHCP...);
\end{description}

\section{Méthodologie et développement}
Le développement du projet suit le schéma suivant : \\

\begin{center}
\begin{tikzpicture}
[scale=1]
% style des nœuds
\tikzstyle{es}=[rectangle,draw,rounded corners=4pt,fill=blue!25]
% style des flèches
\tikzstyle{suite}=[->,>=stealth,thick,rounded corners=4pt]
% placement des nœuds
\node[es] (spe) at (6,8) {Spécification};
\node[es] (con) at (6,6) {Conception};
\node[es] (ana) at (10,4) {Développement};
\node[es] (dev) at (6,1) {Test};
\node[es] (val) at (2,4) {Validation};
% Placement des flèches
\draw[suite] (spe) -- (con);
\draw[suite] (con) -- (ana);
\draw[suite] (ana) -- (dev);
\draw[suite] (dev) -- (val);
\draw[suite] (val) -- (spe) ; 
\draw[suite] (val) -- (con) ; 

\end{tikzpicture}
\end{center}

\paragraph{Spécification}
\begin{itemize}
 \item Déterminer les objectifs;
 \item définir les contraintes;
 \item évaluer les risques.
\end{itemize}
Responsabilité : clients, testeurs et programmeurs.

\paragraph{Conception}
\begin{itemize}
 \item Définir les composants à développer.
\end{itemize}
Responsabilité : testeurs et programmeurs.

\paragraph{Développement}
\begin{itemize}
 \item Développement des composants;
 \item tests unitaires sur ces composants.
\end{itemize}
Responsabilité : programmeurs.

\paragraph{Test}
\begin{itemize}
 \item Tester des scénarii de tests.
\end{itemize}
Responsabilité : testeurs.

\paragraph{Validation}
\begin{itemize}
 \item L'étape de validation détermine si la version développée propose bien les
fonctionnalités attendues et dans ce cas valide la version du logiciel.
Le travail pour la prochaine version peut commencer;
 \item s'il n'y pas validation de la version, alors c'est aux testeurs de
décider s'il faut repasser l'étape de spécification ou s'il faut retourner
directement à l'étape de conception;
 \item dans le cas d'un succès d'une validation, une version du logiciel peut 
être livrée.
\end{itemize}
Responsabilité : clients et testeurs.

\paragraph{}
Cette méthodologie de développement suit la méthode XP (eXtreme Programming). 
Durant le développement de ce projet, il y aura trois itérations.
\begin{enumerate}
 \item Conception théorique du protocole sécurisé entre un SAUD et un site 
quelconque;
 \item développement de la gestion utilisateur du SAUD (site
d'authentification unique et décentralisée);
 \item implantation du protocole, coté serveur (sur le SAUD) et développement 
d'une API permettant d'implanter le coté client du protocole sur un site
quelconque. 
\end{enumerate}
Chacune des ces étapes comportent des modules qui seront développés en binôme
afin d'une part, de faire en sorte que le plus de programmeurs possibles
maitrisent le code source et d'autre part qu'une absence ne nuise pas au
processus de développement.

\section{Organisation et responsabilités}

\begin{center}
\begin{tikzpicture}
[scale=1]
% style des nœuds
\tikzstyle{coach}=[rectangle,draw,rounded corners=2pt,fill=blue!25]
\tikzstyle{prog}=[rectangle,draw,rounded corners=2pt,fill=red!25]
\tikzstyle{cliTest}=[rectangle,draw,rounded corners=2pt,fill=green!25]
% style des flèches
\tikzstyle{suite}=[->,>=stealth,thick,rounded corners=4pt]
% placement des nœuds
\node[coach] (coach) at (5,7.5) {\begin{tabular}{c}
                             Coach/Tracker \\ \\
			    Florian \textsc{Guilbert}
                            \end{tabular}};
\node[prog] (prog1) at (-.5,4) {\begin{tabular}{c}
                             Programmeur \\ \\
			    Ouissem \textsc{Hamdani}
                            \end{tabular}};
\node[prog] (prog2) at (3.3,4) {\begin{tabular}{c}
                             Programmeur\\ \\
			    Giovanni \textsc{Huet}
                            \end{tabular}};
\node[cliTest] (test1) at (7,4) {\begin{tabular}{c}
                             Client/Testeur \\ \\
			    Lynda \textsc{Laceb}
                            \end{tabular}};
\node[cliTest] (test2) at (10.35,4) {\begin{tabular}{c}
                             Client/Testeur \\ \\
			    Safae \textsc{Rebani}
                            \end{tabular}};
% Placement des flèches
\draw (coach) -- (5, 5.5);
\draw (prog1) |- (5, 5.5);
\draw (prog2) |- (5, 5.5);
\draw (test1) |- (5, 5.5);
\draw (test2) |- (5, 5.5);

\end{tikzpicture}
\end{center}


\paragraph{Coach:}
\begin{itemize}
 \item Garant du processus et de la méthodologie;
 \item vérifie que chacun joue son rôle;
 \item organise et anime les réunions et les séances de planifications;
 \item valide les orientations techniques.
\end{itemize}

\paragraph{Client:}
\begin{itemize}
 \item Spécifie les fonctionnalités du logiciel;
 \item communique les informations utiles aux développeurs et reçoit leurs 
``feedback'';
 \item définit les fonctionnalités à partir de scénarii d'utilisations;
 \item spécifie les tests de recette.
\end{itemize}

\paragraph{Programmeur:}
\begin{itemize}
 \item Responsables de la production du code;
 \item conçoit pour assurer la pérennité du code;
 \item teste pour assurer la qualité du code;
 \item émet et révise des estimations de charge.
\end{itemize}

\paragraph{Testeur:}
\begin{itemize}
 \item Conçoit et réalise les tests de recettes définit par le client;
 \item recherche l'automatisation des tests;
 \item développe les outils de tests nécessaires et les scripts à exécuter;
 \item témoigne de l'avancement du projet.
\end{itemize}

\paragraph{Tracker:}
\begin{itemize}
 \item Assure le suivi du planning;
 \item cherche a détecter les difficultés le plus tôt possible et en informe
le coach.
\end{itemize}

\section{Organigramme des tâches}
Le projet va être organisé en trois versions (une par itération), le schéma
suivant décrit les tâches à développer pour chaque version :  

\begin{center}
\begin{tikzpicture}
[scale=1]
% style des nœuds
\tikzstyle{projet}=[rectangle,draw,rounded corners=2pt,fill=blue!25]
\tikzstyle{analyse}=[rectangle,draw,rounded corners=2pt,fill=red!25]
\tikzstyle{ver1}=[rectangle,draw,rounded corners=2pt,fill=green!25]
\tikzstyle{ver2}=[rectangle,draw,rounded corners=2pt,fill=yellow!25]
\tikzstyle{ver3}=[rectangle,draw,rounded corners=2pt,fill=orange!25]

% style des flèches
\tikzstyle{suite}=[->,>=stealth,thick,rounded corners=4pt]
% placement des nœuds
\node[projet] (projet) at (3.5, 6.5) {Service d'authentification unique,
décentralisé};
\node[analyse] (analyse) at (-2,4) {
  \begin{tabular}{c}Analyse et \\ spécification\end{tabular}};
\node[ver1] (ver1) at (1.8,4) {Version 0.1};
\node[ver2] (ver2) at (5,4) {Version 0.2};
\node[ver3] (ver3) at (9.3,4) {Version finale};
\node[analyse] (stb) at (-.5,2.5) {STB};
\node[analyse] (dal) at (-.5,1.5) {DAL};
\node[analyse] (cdr) at (-.5,0.5) {CdR};
\node[analyse] (pdd) at (-.5,-0.5) {PdD};
\node[analyse] (adr) at (-.5,-1.5) {AdR};
\node[ver1] (proto) at (1.8, 2.5) {
    \begin{tabular}{c} 
    Description \\ 
    du protocole
    \end{tabular}};
\node[ver2] (ihm) at (7.2, 2.5) {IHM};
\node[ver2] (bdd) at (7.2, 1) {\begin{tabular}{c}Conception\\de la base\\de
données\end{tabular}};
\node[ver2] (model) at (7.2, -1) {\begin{tabular}{c}Conception\\du
model\end{tabular}};
\node[ver2] (fusion) at (7.2, -2.5) {\begin{tabular}{c}Fusion\\
IHM-model\end{tabular}};
\node[ver2] (admin) at (7.2, -4.2) {\begin{tabular}{c}Implantation du 
\\log de connexion\\(optionnelle)
                                    \end{tabular}};
\node[ver2] (test1) at (7.2, -6) {Tests};

\node[ver3] (server) at (11.6, 2) {\begin{tabular}{c}Implantation \\du
protocole\\ coté serveur\end{tabular}};
\node[ver3] (api) at (11.6, 0) {\begin{tabular}{c}Réalisation \\d'une API pour\\
le client\end{tabular}};
\node[ver3] (client) at (11.6, -2) {\begin{tabular}{c}Développement\\d'un site
client\end{tabular}};
\node[ver3] (paquetage) at (11.6, -4)
{\begin{tabular}{c}Système\\d'installation\\ par paquetage\end{tabular}};
\node[ver3] (test2) at (11.6, -6) {Tests};

% Placement des flèches
\draw (projet) -- (3.5, 5.5);
\draw (analyse) |- (3.5, 5.5);
\draw (ver1) |- (3.5, 5.5);
\draw (ver2) |- (3.5, 5.5);
\draw (ver3) |- (3.5, 5.5);
\draw (analyse) |- (stb);
\draw (analyse) |- (dal);
\draw (analyse) |- (cdr);
\draw (analyse) |- (pdd);
\draw (analyse) |- (adr);
\draw (ver1) -- (proto);
\draw (ver2) |- (ihm);
\draw (ver2) |- (bdd);
\draw (ver2) |- (model);
\draw (ver2) |- (fusion);
\draw (ver2) |- (admin);
\draw (ver2) |- (test1);
\draw (ver3) |- (server);
\draw (ver3) |- (api);
\draw (ver3) |- (client);
\draw (ver3) |- (paquetage);
\draw (ver3) |- (test2);
\end{tikzpicture}
\end{center}

\section{Évalutation du projet et dimensionnement des moyens}

\paragraph{Version 0.1}Cette première version est un peu particulière, car elle
ne modifie en rien le logiciel. Il s'agit pour cette version d'établir un
protocole permettant au produit de fonctionner de manière totalement sécurisée.
C'est donc une partie intégralement théorique.

Aucun besoin de ressources ou de moyens.

\paragraph{Version 0.2}  Le but de cette version est de poser les
bases du logiciel. Par conséquent, il faut développer d'une part l'interface
graphique afin de pouvoir tester le logiciel et d'autre part configurer la base
de données afin de pouvoir, par la suite utiliser des jeux de données pour les
tests.

Dans un second temps et pour pouvoir proposer une version fonctionnelle, il 
faudra implanter une gestion d'utilisateur dans le but qu'un utilisateur puisse
se créer un compte et le gérer.

Pour le développement, les seuls besoins en moyens sont de travailler sur des 
machines disposant de serveur web Apache.

Pour le test, aucune ressource ni moyen requis, au contraire, pour tester la 
compatibilité du produit, il faudra tester avec le plus de technologies web
différentes (différents navigateurs, systèmes d'exploitation, ...).

\paragraph{Version 1.0, finale} Cette version ajoute à la version 0.2 
l'implantation du protocole établie en 0.1.

Le contexte de développement et de tests sont similaires à celle de la version 
0.2.

\section{Planning général}
\begin{center}
\begin{figure}[h]
\includegraphics[scale=0.55]{plan.png} 
\caption{Diagramme de Gant, tâches}
\end{figure}
% \newpage
\begin{figure}[h]
\includegraphics[scale=0.60]{gant.png} 
\caption{Diagramme de Gant, graphes}
\end{figure}
\end{center}
Sur ce graphe, on peut observer trois couleurs, chaque couleur désigne une
version du logiciel différente (rouge : 0.1, bleu-vert : 0.2, bleu : 1.0
(finale).
\section{Procédés de gestion}

\subsection{Gestion de la documentation}
Il y aura deux documentations livrables : 
\begin{itemize}
 \item Le manuel d'utilisation rédigé par l'équipe testeur;
 \item La RFC du protocole rédigé par l'équipe.
\end{itemize}
Les documents utilisées tout au long de la réalisation de ce projet sont : 
\begin{itemize}
 \item Analyse des risques (AdR) rédigé par Giovanni \textsc{Huet} et relu par
Lynda
\textsc{Laceb}
 \item Cahier des recettes (CdR) rédigé par Lynda \textsc{Laceb} et Safae
\textsc{Rebani} et relu par Ouissem \textsc{Hamdani};
 \item Document d'architecture logicielle (DAL) rédigé par Ouissem
\textsc{Hamdani} et relu par Safae \textsc{Rebani};
 \item Plan de développement (PdD) rédigé par Florian \textsc{Guilbert} et relu
par Giovanni \textsc{Huet}.
 \item Spécification technique des besoins (STB) rédigé par Lynda
\textsc{Laceb} et Safae \textsc{Rebani} et relue par Florian \textsc{Guilbert};
\end{itemize}




\subsection{Gestion des configurations}
Toutes les versions du programmes seront gérées par le logiciel de versionnage 
\texttt{Subversion} (SVN). Toutes les machines servant à développer devront
avoir les même configurations des logiciels Apache, PHP et MySQL. Il n'y a pas
de configurations particulières pour les machines visant à tester le logiciel
hormis de disposer d'une connexion internet et d'un navigateur (à jour).

\section{Revues et points clefs}
\paragraph{Revue 1:}
La première revue du projet aura lieu le 19 janvier 2012 lors d'une soutenance
ayant pour but de présenter le projet aux responsables de la matière ``Gestion
de projet'' et aux clients. Il est demandé auparavant d'envoyer aux examinateurs
de cette soutenance une version de chaque document précités : (AdR, CdR, DAL,
PdD, STB).

\section{Procédures de suivi d'avancement}
Le suivi d'avancement du projet sera effectué par l'intermédiaire de réunions
régulières et de jalons. Ceux-ci permettront de contrôler l'état d'avancement
de chaque module et de réagir rapidement à tout écart avec les prévisions en
réorganisant le planning.

\paragraph{}
Il y aura par conséquent une réunion chaque semaine (hormis durant les
vacances) entre les membres du groupe. À chaque validation, il y aura
présentation au client du travail effectué.

Des réunions avec le client pourront aussi être organisées si le client en fait
la demande ou bien qu'un membre du groupe souhaite bénéficier de son avis.

Pour chaque réunion, un compte-rendu sera réalisé et placé à disposition du
groupe par l'intermédiaire de Subversion.

\end{document}
